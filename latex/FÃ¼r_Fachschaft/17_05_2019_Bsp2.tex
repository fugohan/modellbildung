\newpage
\noindent
\textbf{Beispiel 2}\\ \\
a)\\ \\
Mit der Formel \textit{Wärmestromdichte und Wärmedurchgangskoeffizient bei mehrschichtigem Wandaufbau} aus der Formelsammlung lautet der Wärmedurchgangskoeffizient 
\[
	k = \frac{1}{\frac{1}{\alpha_1} + \frac{1}{\alpha_2} + \frac{1}{\alpha} + \frac{l_1}{\lambda_1} + \frac{l_2}{\lambda_2}}
\]
b)\\ \\
Mit 
\[
	P = \dot{Q} = \tau\omega
\]
und
\[
	\dot{Q} = Ak(T - T_\infty)
\]
lautet die maximale Drehgeschwindigkeit
\[
	\omega_{max} = (T_{max} - T_\infty)\frac{2Ak}{\tau_b}
\]
c)\\ \\
Die Dauer bis die Bremsscheibe auf die erwartete Temperatur erwärmt beträgt
\[
	t = -\frac{mc_p}{2kA}\left(1 - (T_0 - T_\infty)\frac{2kA}{P}\right)	
\]
Gelöst durch Lösen einer Differentialgleichung mit Einsetzen der Randbedingung. \\ \\
\textit{Hinweis: Der Rechenweg zum Bestimmen dieser Zeit wäre zu aufwendig gewesen, um sie hier genauer zu beschreiben.} \\ \\