\textbf{Beispiel 4} \\ \\
a) \\ \\
Aus der Zeichnung kann man direkt 
\[
	\textbf{p}_L = \begin{bmatrix}
		s + \sin\varphi \\
		- l \cos\varphi
	\end{bmatrix}
\]
ablesen. Um $\dot{\textbf{p}}_L$ zu bestimmen, muss man $\textbf{p}_L$ nach den Freiheitsgraden $s$ und $\varphi$ ableiten.\\
Geschwindigkeit der Lastposition
\[
	\dot{\textbf{p}}_L = \begin{bmatrix}
		\dot{s} + l\cos\varphi\dot{\varphi} \\
		l\sin\varphi\dot{\varphi}
	\end{bmatrix}
\]
Der Betrag dieses Vektors lautet
\[
	||\dot{\textbf{p}}_L||^2 = \dot{s}^2 + l^2\dot{\varphi}^2 + 2\dot{s}l\dot{\varphi}\cos\varphi
\]
Die kinetische Energie ergibt sich zu
\[
	T = \frac{1}{2} m_K \dot{s}^2 + \frac{1}{2}m_L||\dot{\textbf{p}}_L||^2 + \frac{1}{2} \Theta_{zz}\dot{\varphi}^2
\]
und die potentielle Energie
\[
	V = -m_lgl\cos\varphi + V_0
\]
mit 
\[
	V_0 = m_Lgl
\]
zu
\[
	V = m_Lgl(1 - \cos\varphi) 
\]
b) \\ \\
Um die Bewegungsgleichungen zu bestimmen wird als erstes die Lagrange-Funktion benötigt welche lautet
\begin{align*}
	L &= T - V = \frac{1}{2} m_K \dot{s}^2 + \frac{1}{2}m_L||\dot{\textbf{p}}_L||^2 + \frac{1}{2} \Theta_{zz}\dot{\varphi}^2 - m_Lgl(1 - \cos\varphi) \\
	&=  \frac{1}{2} m_K \dot{s}^2 + \frac{1}{2}m_L (\dot{s}^2 + l^2\dot{\varphi}^2 + 2\dot{s}l\dot{\varphi}\cos\varphi)  + \frac{1}{2} \Theta_{zz}\dot{\varphi}^2 - m_Lgl(1 - \cos\varphi)
\end{align*}
Euler-Lagrange-Formalismus
\[
	\frac{d}{dt}\frac{\partial L}{\partial \dot{\textbf{q}}} - \frac{\partial L}{\partial \textbf{q}} = \textbf{f}_{np}
\]
\newpage
\noindent
Die seperaten Ableitungen
\begin{align*}
	\frac{\partial L}{\partial \textbf{q}} &= \begin{bmatrix}
		0 \\
		-m_Ll\sin\varphi(\dot{s}\dot{\varphi} + g)
	\end{bmatrix}
	\\
	\frac{\partial L}{\partial \dot{\textbf{q}}} &= \begin{bmatrix}
		(m_K + m_L)\dot{s} + m_Ll\dot{\varphi}\cos\varphi \\
		(\Theta_{zz} + m_L)\dot{\varphi} + m_Ll\dot{s}\cos\varphi
	\end{bmatrix}
	\\
	\frac{d}{dt}\frac{\partial L}{\partial \dot{\textbf{q}}} &= \begin{bmatrix}
		(m_k + m_L)\ddot{s} + m_Ll\ddot{\varphi}\cos\varphi - m_Ll\dot{\varphi}^2\sin\varphi \\
		(\Theta_{zz} + m_L)\ddot{\varphi} + m_Ll\ddot{s}\cos\varphi - m_Ll\dot{s}^2\sin\varphi
	\end{bmatrix}
\end{align*}
Die externen Kräfte können direkt aus der Angabe ablesen werden und lauten
\[
	\textbf{f}_{np} = \begin{bmatrix}
		-d_K\dot{s} + F \\ 
		0
	\end{bmatrix}
\]
c) \\ \\
Aus der zweiten Zeile der Bewegungsgleichungen aus Punkt b) erhält man die Bewegungsgleichung der haftenden Laufkatze. Dadurch lautet diese
\[
	(\Theta_{zz} + m_L)\ddot{\varphi} + m_Llg\sin\varphi = 0
\]
d) \\ \\
Als erstes müssen die externen Kräfte um die Haftkraft $F_H$ erweitert werden. Daher lauten diese dann
\[
	\textbf{f}_{np} = \begin{bmatrix}
	-d_K\dot{s} + F + F_H\\ 
	0
	\end{bmatrix}
\]
Betrachtet man laut Hinweis nur die erste Zeile der Bewegungsgleichungen aus Punkt b) mit $\dot{s} = \ddot{s} = 0$ erhält man
\[
	m_Ll\ddot{\varphi}\cos\varphi - m_Ll\dot{\varphi}^2\sin\varphi = F + F_H
\] 
Formt nun auf $F_H$ um und ermittelt man den Betrag erhält man
\[
	|F_H| = |m_Ll\ddot{\varphi}\cos\varphi - m_Ll\dot{\varphi}^2\sin\varphi - F|
\]
Bringt man nun schließlich diese Gleichung auf die Form der Haftreibung erhält man schlussendlich die Haftbedingung
\[
	|F_H| = |m_Ll\ddot{\varphi}\cos\varphi - m_Ll\dot{\varphi}^2\sin\varphi - F| \leq \mu F_N
\]
Mit
\[
	\ddot{\varphi} = - \frac{m_Llg\sin\varphi}{\Theta_{zz} + m_Ll^2}
\]
aus Punkt c), erhält man die vereinfachte Haftbedingung
\begin{align*}
	\biggl|m_Ll\frac{m_Llg\sin\varphi}{\Theta_{zz} + m_Ll^2}\cos\varphi - m_Ll\dot{\varphi}^2\sin\varphi - F\biggl| \leq \mu F_N \\ \\
	\biggl|\frac{gm^2_Ll^2\sin\varphi\cos\varphi}{\Theta_{zz} + m_Ll^2} + m_Ll\dot{\varphi}^2\sin\varphi + F\biggl| \leq \mu F_N
\end{align*}