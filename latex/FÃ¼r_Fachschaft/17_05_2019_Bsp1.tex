\textbf{Beispiel 1}\\
a)\\ \\
Die Gleichgewichtsbedingungen lauten
\begin{align*}
	\textbf{e}_x &: f_{Ax} - f_l\\
	\textbf{e}_y &: -f_m + f_{Ay} - f_p + f_{By} = 0\\
	\textbf{e}_z &: F_{By}(l_1 + l_2) + f_lh - f_ml_1 - f_p(l_1 + l_3\cos(\beta)) = 0
\end{align*}
Als Drehpunkt wurde der Punkt A verwendet. Aus diesen Gleichungen folgen die Kräfte
\begin{align*}
	f_{By} &= \frac{-f_lh + f_ml_1 + f_p(l_1 + l_3\cos(\beta))}{l_1 + l_2} \\
	f_{Ay} &= f_m + f_p - f_{By} \\
	f_{Ax} &= f_l
\end{align*}
b)\\ \\
Aus der Momentengleichung
\[
	f_{Ax}r_3 - f_{Ko}r_2 = 0
\]
folgt die Kettenkraft
\[
	f_{Ko} = \frac{f_{Ax}r_3}{r_2}
\]
c)\\ \\
Aus der Momentengleichung
\[
	-f_pl_3\cos(\beta) + f_{Ko}r_1 = 0
\]
folgt die Pedalkraft
\[
	f_p = \frac{f_{Ax}r_3r_1}{r_2l_3\cos(\beta)}
\]
d)\\ \\
Durch Umformen sämtlicher Gleichgewichtsbedingungen und Einsetzen der Haftreibung lautet die Gleichung, bei der das Hinterrad bei konstanter Geschwindigkeit nicht rutscht
\[
	\mu \geq \frac{|\frac{1}{2}\rho c_wAv^2|(l_1 + l_2)}{f_p(l_2 - l_3\cos(\beta)) + f_ml_2 + \frac{1}{2}\rho c_wAv^2h}
\]
mit 
\[
	f_l = \frac{1}{2}\rho c_wAv^2
\]