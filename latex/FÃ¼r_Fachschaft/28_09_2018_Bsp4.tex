\textbf{Beispiel 4}\\ \\
a)\\ \\
Die Differentialgleichung für das Heißgetränk lautet
\begin{align*}
	\rho c_p dh \frac{\partial T_H(t)}{\partial t} &= -(\dot{q} + \dot{q}_{HU}) \\
	\frac{\partial T_H(t)}{\partial t} &= \left(\frac{4k_bh + \alpha_{HU}d}{\rho c_p dh}\right)(T_h(t) - T_U)
\end{align*}
b)\\ \\
Die Lösung dieser Differentialgleichung lautet
\[
	T_H(t) = T_U + (T_0 - T_U)e^{-\left(\frac{4k_bh + \alpha_{HU}d}{\rho c_p dh}\right)t}
\]
c)\\ \\
Die spezifische Wärmekapazität des Gemisches lautet
\[
	c_{p,G} = \frac{c_{p,M}\beta + c_{p,H}}{1 - \beta}
\]
d) \\ \\
Die Temperatur des Gemisches lautet
\[
	T_G = \frac{c_{p,M}\beta T_M + c_{p,H}T_H}{c_{p,M} + c_{p,H}}
\]
e)\\ \\
i)\\
Durch Einsetzen der Bedingungen in die Lösung der Differentialgleichung folgt für $t_1$
\[
	t_1 = -\frac{\rho(c_{p,M}\beta + c_{p,H})dh}{4k_Bh + \alpha_{HU}d}\ln\left(\frac{(c_{p,M}\beta + c_{p,H})T_T - c_{p,M}\beta T_M + c_{p,H}T_U}{c_{p,H}(T_0 - T_U)}\right)
\]
ii)\\
Durch Einsetzen der Bedingungen in die Lösung der Differentialgleichung folgt für $t_2$
\[
	t_2 = -\frac{\rho c_{p,H}dh}{4k_Bh + \alpha_{HU}d}\ln\left(\frac{(c_{p,M}\beta + c_{p,H})T_T - c_{p,M}\beta T_M + c_{p,H}T_U}{c_{p,H}(T_0 - T_U)}\right)
\]
f)\\ \\
Durch Einsetzen der Bedingungen aus der Angabe ergibt sich für das Mischverhältnis $\beta > 0$.