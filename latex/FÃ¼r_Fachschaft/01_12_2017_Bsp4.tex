\newpage
\noindent
\textbf{Beispiel 4}\\ \\
a)\\ \\
Die Wärmestromdichte aufgrund der freien Konvektion lautet
\[
	\dot{q}_2 = -\alpha(T_2 - T_\infty)
\]
Der Wärmestrom der durch das Gasgemisch hervorgeht lautet
\[
	\dot{q}_2 = \frac{1}{d_1\pi}\dot{q}_G^\circ
\]
Somit ergibt sich für die gesuchte Temperatur
\begin{align*}
	-\alpha(T_2 &- T_\infty) = \frac{1}{d_2\pi}\dot{q}_G^\circ \\
	T_2 &= \frac{1}{d_1\pi\alpha}\dot{q}_G^\circ + T_\infty
\end{align*}
b)\\ \\
Die Sichtfaktoren lauten für dieses System
\begin{align*}
	F_{11} &= 0 \quad A_1 \text{ist konvex} \\
	F_{12} &= 1 - F_{11} = 1 \\
	F_{21} &= \frac{A_1}{A_2}F_{12} = \frac{d_1}{d_2} = D \\
	F_{22} &= 1 - F_{21} = 1 - D
\end{align*}
Damit lautet die Sichtfaktormatrix
\[
	\textbf{F} = \begin{bmatrix}
		0 & 1 \\
		D & 1 - D
	\end{bmatrix}
\]
c)\\ \\
Mit der Formel für die Nettowärmestromdichte aus der Formelsammlung ergibt sich der Zusammenhang
\[
	\begin{bmatrix}
		\dot{q}_1 \\
		\dot{q}_2
	\end{bmatrix}
	=
	\frac{\sigma\varepsilon_2}{\varepsilon_2(1 - D) + 1}
	\begin{bmatrix}
		1 & -1 \\
		-D & D
	\end{bmatrix}
	\begin{bmatrix}
		T_1^4 \\
		T_2^4
	\end{bmatrix}
\]
d)\\ \\
Relativ analog zu Punkt a) ergibt sich
\begin{align*}
	\dot{q}_1 &= k(T_1^4 - T_2^4), \, \text{mit} \, k = \frac{\sigma\varepsilon_2}{\varepsilon_2(1 - D) + 1} \\
	\dot{q}_1 &= \frac{1}{d_1\pi}\dot{q}^\circ \\
	T_1 &= \left(\frac{1}{kd_1\pi}\dot{q}^\circ + T_2^4\right)^{\frac{1}{4}}
\end{align*}