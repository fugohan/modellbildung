\newpage
\noindent
\textbf{Beispiel 4}\\ \\
a)\\ \\
Sämtliche Sichtfaktoren wurden mithilfe der Summationsregel und dem Reziproitätsgesetz ermittelt. Da der Zug einen konvexen Körper darstellt, lautet der entsprechende Sichtfaktor auch
\[
	F_{z,z} = 0
\]
Die restlichen Sichtfaktoren lauten somit
\begin{align*}
	F_{z,t} &= 1 - F_{z,z} = 1 \\
	F_{t,z} &= \frac{A_z}{A_t}F_{z,t} = \frac{1}{2} \\
	F_{t,t} &= 1 - F_{t,z} = 
\end{align*}
Damit lautet die Sichtfaktormatrix
\[
	\textbf{F} = \begin{bmatrix}
		\frac{1}{2} & \frac{1}{2} \\
		1 & 0
	\end{bmatrix}
\]
b) \\ \\
Durch Auswerten des Formalismus für die Nettowärmestromdichte ergibt sich für die beiden Nettowärmestromdichten
\[
	\dot{q} = \begin{bmatrix}
		\dot{q}_t \\
		\dot{q}_z
	\end{bmatrix}
	=
	\frac{\varepsilon\sigma}{\varepsilon - 3}\begin{bmatrix}
		-(T_w^4 - T_z^4) \\
		2(T_w^4 - T_z^4)
	\end{bmatrix}
\]
c) \\ \\ 
Um die gesuchte Differentialgleichung aufzustellen benötigt man die Wärmeleitgleichung für die Zylinderkoordinaten.
Die Differentialgleichung lautet daher
\[
	\rho_tc_t\frac{\partial T}{\partial t} = \lambda_t\left(\frac{1}{r}\frac{\partial}{\partial r}\left(r\frac{\partial T}{\partial r}\right)\right)
\]
mit den Randbedingungen
\begin{align*}
	\lambda_t\frac{\partial T(r,t)}{\partial r}\Bigl|_{r = r_o} = -\dot{q}_t \\
	\lambda_t\frac{\partial T(r,t)}{\partial r}\Bigl|_{r = r_N} = 0
\end{align*}
d)\\ \\
Die diskretisierten Randbedingungen können mit Tabelle \textit{Differenzenquotienten für gleichförmige Schrittweiten} aus Formelsammlung bestimmt werden. Dieses lauten daher
\begin{align}
	\dot{q}_t(T_0) &= \lambda\frac{T_1 - T_{-1}}{2\Delta r}
	0 &= \lambda_t\frac{T_{N+1} - T_{N-1}}{2\Delta r}
\end{align}
Das gleiche gilt auch für die Differentialgleichung. Daher lautet die diskretisierte Differentialgleichung
\[
	\rho_tc_t\dot{T}_i = \lambda_t\frac{1}{r_i}\left(\frac{T_{i+1} - T_{i-1}}{2\Delta r}\right) + \lambda_t\left(\frac{T_{i+1} - 2T_i + T_{i-1}}{\Delta r^2}\right) \qquad i\in\{0,...,N-1\}
\]
\textit{Hinweis: Beim Vereinfachen der Differentialgleichung, die Kettenregel nicht vergessen}