\newpage
\noindent
\textbf{Beispiel 2}\\ \\
a)\\ \\
Der gesuchte Ortsvektor lautet
\[
	\textbf{r}(\textbf{q}) = \begin{bmatrix}
		\sin(\alpha) \\
		\cos(\alpha)
	\end{bmatrix}
	(l + s)
\]
b)\\ \\
Bevor die kinetische Energie bestimmt werden kann, wird noch der Geschwindigkeitsvektor benötigt und des Betragsquadrat benötigt. Diese lauten
\begin{align*}
	\dot{\textbf{r}} &= \begin{bmatrix}
		\sin(\alpha) \\
		-\cos(\alpha)
	\end{bmatrix}
	\dot{s} + \begin{bmatrix}
		\cos(\alpha) \\
		\sin(\alpha)
	\end{bmatrix}
	(l + s)\dot{\alpha}
	\\
	||\dot{\textbf{r}}||^2_2 &= \dot{s}^2 + (l + s)^2\dot{\alpha}^2
\end{align*}
Somit lautet die kinetische Energie
\begin{align*}
	T = \frac{m}{2}\left(\dot{s}^2 + (l + s)^2\dot{\alpha}^2\right) + \frac{\theta}{2}\underbrace{\left(\dot{\alpha}^2 + 2\dot{\alpha}\dot{\beta} + \dot{\beta}^2\right)}_{(\dot{\alpha} + \dot{\beta})^2}
\end{align*}
c)\\ \\
Die potentielle Energie des Systems lautet 
\[
	V = -mg(l + s)\cos(\alpha) + \frac{1}{2}cs^2
\]
d)\\ \\
Der Vektor lautet für die Dämpferkraft
\[
	\textbf{f}_d = -d\dot{s}\begin{bmatrix}
		\sin(\alpha) \\
		-\cos(\alpha)
	\end{bmatrix}
\]
und die externe Kraft
\[
	\textbf{f}_t = f_t\begin{bmatrix}
		\cos(\alpha) \\
		\sin(\alpha)
	\end{bmatrix}
\]
Der Angriffspunkt dieser beiden Kräfte ist der Vektor aus Punkt a). Sämtlich notwendigen Ableitung werden in der Manipulator-Jacobimatrix zusammengefasst. Diese lautet
\[
	\frac{\partial \textbf{r}}{\partial \textbf{q}} = \begin{bmatrix}
		\sin(\alpha) & (l + s)\cos(\alpha) & 0 \\
		-\cos(\alpha) & (l + s)\sin(\alpha) & 0
	\end{bmatrix}	
\]
Dadurch folgt für die verallgemeinerten Kräfte der Vektor
\[
	\textbf{f}_q = \begin{bmatrix}
	1
	\end{bmatrix}
\]
\newpage
\noindent
e)\\ \\
Als erstes wird die Lagrange-Funktion benötigt. Dieses lautet für dieses System
\begin{align*}
	L &= T - V \\
	  &= \frac{m}{2}\left(\dot{s}^2 + (l + s)^2\dot{\alpha}^2\right) + \frac{\theta}{2}\left(\dot{\alpha}^2 + 2\dot{\alpha}\dot{\beta} + \dot{\beta}^2\right) + mg(l + s)\cos(\alpha) - \frac{1}{2}cs^2
\end{align*}
Die notwendigen Ableitungen lauten
\begin{align*}
	\frac{\partial L}{\partial s} &= m(l + s)\dot{\alpha}^2 + mg\cos(\alpha) -cs \\
	\frac{\partial L}{\partial \dot{s}} &= m\dot{s} \\
	\frac{\text{d}}{\text{d}t}\frac{\partial L}{\partial \dot{s}} &= m\ddot{s} 
	\\ \\
	\frac{\partial L}{\partial \alpha} &= -mg(l + s)\sin(\alpha) \\
	\frac{\partial L}{\partial \dot{\alpha}} &= m(l + s)^2\dot{\alpha} + \theta(\dot{\alpha} + \dot{\beta}) \\
	\frac{\text{d}}{\text{d}t}\frac{\partial L}{\partial \dot{\alpha}} &= 2m(l + s)\dot{s}\dot{\alpha} + m(l + s)^2\ddot{\alpha} + \theta(\ddot{\alpha} + \ddot{\beta})
	\\ \\
	\frac{\partial L}{\partial \beta} &= 0 \\
	\frac{\partial L}{\partial \dot{\beta}} &= \theta(\dot{\alpha} + \dot{\beta}) \\
	\frac{\text{d}}{\text{d}t}\frac{\partial L}{\partial \dot{\beta}} &= \theta(\ddot{\alpha} + \ddot{\beta})
\end{align*}
Somit lauten die Bewegungsgleichungen
\[
	\begin{bmatrix}
		m\ddot{s} - m(l + s)\dot{\alpha}^2 - mg\cos(\alpha)	+ cs \\
		2m(l + s)\dot{s}\dot{\alpha} + m(l + s)^2\ddot{\alpha} + \theta\ddot{\alpha} + \theta\ddot{\beta} + mg(l + s)\sin(\alpha) \\
		\theta\ddot{\alpha} + \theta\ddot{\beta}
	\end{bmatrix}
	=
	\begin{bmatrix}
		-d\dot{s} \\
		(l + s)f_t \\
		0
	\end{bmatrix}
\]