\textbf{Beispiel 1}\\ \\
a)\\ \\
Die Energieerhaltung für die Flüssigkeit lautet
\[
	\rho V(h)c_p\dot{T} = -A(h)\dot{q}(T,T_\infty)
\]
b)\\ \\
Beweis:
\begin{align*}
	\rho V(h)c_p\dot{T} &= -A(h)\dot{q}(T,T_\infty) \\
	\dot{T} &= -\frac{A(h)}{V(h)}\frac{1}{\rho c_p}\dot{q}(T,T_\infty) = -\text{const}\frac{1}{\rho c_p}\dot{q}(T,T_\infty)
\end{align*}
c)\\ \\
Auswertung:
\[
	\frac{A(h_0)}{V(h_0)} = \frac{r_0^2\pi}{h_0r^2_0\pi} = \frac{1}{h_0}
\]
Die Differentialgleichung lautet
\begin{align*}
	V(h) &= \int_{0}^{h}A(\tilde{h})\text{d}\tilde{h} \\
	\frac{\text{d}V(h)}{\text{d}h} &= \frac{\text{d}}{\text{d}h}\int_{0}^{h}A(\tilde{h})\text{d}\tilde{h} = A(h) \\
	\frac{\text{d}V(h)}{\text{d}h} &= \frac{1}{h_0}V(h)
\end{align*}
Allgemeine Lösung der Differentialgleichung
\begin{align*}
	V(h) = Ce^{\frac{h}{h_0}}
\end{align*}
Durch Einsetzen von $h = h_0$ lautet die explizite Lösung
\[
	V(h) = V_0e^{\frac{h}{h_0} - 1}
\]
e)\\ \\
Mit
\[
	\dot{q}(T,T_\infty) = \alpha(T - T\infty)
\]
und der Gleichungen a) und b) und der Starttemperatur ergibt für den Verlauf des Abkühlvorganges
\[
	T = T_\infty + (T_0 - T_\infty)e^{-\frac{\alpha}{h_0\rho c_p}t}
\]