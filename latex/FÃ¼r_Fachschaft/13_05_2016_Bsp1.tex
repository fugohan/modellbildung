\textbf{Beispiel 1} \\ \\
a) \\ \\
Der Schwingenwinkel lässt sich folgendermaßen bestimmen
\begin{align*}
	\tan\varphi_S &= \frac{r_K\cos\varphi_K}{h_K + r_K\sin\varphi_K} \\
	\varphi_S &= \arctan\left( \frac{r_K\cos\varphi_K}{h_K + r_K\sin\varphi_K}\right)
\end{align*}
Dessen Ableitung lautet
\begin{align*}
	\dot{\varphi_S} &= \frac{1}{\left( \frac{r_K\cos\varphi_K}{h_K + r_K\sin\varphi_K}\right)^2 + 1} \frac{-r_K\sin\varphi_K(h_K + r_K\sin\varphi_K) - r_K\cos\varphi_K(r_K\cos\varphi_K)}{h_K^2 + 2h_Kr_K\sin\varphi_K + r_K^2\sin^2\varphi_K}\dot{\varphi_K}\\
	&= \frac{1}{\frac{r_K^2\cos^2\varphi_K}{h_K^2 + 2h_Kr_K\sin\varphi_K + r_K^2\sin^2\varphi_K} + 1}\frac{-r_K\sin\varphi_K(h_K + r_K\sin\varphi_K) - r_K\cos\varphi_K(r_K\cos\varphi_K)}{h_K^2 + 2h_Kr_K\sin\varphi_K + r_K^2\sin^2\varphi_K}\dot{\varphi_K} \\
	&= \frac{1}{\frac{r_K^2\cos^2\varphi_K + h_K^2 + 2h_Kr_K\sin\varphi_K + r_K^2\sin^2\varphi_K}{h_K^2 + 2h_Kr_K\sin\varphi_K + r_K^2\sin^2\varphi_K}}\frac{-r_K\sin\varphi_K(h_K + r_K\sin\varphi_K) - r_K\cos\varphi_K(r_K\cos\varphi_K)}{h_K^2 + 2h_Kr_K\sin\varphi_K + r_K^2\sin^2\varphi_K}\dot{\varphi_K} \\
	&= \frac{h_K^2 + 2h_Kr_K\sin\varphi_K + r_K^2\sin^2\varphi_K}{r_K^2\cos^2\varphi_K + h_K^2 + 2h_Kr_K\sin\varphi_K + r_K^2\sin^2\varphi_K}\frac{-r_K\sin\varphi_K(h_K + r_K\sin\varphi_K) - r_K\cos\varphi_K(r_K\cos\varphi_K)}{h_K^2 + 2h_Kr_K\sin\varphi_K + r_K^2\sin^2\varphi_K}\dot{\varphi_K} \\
	&= \frac{-r_K\sin\varphi_K(h_K + r_K\sin\varphi_K) - r_K\cos\varphi_K(r_K\cos\varphi_K)}{r_K^2\cos^2\varphi_K + h_K^2 + 2h_Kr_K\sin\varphi_K + r_K^2\sin^2\varphi_K}\dot{\varphi_K} \\
	&= -\frac{r_K^2 + r_Kh_K\sin\varphi_K}{r_K^2 + 2\sin\varphi_Kr_kh_K + h_K^2}\dot{\varphi_K}\\
	&= -\frac{r_K(r_K + h_K\sin\varphi_K)}{r_K^2 + 2\sin\varphi_Kr_kh_K + h_K^2}\dot{\varphi_K}
\end{align*}
b) \\ \\
Die Länge der Feder beträgt
\[
	l_F = h_s\tan\varphi_S + 2l_{F0}
\]
Der Tangens ist aus der Geometrie der Angabe ersichtlich. \\ \\
c) \\ \\
Den Schwerpunktpunktsvektor kann man hier direkt aus der Angabe ablesen
\[
	\textbf{r}_S = l_S \begin{bmatrix}
		\sin\varphi_S \\
		\cos\varphi_S
	\end{bmatrix},
\]
und der zugehörige Geschwindigkeitsvektor lautet
\[
	\textbf{v}_S = l_s\begin{bmatrix}
		\cos\varphi_S \\
		-\sin\varphi_S
	\end{bmatrix}
	\dot{\varphi_S}
\]
d) \\ \\
Die beiden kinetischen Energie lauten
\begin{align*}
	T_K &= \frac{1}{2} I_K \dot{\varphi_K}^2 \\
	T_S &= \frac{1}{2}I_S\dot{\varphi_S}^2 + \frac{1}{2}m_Sl_S^2\dot{\varphi_S}^2 \\
	    &= \frac{1}{2}(m_S + l_s^2)\dot{\varphi_S}^2
\end{align*}
e)\\ \\
Die beiden potentiellen Energien lauten
\begin{align*}
	V_S &= l_Sm_sg\cos\varphi_S \\
	V_F &= \frac{1}{2}k_F(l_F - l_{F0}) \\
	    &= \frac{1}{2}k_F(h_s\tan\varphi_S + l_{F0})
\end{align*}
f) \\ \\
Die generalisierte Kraft kann direkt aus der Angabe hergeleitet werden
\[
	Q = M_K
\]
g)\\ \\
Die Lagrange-Funktion lautet hier
\begin{align*}
	L &= T - V \\
	  &= \frac{1}{2}\left(I_K +(I_S + m_Sl_S^2)\left( \frac{r_K(r_K + h_K\sin\varphi_K)}{r_K^2 + 2\sin\varphi_Kr_kh_K + h_K^2}\right)^2\right)\dot{\varphi_K}^2 \\
	  &- l_Sm_sg\cos\left( \arctan\left( \frac{r_K\cos\varphi_K}{h_K + r_K\sin\varphi_K}\right)\right) \\
	  &- \frac{1}{2}k_F\left(h_s \frac{r_K\cos\varphi_K}{h_K + r_K\sin\varphi_K} + l_{F0}\right)
\end{align*}
Die Bewegungsgleichung lässt sich mit folgender Formel bestimmen
\[
	\frac{d}{dt}\left(\frac{\partial L}{\partial \dot{\varphi_K}}\right) - \frac{\partial L}{\partial \varphi_K} = Q
\]