\newpage
\noindent
\textbf{Beispiel 2} \\ \\
a) \\ \\
%\noindent
Die Oberfläche des Öl's strahlt nicht auf sich selbst, deshalb lautet der Sichtfaktor Matrixeintrag \(F_{OO} = 0\), aus der Summationsregel folgt \(F_{OH} = 1 - F_{OO} = 1\). Aus dem Reziprozitätsgesetz folgt \(F_{HO}=\frac{A_O}{A_H}F_{OH} \). Der Letzte Eintrag lautet wegen der Summationsregel \(F_{HH}=1-\frac{A_O}{A_H}\)

\[\textbf{F}=\left[\begin{matrix}
0 & 1 \\ \frac{A_O}{A_H} & 1 -\frac{A_O}{A_H} \end{matrix}\right]\]

\begin{tiny}
%\hspace{100px}	
\( \hspace{-75pt}\dot{\textbf{q}}=  \left[ \begin {array}{c} \sigma\, \left(  \left( {\frac {\varepsilon_
		{H}\,A_{H}-\varepsilon_{H}\,A_{O}+A_{O}}{-A_{O}\,\varepsilon_{H}\,
		\varepsilon_{O}+\varepsilon_{H}\,A_{H}+A_{O}\,\varepsilon_{O}}}+{
	\frac { \left( -1+\varepsilon_{H} \right) A_{O}}{-A_{O}\,\varepsilon_{
			H}\,\varepsilon_{O}+\varepsilon_{H}\,A_{H}+A_{O}\,\varepsilon_{O}}}
\right) \varepsilon_{O}\,{{\it TO}}^{4}+ \left( -{\frac { \left( -1+
		\varepsilon_{O} \right) A_{H}}{-A_{O}\,\varepsilon_{H}\,\varepsilon_{O
		}+\varepsilon_{H}\,A_{H}+A_{O}\,\varepsilon_{O}}}-{\frac {A_{H}}{-A_{O
		}\,\varepsilon_{H}\,\varepsilon_{O}+\varepsilon_{H}\,A_{H}+A_{O}\,
		\varepsilon_{O}}} \right) \varepsilon_{H}\,{{\it TH}}^{4} \right) 
\\ \noalign{\medskip}\sigma\, \left(  \left( -{\frac {A_{O}\, \left( 
		\varepsilon_{H}\,A_{H}-\varepsilon_{H}\,A_{O}+A_{O} \right) }{A_{H}\,
		\left( -A_{O}\,\varepsilon_{H}\,\varepsilon_{O}+\varepsilon_{H}\,A_{H
		}+A_{O}\,\varepsilon_{O} \right) }}-{\frac {{A_{O}}^{2} \left( -1+
		\varepsilon_{H} \right) }{A_{H}\, \left( -A_{O}\,\varepsilon_{H}\,
		\varepsilon_{O}+\varepsilon_{H}\,A_{H}+A_{O}\,\varepsilon_{O} \right) 
}} \right) \varepsilon_{O}\,{{\it TO}}^{4}+ \left( {\frac {A_{O}\,
		\left( -1+\varepsilon_{O} \right) }{-A_{O}\,\varepsilon_{H}\,
		\varepsilon_{O}+\varepsilon_{H}\,A_{H}+A_{O}\,\varepsilon_{O}}}+{
	\frac {A_{O}}{-A_{O}\,\varepsilon_{H}\,\varepsilon_{O}+\varepsilon_{H}
		\,A_{H}+A_{O}\,\varepsilon_{O}}} \right) \varepsilon_{H}\,{{\it TH}}^{
	4} \right) \end {array} \right] 
\)
\end{tiny}
\\ \newline
\bigskip
\noindent
Der Vektor \(\dot{\textbf{q}}\) besteht aus den Einträgen \(\dot{q}_O\) und \(\dot{q}_H\), wir benötigen aber nur den ersten Eintrag, weil \(\dot{Q}_{rad} = A_O \dot{q}_O\) lautet.
\bigskip \\
b) \\
\\
Um die stationäre thermische Energiebilanz zu bestimmen müssen alle Wärmeströme die ein- oder austreten addiert werden. 

\[\dot{m}_{LM} c_{p,LM} (T_\infty-T_L)+0.2\dot{m}_{LM} c_{p,O}-A_O \alpha_{OL}(T_O-T_\infty)-\dot{Q}_{rad}+P_{el}\eta=0\]
\bigskip 

\noindent c)  \\ 
\\ 
Die Temperatur \(T_O\) lautet nun \(\overline{T}_O\) und wird in die Lösung von b eingesetzt. Danach muss die Gleichung nach \(P_{el}\) umgeformt werden. Daraus ergibt sich. 

\[P_{el}=\frac{1}{\eta}\left(-\dot{m}_{LM} c_{p,LM} (T_\infty-T_L)-0.2\dot{m}_{LM} c_{p,O}-A_O \alpha_{OL}(T_O-T_\infty)+\dot{Q}_{rad}\right)\]

\bigskip 

\noindent d)  \\ 
\\
Die Anfangsbedienung der Differentialgleichung lautet \(T_O(t=0)=\overline{T}_O\) und die Differentialgleichung erhält man indem man die Fouriersche Wärmeleitgleichung nach dem Volumen integriert. Zu beachten ist das auf der rechten Seite alle ein- und ausfließenden Wärmeströme stehen: 
\[
\int \rho c_{p,O} \frac{\partial T_O}{\partial t} dV = \int -\alpha_{O,L}(T_O-T_\infty) - \dot{q}_{rad}\, dA
\]
\[
m_O\,c_{p,O} \frac{\partial T_O}{\partial t} = A_O \alpha_{O,L} (T_O-T_\infty) - \dot{Q}_{rad}
\]