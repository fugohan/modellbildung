\noindent
\textbf{Beispiel 1} \\ \\
a) \\ \\
Der Richtungsvektor vom Urspung zum Schwerpunkt der Masse $m$ lautet:
\[
	\textbf{p}_m = \left[ \begin {array}{c} r\cos \left( \alpha \right) +b
	\\ r\sin \left( \alpha \right) -h\end {array}
	\right]	
\]
Mithilfe diesen Vektor kann ebenfalls auch der Geschwindigkeitsvektor bestimmt werden, indem man den Richtungsvektor nach der Zeit ableitet.
\[
	\dot{\textbf{p}}_m =\left[ \begin {array}{c} -r \dot{\alpha}
	  \sin \left( \alpha 
	\right) + \dot{b} \\ 
	r \dot{\alpha} \cos \left( \alpha  \right) -
		\dot{h} \end {array} \right] 
\]
b) \\ \\
Um die kinetischen Energien zu berechnen, müssen zuerst einige Nebenrechnungen durchgeführt werden.\\
Nebenrechnungen:
\begin{align}
	\dot{\textbf{p}}_m^T \dot{\textbf{p}}_m &= \left[ \begin{matrix}
		-r\dot{\alpha}\sin \left( \alpha \right) + \dot{b} & r\dot{\alpha}\cos \left( \alpha \right) -\dot{h}
	\end{matrix}\right] \left[ \begin {array}{c} -r \dot{\alpha}
	\sin \left( \alpha 
	\right) + \dot{b} \\ 
	r \dot{\alpha} \cos \left( \alpha  \right) -
	\dot{h} \end {array} \right] \\
	&= r^2\dot{\alpha}^2\underbrace{\left(\cos \ \alpha  + \sin  \alpha\right)  }_{=1} -2r\dot{\alpha}\dot{b}\sin\alpha - 2r\dot{\alpha}\dot{h}\cos\alpha + \dot{b}^2 + \dot{h}^2 \\
	&= r^2\dot{\alpha}^2 -2r\dot{\alpha}\dot{b}\sin\alpha - 2r\dot{\alpha}\dot{h}\cos\alpha + \dot{b}^2 + \dot{h}^2 \\
	\varphi &= \arctan\left(\frac{h}{b}\right) \\
	\dot{\varphi} &= \frac{\dot{h}b - h\dot{b}}{b^2 + h^2}
\end{align}
kinetische Energie des Systems:
\begin{align*}
	T_{tm} &= \frac{1}{2}m\dot{\textbf{p}}_m^T \dot{\textbf{p}}_m = \frac{1}{2} m \left( r^2\dot{\alpha}^2 -2r\dot{\alpha}\dot{b}\sin\left(\alpha\right) - 2r\dot{\alpha}\dot{h}\cos\left(\alpha\right) + \dot{b}^2 + \dot{h}^2\right) \\
	T_{rm} &= \frac{1}{2} \Theta_m\dot{\varphi}^2 = \frac{1}{2} \Theta_m \left( \frac{\dot{h}b - h\dot{b}}{b^2 + h^2}\right)^2 \\
	T_{rr} &= \frac{1}{2}\Theta_r\dot{\alpha}^2 \\
	T &= T_{tm} + T_{rm} + T_{rr} \\
	  &= \frac{1}{2}m\dot{\textbf{p}}_m^T \dot{\textbf{p}}_m = \frac{1}{2} m \left( r^2\dot{\alpha}^2 -2r\dot{\alpha}\dot{b}\sin\left(\alpha\right) - 2r\dot{\alpha}\dot{h}\cos\left(\alpha\right) + \dot{b}^2 + \dot{h}^2\right) 
	  + \frac{1}{2} \Theta_m \left( \frac{\dot{h}b - h\dot{b}}{b^2 + h^2}\right)^2 + \frac{1}{2}\Theta_r\dot{\alpha}^2
\end{align*}
\newpage
\noindent
c) \\ \\
potentielle Energie des Systems:
\begin{align*}
	V_m &= mg\left(r\sin\alpha - h\right) \\
	V_c &= \frac{1}{2} c \left(\sqrt{b^2 + h^2} s_0\right)^2 \\
	V &= V_m + V_c = mg\left(r\sin\alpha - h\right) + \frac{1}{2} c \left(\sqrt{b^2 + h^2} s_0\right)^2
\end{align*}
d) \\ \\
Die viskose Reibkraft muss in der Form $f_r = \mu_V \Delta v$ beschrieben werden. In diesen Beispiel ist $\Delta v = \dot{\varphi}\textbf{r}$. \textbf{r} ist der Vektor vom Angriffspunkt der Feder zum Schwerpunkt der Masse.
\[
	\textbf{f}_V = \mu_V \underbrace{\frac{\dot{h}b - h\dot{b}}{b^2 + h^2}}_{\dot{\varphi}} \underbrace{\begin{bmatrix}
	b \\ -h
	\end{bmatrix}}_{\textbf{r}}
\]
partielle Ableitungen:
\[
	\frac{\partial \dot{\textbf{p}}_m}{\partial \alpha}  = \left[ \begin{matrix}
		-r\sin\alpha \\ r\cos\alpha
	\end{matrix}\right] \qquad \frac{\partial \dot{\textbf{p}}_m}{\partial b} = \left[ \begin{matrix}
	 1 \\ 0
	\end{matrix}\right] \qquad \frac{\partial \dot{\textbf{p}}_m}{\partial h} = \left[ \begin{matrix}
	0 \\ -1
	\end{matrix}\right]
\]
\noindent
generalisierte Kräfte:\\ \\
Multipliziert man die viskose Reibungskraft mit allen partiellen Ableitungen erhält man
\begin{align*}
	\textbf{f}_{q,v} &= \mu_V \frac{\dot{h}b - h\dot{b}}{b^2 + h^2} \left[ \begin{matrix}
		r\sin\alpha b + r\cos\alpha h \\
		-b \\
		-h
	\end{matrix}\right]
\end{align*}
Die andere externe Kraft ist
\[
	\textbf{f}_x = \left[ \begin{matrix}
		f_x \\
		0
	\end{matrix}\right]
\]
Multipliziert mit den partiellen Ableitungen erhält man
\[
	\textbf{f}_{q,x} = \left[ \begin{matrix}
		-f_x r \sin\alpha \\
		f_x \\
		0
	\end{matrix}\right]
\]
Der gesamte Vektor der generalisierten Kräfte beträgt
\[
	\textbf{f}_q = \mu_V \frac{\dot{h}b - h\dot{b}}{b^2 + h^2} \left[ \begin{matrix}
	r\sin\alpha b + r\cos\alpha h \\
	-b \\
	-h
	\end{matrix}\right] 
	+ 
	\left[ \begin{matrix}
	-f_x r \sin\alpha \\
	f_x \\
	0
	\end{matrix}\right]
\]
e) \\ \\
Verwendet man aus der Formelsammlung im Punkt generalisierte Kräfte die 2.te Formel und passt man diese an den stationären Fall an erhält man
\[
	\frac{\partial V}{\partial \textbf{q}} = \left[ \begin{matrix}
		mgr\cos\alpha\\
		\frac{c\left( \sqrt{b^2 + h^2} - s_0\right)}{\sqrt{b^2 + h^2}} b \\
		\frac{c\left( \sqrt{b^2 + h^2} - s_0\right)}{\sqrt{b^2 + h^2}} h - mg
	\end{matrix}\right]
	=
	\left[ \begin{matrix}
	-f_x r \sin\alpha \\
	f_x \\
	0
	\end{matrix}\right]
\]
Nun wird $s_0$ mit 0 angenommen und anschließend werden die generalisierten Koordinaten bestimmt. \\ \\
1.Koordinate:
\begin{align*}
	mgr\cos\alpha &= -f_xr\sin\alpha \\
	-\frac{\cos\alpha}{\sin\alpha} &= \frac{mg}{f_x}  \\
	-\tan\alpha &= \frac{mg}{f_x} \\
	\alpha &= -\arctan\left( \frac{mg}{f_x}\right)
\end{align*}
2.Koordinate:
\begin{align*}
	\frac{c\left(\cancel{\sqrt{b^2 + h^2}}\right)}{\cancel{\sqrt{b^2 + h^2}}} b &= f_x \\
	cb &= f_x \\
	b &= \frac{f_x}{c}
\end{align*}
3.Koordinate
\begin{align*}
	\frac{c\left(\cancel{\sqrt{b^2 + h^2}}\right)}{\cancel{\sqrt{b^2 + h^2}}} h -mg &= 0 \\
	ch - mg &= 0 \\
	h &= \frac{mg}{c}
\end{align*}
f) \\ \\
Die resultierende Reibkraft wird mithilfe der Formel 2.96 aus dem Vorlesungsskript berechnet und lautet hier:
\[
	\textbf{f}_F = -c_W A \frac{\rho}{2} |\dot{\textbf{p}}_m|\dot{\textbf{p}}_m
\]
\begin{align*}
	|\dot{\textbf{p}}_m| &= \sqrt{\left( -r \dot{\alpha}
		\sin \left( \alpha 
		\right) + \dot{b}\right)^2 
	+
	\left( 	r \dot{\alpha} \cos \left( \alpha  \right) -
	\dot{h}\right)^2} \\
	&= \sqrt{r^2\dot{\alpha}^2\sin^2\alpha -2r\dot{\alpha}\dot{b}\sin\alpha + \dot{b}^2 + r^2\dot{\alpha}^2\sin^2\alpha - 2r\dot{\alpha}\dot{h}\cos\alpha + \dot{h}^2} \\
	&= \sqrt{r^2\dot{\alpha}^2\underbrace{\left(\sin^2\alpha + \cos^2\alpha\right)}_{=1} - 2r\dot{\alpha}\dot{b}\sin\alpha - 2r\dot{\alpha}\dot{h}\sin\alpha + \dot{b}^2 + \dot{h}^2} \\
	&= \sqrt{r^2\dot{\alpha}^2 - 2r\dot{\alpha}\dot{b}\sin\alpha - 2r\dot{\alpha}\dot{h}\sin\alpha + \dot{b}^2 + \dot{h}^2}
\end{align*}
Multipliziert man nun $\textbf{f}_F$ mit den partiellen Ableitungen aus d) und vereinfacht man so weit wie möglich erhält man
\begin{align*}
	\textbf{f}_{q,F} &= -c_W A \frac{\rho}{2}|\dot{\textbf{p}}_m|
	\left[\begin{matrix}
		r\dot{\alpha}\underbrace{\left( \sin^2\alpha + \cos^2\alpha \right)}_{=1} - r\dot{b}\sin\alpha - r\dot{h}\cos\alpha \\
		-r\dot{\alpha}\sin \left( \alpha \right) + \dot{b} \\
		-r\dot{\alpha}\cos \left( \alpha \right) +\dot{h}
	\end{matrix}
	\right] \\
	&= -c_W A \frac{\rho}{2}|\dot{\textbf{p}}_m|
	\left[\begin{matrix}
	r\dot{\alpha} - r\dot{b}\sin\alpha - r\dot{h}\cos\alpha \\
	-r\dot{\alpha}\sin \left( \alpha \right) + \dot{b} \\
	-r\dot{\alpha}\cos \left( \alpha \right) +\dot{h}
	\end{matrix}
	\right]
\end{align*}