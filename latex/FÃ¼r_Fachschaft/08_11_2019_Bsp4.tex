\textbf{Beispiel 4}\\ \\
a) \\ \\
Die Sichtfaktoren lauten
\begin{align*}
	F_{QQ} &= 0 \\
	F_{QS} &= 1 - F_{QQ} = 1 \\
	F_{SQ} &= \frac{A_Q}{A_S}F_{QS} = \frac{2\pi R}{8a} \\
	F_{SS} &= 1 - F_{SQ} = 1 - \frac{2\pi R}{8a}
\end{align*}
b) \\ \\
Mit 
\[
	a = \frac{\pi R}{2}
\]
lautet die Sichtfaktormatrix
\[
	\textbf{F} = \begin{bmatrix}
		\frac{1}{2} & \frac{1}{2} \\
		1 & 0
	\end{bmatrix}
\]
Rechenweg für die gesuchte Nettowärmestromdichte
\begin{align*}
	\dot{\textbf{q}} &= (\textbf{E} - \textbf{F})(\textbf{E} - (\textbf{E} - \text{diag}\{\varepsilon\})\textbf{F})^{-1}\text{diag}\{\varepsilon\}\sigma\textbf{T}^4 \\
	\begin{bmatrix}
		\dot{q}_S \\
		\dot{q}_Q
	\end{bmatrix}&=\begin{bmatrix}
		\frac{1}{2} & - \frac{1}{2} \\
		-1 & 1
	\end{bmatrix}
	\begin{bmatrix}
		\frac{1 + \varepsilon_S}{2} & \frac{-1 + \varepsilon_S}{2} \\
		-1 + \varepsilon_Q	&	1
	\end{bmatrix}^{-1}
	\begin{bmatrix}
		\varepsilon_S	&	0\\
		0	&	\varepsilon_Q		
	\end{bmatrix}
	\sigma
	\begin{bmatrix}
		T_S^4 \\
		T_Q^4
	\end{bmatrix}
	\\
	&= \frac{1}{\varepsilon_S\varepsilon_Q - 2\varepsilon_S - \varepsilon_Q}
	\begin{bmatrix}
		-\varepsilon_Q & \frac{\varepsilon_S}{2} \\
		2\varepsilon_Q & -2\varepsilon_S
	\end{bmatrix}
	\begin{bmatrix}
	\varepsilon_S	&	0\\
	0	&	\varepsilon_Q		
	\end{bmatrix}
	\sigma
	\begin{bmatrix}
	T_S^4 \\
	T_Q^4
	\end{bmatrix}
	\\
	&= \frac{1}{\varepsilon_S\varepsilon_Q - 2\varepsilon_S - \varepsilon_Q}
	\begin{bmatrix}
	-\varepsilon_Q & \frac{\varepsilon_S}{2} \\
	2\varepsilon_Q & -2\varepsilon_S
	\end{bmatrix}
	\begin{bmatrix}
		\sigma\varepsilon_ST_S^4 \\
		\sigma\varepsilon_QT_Q^4
	\end{bmatrix}
	\\
	& \Rightarrow \dot{q}_Q = \frac{2\varepsilon_S\varepsilon_Q\sigma(T_S^4 - T_Q^4)}{\varepsilon_S\varepsilon_Q - 2\varepsilon_S - \varepsilon_Q}
\end{align*}
Mit $\varepsilon_S = \varepsilon_Q = 1$ lautet die gewünschte Nettowärmestromdichte
\[
	\dot{q}_Q = \sigma(T_Q^4 - T_S^4)
\]
\newpage
\noindent
c) \\ \\
Die allgemeine Form der Wärmeleitunggleichung die verwendet werden soll, ist die für die Zylinderkoordinaten. Die Wärmeleitgleichung lautet daher
\[
	\rho c_p\frac{\partial T_Q(r,t)}{\partial t} = \lambda \left(\frac{1}{r}\frac{\partial}{\partial r}(rT_Q(r,t))\right) + g
\]
mit den Randbedingungen
\begin{align*}
	\frac{\partial T_Q(r,t)}{\partial}\Biggl|_{r = 0} &= 0 \\
	T_Q(r,t)\biggl|_{r = R} &= T_Q(R,t1)
\end{align*}
d) \\ \\
Um das stationäre Temperaturprofil zu bestimmen, muss man die zeitliche Ableitung in der partiellen Differentialgleichung 0 setzen.
\[
	\lambda \left(\frac{1}{r}\frac{\partial}{\partial r}(rT_Q(r,t))\right) + g = 0
\]
Durch Lösen dieser Gleichung und Einsetzen der Randbedingung ergibt sich für das stationäre Temperaturprofil der Quelle
\[
	T(r,\infty) = T_Q(R,\infty) + \frac{g}{4\lambda}\left(R^2 - r^2\right)
\]