\textbf{Beispiel 1} \\ \\
a)\\ \\
Um das Bremsmoment ermitteln zu können benötigt man zu aller erst einmal die Wirkfläche. In diesem  Beispiel beträgt diese Fläche
\[
	A = \int_{R_i}^{R_a}\int_{0}^{\frac{\pi}{3}}rd\varphi dr = \frac{\pi}{6}\left(R_a^2 - R_i^2\right)
\]
Das Bremsmoment lässt sich mit 
\[
	M_R = \int_{R_i}^{R_a}\int_{0}^{\frac{\pi}{3}} r^2\mu\frac{F_B}{A}d\varphi dr
\]
zu 
\begin{align*}
	M_R &= \int_{R_i}^{R_a}\int_{0}^{\frac{\pi}{3}} r^2\mu F_B \frac{6}{\pi \left(R_a^2 - R_i^2\right)}d\varphi dr = \\
	M_R &= \int_{R_i}^{R_a} r^2 \mu F_B \frac{\cancelto{2}{6}}{\cancel{\pi} \left(R_a^2 - R_i^2\right)} \frac{\cancel{\pi}}{\cancel{3}} dr = \\
	M_R &= \mu F_B \frac{2}{3}\frac{R_a^3 - R_i^3}{R_a^2 - R_i^2}
\end{align*}
Da zwei Bremsbacken zum Abbremsen verwendet werden lautet das gesamte Bremsmoment 
\[
	M_{ges} = 2M_R = \mu F_B \frac{4}{3}\frac{R_a^3 - R_i^3}{R_a^2 - R_i^2}
\]
b) \\ \\
Die Bewegungsdifferenzialgleichung der Schwungscheibe lautet
\[
	\ddot{\varphi} = - \frac{M_{ges}}{\Theta_{zz}}
\]
Das "$-$" vor dem Brauch kommt daher, dass die $\ddot{\varphi}$ mit der Zeit immer kleiner werden soll, da ja ein Bremsvorgang stattfindet.
Die Anfangsbedingungen für dieses Problem lauten
\[
	\varphi(0) = 0
\]
kann aber beliebig gewählt werden, und 
\[
	\omega(0) = \overline{\omega}
\]
\newpage
\noindent
Als nächstes muss man diese Differenzialgleichung lösen. Dies erreicht man durch 2-maliges Integrieren.
\begin{align*}
	\omega(t) = \dot{\varphi}(t) &= - \frac{M_{ges}}{\Theta_{zz}}t + \overline{\omega} \\
	\varphi(t) &= - \frac{M_{ges}}{\Theta_{zz}} \frac{t^2}{2} + \overline{\omega}t
\end{align*}
Damit ergibt sich die Bremsdauer $t_B$ zu
\begin{align*}
	0 &= \overline{\omega} - \frac{M_{ges}}{\Theta_{zz}}t_B \\ 
	\overline{\omega} &= \frac{M_{ges}}{\Theta_{zz}}t_B \\
	t_B &= \frac{\Theta_{zz}\overline{\omega}}{M_{ges}}
\end{align*}
Um die Bremskraft $F_B$ zu bestimmen setzt man $M_{ges}$ ein und formt einfach auf $F_H$ um und man erhält
\[
	F_B = \frac{3}{4}\frac{R_a^2 - R_i^2}{R_a^3 - R_i^3}\frac{\Theta_{zz}\overline{\omega}}{t_B\mu}
\]
c) \\ \\
Die Bremsleistung lässt sich mit $\dot{W}_R = -M_{ges}\omega$ ermitteln. In diesem Fall lautet somit die Bremsleistung
\[
	\dot{W}_R = \frac{M_{ges}^2}{\Theta_{zz}}t - M_{ges}\overline{\omega}
\]
Integriert man nun diese Leistung über die Bremsdauer erhält man die Energie
\begin{align*}
	E_B &= \int_{0}^{t_B}\frac{M_{ges}^2}{\Theta_{zz}}t - M_{ges}\overline{\omega} dt = \\
	&= \frac{1}{2}\frac{M_{ges}^2}{\Theta_{zz}}\frac{\Theta_{zz}^2\overline{\omega}^2}{M_{ges}^2} - M_{ges}\overline{\omega}\frac{\Theta_{zz}\overline{\omega}}{M_{ges}} \\
	&= -\frac{1}{2}\Theta_{zz}\overline{\omega}
\end{align*}
Dies entspricht der im Schwungrad gespeicherten Energie.\\ \\
d) \\ \\
Aus dem ersten Hauptsatz der Thermodynamik ergibt sich die Differenzialgleichung 
\[
	m_Bc_B\frac{dT_B(t)}{dt} = \frac{1}{2}\left(M_{ges}\overline{\omega} - \frac{M_{ges}^2}{\Theta_{zz}}t\right)
\]
Löst man diese Gleichung durch einfaches Integrieren erhält man
\begin{align*}
	T_B(t) = \frac{1}{2m_Bc_B}\left(M_{ges}\overline{\omega}t - \frac{1}{2}\frac{M_{ges}^2}{\Theta_{zz}}t^2\right) + C
\end{align*}
Setzt man die AB $T_B(0) = T_{B,0}$ ein erhält man
\[
	T_B(t) = T_{B,0} + \frac{1}{2m_Bc_B}\left(M_{ges}\overline{\omega}t - \frac{1}{2}\frac{M_{ges}^2}{\Theta_{zz}}t^2\right)
\]
Zum Zeitpunkt $t_B$ ergibt sich nun folgende Temperatur
\begin{align*}
		T_B(t) &= T_{B,0} + \frac{1}{2m_Bc_B}\left(M_{ges}\overline{\omega}t_B - \frac{1}{2}\frac{M_{ges}^2}{\Theta_{zz}}t_B^2\right) = \\
		&= T_{B,0} + \frac{\Theta_{zz}\overline{\omega}^2}{4m_Bc_B} = \\
		&= T_{B,0} - \frac{1}{m_Bc_B}\frac{E_B}{2} 
\end{align*}