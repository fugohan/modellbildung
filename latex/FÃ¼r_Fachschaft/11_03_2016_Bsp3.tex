\textbf{Beispiel 3} \\ \\
a) \\ \\
Die hier geeigneten generalisierten Koordinaten sind
\[
	\textbf{q} = \begin{bmatrix}
		\varphi_1 \\
		\varphi_2 \\
		s
	\end{bmatrix}
\]
b)\\ \\
Die kinetische Energie dieses Systems lautet
\[
	T = \frac{1}{2}m_s\dot{s}^2 + \frac{1}{2} \Theta_1 \dot{\varphi_1}^2 + \frac{1}{2} \Theta_2 \dot{\varphi_2}^2
\]
c) \\ \\
Die potentielle Energie dieses Systems lautet
\[
	V = \frac{1}{2} c_1 (s - r\varphi_1)^2 + \frac{1}{2} c_2 (r\varphi_2 - s)^2 + \frac{1}{2} c_3 r^2(\varphi_1 - \varphi_2)^2
\]
d) \\ \\
Um den Euler-Lagrange-Formalimus anwenden zu können benötigt man sowohl die Lagrange-Funktion L und die generalisierten Kräfte. \\
Lagrange-Funktion
\[
	L = T - V = \frac{1}{2}m_s\dot{s}^2 + \frac{1}{2} \Theta_1 \dot{\varphi_1}^2 + \frac{1}{2} \Theta_2 \dot{\varphi_2}^2 - \frac{1}{2} c_1 (s - r\varphi_1)^2 - \frac{1}{2} c_2 (r\varphi_2 - s)^2 - \frac{1}{2} c_3 r^2(\varphi_1 - \varphi_2)^2
\]
Die generalisierten Kräfte können direkt aus der Angabe abgelesen werden.
\[
	f_{np} = \begin{bmatrix}
		M_1 \\
		0 \\
		-\mu_V\dot{s}
	\end{bmatrix}
\]
Euler-Lagrange-Formalimus:
\begin{align*}
	\frac{d}{dt}\left(\frac{\partial L}{\partial \dot{\varphi_1}}\right) - \left(\frac{\partial L}{\partial\varphi_1}\right) &= M_1\\
	\frac{d}{dt}\left(\frac{\partial L}{\partial \dot{\varphi_2}}\right) - \left(\frac{\partial L}{\partial \varphi_2}\right) &= 0\\
	\frac{d}{dt}\left(\frac{\partial L}{\partial \dot{s}}\right) - \left(\frac{\partial L}{\partial s}\right) &= -\mu_V\dot{s}
\end{align*}
Einzelne Ableitungen nach $\textbf{q}$:
\begin{align*}
	\frac{\partial L}{\partial \varphi_1} &= c_1r(s - r\varphi_1) - c_3r^2(\varphi_1 - \varphi_2) \\
	\frac{\partial L}{\partial \varphi_2} &= - c_2r(r\varphi_2 - s) + c_3r^2(\varphi_1 - \varphi_2) \\
	\frac{\partial L}{\partial s} &= -c_1(s - r\varphi_1) + c_2(r\varphi_2 - s)
\end{align*}
Einzelne Ableitungen nach $\dot{\textbf{q}}$:
\begin{align*}
	\frac{\partial L}{\partial \dot{\varphi_1}} &= \Theta_1\dot{\varphi_1} \\
	\frac{\partial L}{\partial \dot{\varphi_2}} &= \Theta_1\dot{\varphi_2} \\
	\frac{\partial L}{\partial \dot{s}} &= m_s\dot{s}
\end{align*}
Deren zeitliche Ableitungen:
\begin{align*}
	\frac{d}{dt}\left(\frac{\partial L}{\partial \dot{\varphi_1}} \right) &= \Theta_1\ddot{\varphi_1} \\
	\frac{d}{dt}\left(\frac{\partial L}{\partial \dot{\varphi_2}} \right) &= \Theta_1\ddot{\varphi_2} \\
	\frac{d}{dt}\left(\frac{\partial L}{\partial \dot{s}} \right) &= m_s\ddot{s}
\end{align*}
Der Euler-Lagrange-Formalismus lautet ausgeschrieben
\begin{align*}
	\Theta_1\ddot{\varphi_1} - c_1r(s - r\varphi_1) + c_3r^2(\varphi_1 - \varphi_2) &= M_1 \\
	\Theta_1\ddot{\varphi_2} + c_2r(r\varphi_2 - s) - c_3r^2(\varphi_1 - \varphi_2) &= 0 \\
	m_s\ddot{s} + c_1(s - r\varphi_1) - c_2(r\varphi_2 - s) &= -\mu_V\dot{s}
\end{align*}
\newpage
\noindent
Somit lauten die Bewegungsgleichungen
\begin{align*}
	\ddot{\varphi_1} &= \frac{1}{\Theta_1}(M_1 + c_1r(s - r\varphi_1) - c_3r^2(\varphi_1 - \varphi_2))\\
	\ddot{\varphi_2} &= \frac{1}{\Theta_2}(-c_2(r\varphi_2 - s) + c_3r^2(\varphi_1 - \varphi_2)) \\
	\ddot{s} &= \frac{1}{m_s}(-c_1(s - r\varphi_1) + c_2(r\varphi_2 - s) - \mu_V\dot{s})
\end{align*}
e) \\ \\
Die Feder 1 ist parallel zu Federn 2 und 3, die Serie zu sehen sind. Daraus folgt eine Gesamtsteifigkeit von
\[
	c = c_1 + \frac{c_2c_3}{c_2 + c_3}
\]
Hier muss nur beachtet werden wie sich Federn verhalten wenn sie in Serie oder parallel verbaut sind. \\ \\
f) \\ \\
Die Bewegungsgleichungen des vereinfachten Systems können anhand der Gleichungen aus Punkt d) leicht bestimmt werden.
\begin{align*}
	\ddot{\varphi_1} &= \frac{1}{\Theta_1}(M_1 + cr(s - r\varphi_1)) \\
	\ddot{s} &= \frac{1}{m_s}(-c(s - r\varphi_1) - \mu_V\dot{s})
\end{align*}