\textbf{Beispiel 4}\\ \\
a)\\ \\
Mithilfe der Bedingungen aus der Angabe folgen für die Randbedingungen
\begin{align*}
	T(0) &= T_0 \\
	\frac{\text{d}}{\text{d}x}T(x)\Bigl|_{x = L} &= 0 \qquad ...\text{adiabate Randbedingung}
\end{align*}
b)\\ \\
Unter Beachtung der Hinweise und des infinitesimalen Kontrollvolumen folgt für die einzelnen Wärmeströme
\begin{align*}
	\dot{Q}(x) &= -\lambda\pi\frac{D^2}{4}\frac{\text{d}}{\text{d}x}T(x) \\
	\dot{Q}(x + \text{d}x) &= 	\dot{Q}(x) - \lambda\pi\frac{D^2}{4}\frac{\text{d}^2}{\text{d}x^2}T(x)\text{d}x \\
	\text{d}\dot{Q}_M &= \alpha\pi D(T(x) - T_\infty)\text{d}x
\end{align*}
Somit folgt schließlich für die Differentialgleichung
\begin{align*}
	& \dot{Q}_x - \dot{Q}(x + \text{d}x) - \text{d}\dot{Q}_M = 0 \\
	& \Rightarrow \frac{\text{d}^2}{\text{d}x^2}T(x) - \frac{4\alpha}{\lambda D}\left(T(x) - T_\infty\right) = 0
\end{align*}
c)\\ \\
Verwendet man die geforderte Diskretisierung lautet die gegebene Differentialgleichung
\[
	\frac{T(x - \Delta x) - 2T(x) + T(x + \Delta x)}{\Delta x^2} + a_1T(x) + a_2 = 0
\]
Unter Einhaltung der gegebenen Form aus der Angabe und den Randbedingungen aus Punkt a) lauten die gesuchten Größen \\
\[
	\textbf{T} = \begin{bmatrix}
		T_1 \\
		T_2 \\
		T_3 
	\end{bmatrix}
	, \quad 
	\textbf{D} = \frac{1}{\Delta x^2}\begin{bmatrix}
		-2 & 1 & 0 \\
		1 & -2 & 1 \\
		0 & -2 & 2
	\end{bmatrix}
	, \quad 
	\textbf{b}_2 = a_2\begin{bmatrix}
		1 \\
		1 \\
		1
	\end{bmatrix}
	+ \frac{1}{\Delta x^2}
	\begin{bmatrix}
		T_0 \\
		0 \\
		0
	\end{bmatrix}
	, \quad 
	b_1 = a_1
\]