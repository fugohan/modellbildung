\textbf{Beispiel 1} \\ \\
a)\\ \\
Die generalisierten Koordinaten lauten
\[
	\textbf{q} = \left[ \begin{matrix}
		\varphi \\
		\psi
	\end{matrix}\right]
\]
Diese beiden sind deswegen die geeigneten generalisierten Koordinaten, weil diese sich mit der Zeit ändern können. \\ 
Die beiden Ortsvektoren $\textbf{r}_2$ und $\textbf{r}_K$ werden direkt aus Angabe abgelesen.
\[
	\textbf{r}_2 = \left[\begin{matrix}
		\frac{1}{2} L_2 + L_1 \sin\varphi \\
		-L_1\cos\varphi
	\end{matrix}\right] , 
	\textbf{r}_K = \left[\begin{matrix}
		\frac{1}{2} L_2 + L_1\sin\varphi + l\sin\psi \\
		-L_1\cos\varphi - l\cos\psi
	\end{matrix}\right]
\]
b) \\ \\
Den translatorische Geschwindigkeitsvektor erhält man durch die zeitliche Ableitung des entsprechenden Ortsvektors.
\[
	\dot{\textbf{r}}_K = \underbrace{L_1\dot{\varphi}\left[\begin{matrix}
		\cos\varphi \\
		\sin\varphi
	\end{matrix}\right]}_{\dot{\textbf{r}}_2}
	+
	l\dot{\psi}\left[\begin{matrix}
		\cos\psi\\
		\sin\psi
	\end{matrix}\right]
\]
c) \\ \\
Zwischenrechnungen für die Ermittlung der kinetischen Energie:
\begin{align*}
	\dot{\textbf{r}}_2^T\textbf{r}_2 &= L_1^2 \dot{\varphi}^2 \left[\begin{matrix}
	 \cos\varphi & \sin\varphi
	\end{matrix}\right]
	\left[\begin{matrix}
		\cos\varphi \\
		\sin\varphi
	\end{matrix}\right] \\
	&= L_1^2\dot{\varphi}^2\underbrace{\left(\cos^2\varphi + \sin^2\varphi\right)}_{=1} \\
	&= L_1^2\dot{\varphi}^2
\end{align*}
\begin{align*}
	\dot{\textbf{r}}_K^T\textbf{r}_K &= \left(L_1\dot{\varphi}\left[\begin{matrix}
		\cos\varphi &
		\sin\varphi
		\end{matrix}\right]
	+
	l\dot{\psi}\left[\begin{matrix}
	\cos\psi &
	\sin\psi
	\end{matrix}\right]\right)
	\left(L_1\dot{\varphi}\left[\begin{matrix}
		\cos\varphi \\
		\sin\varphi
		\end{matrix}\right]
	+
	l\dot{\psi}\left[\begin{matrix}
	\cos\psi\\
	\sin\psi
	\end{matrix}\right]\right) \\
	&= L_1^2\dot{\varphi}^2\underbrace{\left(\cos^2\varphi + \sin^2\varphi\right)}_{=1} + l^2\dot{\psi}^2\underbrace{\left( \cos^2\psi + \sin^2\psi\right)}_{=1} + 2L_1l\left(\sin\varphi\sin\psi + \cos\varphi\cos\psi\right) \\
	&= L_1^2\dot{\varphi}^2 + l^2\dot{\psi}^2 + 2L_1l\dot{\varphi}\dot{\psi}\left(\sin\varphi\sin\psi + \cos\varphi\cos\psi\right)
\end{align*}
\newpage
Nun können die kinetischen Energien des Systems bestimmt werden. Diese lauten hier
\begin{align*}
	T_{tr} &= \frac{1}{2} \left(m_2 + m_M\right) L_1^2\dot{\varphi}^2 + \frac{1}{2} m_K \left[ L_1^2\dot{\varphi}^2 + l^2\dot{\psi}^2 + 2L_1l\dot{\varphi}\dot{\psi}\left(\sin\varphi\sin\psi + \cos\varphi\cos\psi\right)\right] \\
	T_{rot} &= 2 \frac{1}{2} \left(\frac{1}{12}m_1L_1^2 + m_1\left(\frac{L_1}{2}\right)^2\right)\dot{\varphi}^2 \\
	T &= T_{tr} + T_{rot} \\
	  &= \frac{1}{2} \left(m_2 + m_M\right) L_1^2\dot{\varphi}^2 + \frac{1}{2} m_K \left[ L_1^2\dot{\varphi}^2 + l^2\dot{\psi}^2 + 2L_1l\dot{\varphi}\dot{\psi}\left(\sin\varphi\sin\psi + \cos\varphi\cos\psi\right)\right] \\
	  &+ 2\frac{1}{2} \left(\frac{1}{12}m_1L_1^2 + m_1\left(\frac{L_1}{2}\right)^2\right)\dot{\varphi}^2
\end{align*}
d) \\ \\
Nun kann wie folgt die potentielle Energie bestimmt werden. Die potentielle Energie der Ruhelage lautet
\begin{align*}
	V &= 2m_1g\frac{L_1}{2}\left(1-\cos\varphi\right) + \left(m_2 + m_M\right)gL_1 \left(1 - \cos\varphi\right) \\
	&+ m_Kg\left[L_1\left(1-\cos\varphi\right) +l\left(1 - \cos\psi\right)\right] + 2 \frac{1}{2}c_1\varphi^2
\end{align*}
\textit{Hinweis}: \\
Die Ruhelage befindet sich bei \(\varphi = 0\), d.h. das Maximum der potentiellen Energie tritt bei \(\varphi = \pm \frac{\pi}{2}\) und deswegen wird \(\cos\varphi\) durch \(\left(1-\cos\varphi\right)\) ersetzt. \\ \\
e) \\ \\
Um die Bewegungsgleichungen zu bestimmen, benötigt man als erstes aller erstes die Langrange-Funktion die wie folgt lautet
\begin{align*}
	L &= T - V \\
	  &= \frac{1}{2} \left(m_2 + m_M\right) L_1^2\dot{\varphi}^2 + \frac{1}{2} m_K \left[ L_1^2\dot{\varphi}^2 + l^2\dot{\psi}^2 + 2L_1l\dot{\varphi}\dot{\psi}\left(\sin\varphi\sin\psi + \cos\varphi\cos\psi\right)\right] 
	  + 2\frac{1}{2} \left(\frac{1}{12}m_1L_1^2 + m_1\left(\frac{L_1}{2}\right)^2\right)\dot{\varphi}^2 \\
	  &- 2m_1g\frac{L_1}{2}\left(1-\cos\varphi\right) - \left(m_2 + m_M\right)gL_1 \left(1 - \cos\varphi\right) - m_Kg\left[L_1\left(1-\cos\varphi\right) - l\left(1 - \cos\psi\right)\right] - 2 \frac{1}{2}c_1\varphi^2
\end{align*}
Als nächstes benötigt man die generalisierten Kräfte die hier wie folgt lauten
\[
	\textbf{f}_q = \left[\begin{matrix}
		-2d_1\varphi \\
		\tau
	\end{matrix}\right]
\]
Nun wertet man den Euler-Lagrange-Formalismus aus, der hier folgendermaßen aussieht
\begin{align*}
	\frac{d}{dt}\left(\frac{\partial L}{\partial \dot{\varphi}}\right) - \left(\frac{\partial L}{\partial \varphi}\right) &= -2d_1\dot{\varphi}\\
	\frac{d}{dt}\left(\frac{\partial L}{\partial \dot{\psi}}\right) - \left(\frac{\partial L}{\partial \psi }\right) &= \tau
\end{align*}
\newpage
\noindent
partiellen Ableitungen \\ \\
1. generalisierte Koordinate:
\begin{align*}
	\frac{\partial L}{\partial \dot{\varphi}} &= \left(m_2 + m_M\right)L_1^2\dot{\varphi} + m_K \left( L_1^2\dot{\varphi} + L_1l\dot{\psi}\left(\sin\varphi \sin\psi + \cos \varphi \cos\psi\right)\right) + 2\left(\frac{1}{12}m_1L_1^2 + m_1\left(\frac{L_1}{2}\right)^2\right)\dot{\varphi} \\
	\frac{d}{dt}\left(\frac{\partial L}{\partial \dot{\varphi}}\right) &= \left(m_2 + m_M\right)L_1^2 \ddot{\varphi} + m_K \left( L_1^2\ddot{\varphi} + L_1l\frac{d}{dt}\left(\dot{\psi}\left(\sin\varphi \sin\psi + \cos \varphi \cos\psi\right)\right)\right) + 2\left(\frac{1}{12}m_1L_1^2 + m_1\left(\frac{L_1}{2}\right)^2\right)\ddot{\varphi}\\
	&= \left( 2\left(\frac{1}{12}m_1L_1^2 + m_1\left(\frac{L_1}{2}\right)^2\right) + \left( m_2 + m_M + m_K\right)\right)\ddot{\varphi} + m_K L_1 \frac{d}{dt}\left(\psi\left(\sin\varphi \sin\psi + \cos\varphi \cos\psi\right)\right) \\
	\frac{\partial L}{\partial \varphi} &= m_KL_1l\dot{\varphi}\dot{\psi}\left(\cos\varphi \sin\psi + \sin\varphi \cos\psi\right)-m_1gL_1\sin\varphi - \left(m_2 + m_M\right)L_1g - m_KgL_1\sin\varphi - 2c_1\varphi \\
	&= m_KL_1l\dot{\varphi}\dot{\psi}\left(\cos\varphi \sin\psi + \sin\varphi \cos\psi\right) - \left(m_1 + m_M + m_2 + m_K\right)gL_1 - 2c_1\varphi
\end{align*}
2. generalisierte Koordinate:
\begin{align*}
	\frac{\partial L}{\partial \dot{\psi}} &= m_Kl^2\dot{\psi} + m_KL_1 \dot{\varphi}\left(\sin\varphi \sin\psi + \cos\varphi \cos\psi\right) \\
	\frac{d}{dt} \left(\frac{\partial L}{\partial \dot{\psi} }\right) &= m_Kl^2\ddot{\psi} + m_KL_1 \frac{d}{dt}\left(\dot{\varphi}\left(\sin\varphi \sin\psi + \cos\varphi \cos\psi\right)\right) \\
	\frac{\partial L}{\partial \psi} &= m_KL_1l\dot{\varphi}\dot{\psi}\left(\sin\varphi \cos\psi - \cos\varphi \sin\psi\right) - m_Kgl\sin\psi
\end{align*}
Somit lauten die Bewegungsgleichungen
\begin{align*}
	\left(2\left(\frac{1}{12}m_1L_1^2 + m_1\left(\frac{L_1}{2}\right)^2\right) + \left( m_2 + m_M + m_K\right)\right)\ddot{\varphi} + m_K L_1 \frac{d}{dt}\left(\psi\left(\sin\varphi \sin\psi + \cos\varphi \cos\psi\right)\right) \\
	- m_KL_1l\dot{\varphi}\dot{\psi}\left(\cos\varphi \sin\psi + \sin\varphi \cos\psi\right) + \left(m_1 + m_M + m_2 + m_K\right)gL_1 + 2c_1\varphi = -2d_1\dot{\varphi}
\end{align*}
\begin{align*}
	m_Kl^2\ddot{\psi} + m_KL_1 \frac{d}{dt}\left(\dot{\varphi}\left(\sin\varphi \sin\psi + \cos\varphi \cos\psi\right)\right) \\
	- m_KL_1l\dot{\varphi}\dot{\psi}\left(\sin\varphi \cos\psi - \cos\varphi \sin\psi\right) + m_Kgl\sin\psi = \tau
\end{align*}