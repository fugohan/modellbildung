\textbf{Beispiel 1}\\ \\
a)\\ \\
Grundsätzlich gilt für den Riemen hier
\[
	l_1 + l_2 = l
\]
Durch Einzeichnen zweier rechtwinkeligen Dreiecke in die Angabe gilt für die Länge 1 
\[
	l_1 = \sqrt{(x_R + a)^2 + y_R^2}
\]
und die Länge 2
\[
	l_2 = \sqrt{(x_R - a)^2 + y_R^2}
\]
Durch Einsetzen in die obige Bedingungen folgt
\[
	\sqrt{(x_R + a)^2 + y_R^2} + \sqrt{(x_R - a)^2 + y_R^2} - l = 0
\]
Durch Auflösen nach $y_R$ erhält man schließlich
\[
	y_R = \pm \frac{\sqrt{l^2 - 4a^2}}{2l}\sqrt{l^2 - 4x_R^2}
\]
b)\\ \\
Die gesuchten Größen lauten für die Rolle
\[
	\textbf{p}_R = \begin{bmatrix}
		x_R \\
		-\kappa\sqrt{l^2 - 4x_R^2}
	\end{bmatrix}
	\quad,\quad
	\dot{\textbf{p}}_R = \begin{bmatrix}
		\dot{x}_R \\
		\kappa\frac{4x_R\dot{x}_R}{\sqrt{l^2 - 4x_R^2}}
	\end{bmatrix}
\]
und für die Last
\[
	\textbf{r}_L = \begin{bmatrix}
		x_R + r\sin(\varphi) \\
		-\kappa\sqrt{l^2 - 4x_R^2} - r\cos(\varphi)
	\end{bmatrix}
	\quad,\quad
	\dot{\textbf{p}}_R = \begin{bmatrix}
	\dot{x}_R + r\cos(\varphi)\dot{\varphi}\\
	\kappa\frac{4x_R\dot{x}_R}{\sqrt{l^2 - 4x_R^2}} + r\sin(\varphi)\dot{\varphi}
	\end{bmatrix}
\]
c)\\ \\
Mit den Geschwindigkeiten aus Punkt b) folgt für die kinetische Energie des Systems
\begin{align*}
	T(\dot{q},\dot{\textbf{q}}) = &+\frac{1}{2}\theta_L\dot{\varphi}^2 + \frac{1}{2}m_R\left(\dot{x}_R^2 + \kappa^2\frac{16x^2_R\dot{x}^2_R}{l^2 - 4x_R}\right) \\
	&+ \frac{1}{2}m_L\left(\dot{x}_R^2 + 2\dot{x}_R^2r\cos(\varphi)\dot{\varphi} + r^2\dot{\varphi}^2 + \kappa^2\frac{16x^2_R\dot{x}^2_R}{l^2 - 4x_R} + 2\kappa r\frac{4x_R\dot{x}_R}{\sqrt{l^2 - 4x_R^2}}\sin(\varphi)\dot{\varphi}\right)
\end{align*}
d)\\ \\
Die gesamte potentielle Energie des Systems beläuft sich auf
\[
	V(\textbf{q},\dot{\textbf{q}}) = -(m_R + m_L)g\kappa\sqrt{l^2 - 4x_R^2} - m_Lgr\cos(\varphi)
\]
Hierzu wurden die y-Koordinaten der Ortsvektoren aus Punkt b) verwendet.
\newpage
\noindent
e)\\ \\
Der Vektor der externen Kraft lautet
\[
	\textbf{f}_{ext} = \begin{bmatrix}
		F_{ext} \\
		0
	\end{bmatrix}
\]
Der Vektor zum Angriffspunkt dieser Kraft wiederum lautet
\[
	\textbf{r}_{f_{ext}} = \begin{bmatrix}
		x_R + r\sin(\varphi) \\
		-\kappa\sqrt{l^2 - 4x_R^2} - r\cos(\varphi)
	\end{bmatrix}
\]
Die notwendigen Ableitungen lauten
\[
	\frac{\partial 	\textbf{r}_{f_{ext}}}{\partial x_R} = \begin{bmatrix}
		1 \\
		\kappa\frac{4x_R}{\sqrt{l^2 - 4x_R^2}}
	\end{bmatrix}
	\quad,\quad
		\frac{\partial 	\textbf{r}_{f_{ext}}}{\partial \varphi} = \begin{bmatrix}
		r\cos(\varphi) \\
		r\sin(\varphi)
	\end{bmatrix}
\]
Mit der Rechenvorschrift aus der Formelsammlung ergibt sich nun für den Vektor der generalisierten Kraft
\[
	\textbf{f}_q = F_{ext}\begin{bmatrix}
		1 \\
		r\cos(\varphi)
	\end{bmatrix}
\]
f)\\ \\
Durch den Ansatz des Euler-Lagrange-Formalismus und den Bedingungen aus der Angabe die erfüllt sein müssen, muss hier nur eine Ableitung getätigt werden aus der man die Lösung direkt ableiten kann. Dieses lautet 
\[
	\frac{\partial L}{\partial \varphi} = m_Lgr\sin(\varphi)
\]
Mit $L = T - V$ und sämtlich anderen Terme der Ableitung inklusive dieser fallen aufgrund der geforderten Bedingungen weg. Somit lautet die stationären Punkte des Systems
\[
	\textbf{q}_{R,1}^T = \begin{bmatrix}
		0 & 0
	\end{bmatrix}
	\quad,\quad
	\textbf{q}_{R,2}^T = \begin{bmatrix}
	0 & \pi
	\end{bmatrix}
\]
g)\\ \\
Eine andere Option für die generalisierten Koordinaten wäre
\[
	\textbf{q}^T = \begin{bmatrix}
		\psi & \varphi
	\end{bmatrix}
\]
Mit
\[
	\psi = \arctan\left(\frac{y_R}{x_R}\right)
\]