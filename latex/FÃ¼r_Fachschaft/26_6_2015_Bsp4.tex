\textbf{Beispiel 4} \\ \\
a) \\ \\
Die Wärmestromdichte $\dot{q}$ über die Schichtdicke $d_m$ lautet
\[
	\dot{q} = \frac{\lambda \left(T_K - T_O\right)}{d_m}
\]
b) \\ \\
Hier verwendet man die Formel für die Wärmestromdichte bei Konvektion. Diese lautet
\[
	\dot{q} = \alpha \left(T - T_\infty\right)
\]
Auf dieses Beispiel angewandt folgt
\[
	\dot{q} = \alpha \left(T_O - T_U\right)
\]
c) \\ \\
Die Zeichnung kann mit Abbildung 6 in der Angabe verglichen werden. Das erste thermische Widerstandspaar in Abbildung 6 ist die zu zeichnende Schaltung.\\
Analog zum spezifischen elektrischen Widerstand
\[
	R_{el}= \frac{1}{\gamma}\frac{l}{A}
\]
lässt sich der thermische Wiederstand eines Körpers wie folgt bestimmen.
\[
	R_{th} = \frac{1}{\lambda}\frac{l}{A}
\]
Daraus folgt
\[
	R_M = \frac{d_M}{\lambda A}
\]
und
\[
	R_A = \frac{1}{\alpha A}
\]
d) \\ \\
Aus Abbildung 6 lassen sich sämtliche Gleichungen aufstellen die man benötigt um $B_0$ zu berechnen.\\
Gleichungen:
\begin{align*}
	\dot{Q} &= \dot{Q}_1 + \dot{Q}_2 \\
	\left(R_K + R_S + R_F\right) \dot{Q}_2 &= T_{OS} - T_U \\
	R_A\dot{Q}_1 &= T_{OS} - T_U \\
	R_M\dot{Q} + R_A\dot{Q}_1 &= T_K - T_U
\end{align*}
Durch Umformen folgt schließlich der Kopplungsfaktor:
\begin{align*}
R_A\dot{Q}_1 = T_{OS} - T_U \\
	\dot{Q}_1 = \frac{T_{OS} - T_U}{R_M} \\
	\left(R_K + R_S + R_F\right) \dot{Q}_2 = T_{OS} - T_U \\
	\dot{Q}_2 = \frac{T_{OS} - T_U}{R_K + R_S + R_F} \\
	R_M\dot{Q} + R_A\dot{Q}_1 = T_K - T_U \\
	R_M\left(\dot{Q}_1 + \dot{Q}_2\right) + R_A\dot{Q}_1 = T_K - T_U \\
	\left(\frac{R_M}{R_A} + \frac{R_M}{R_K + R_S + R_F}\right)\left(T_{OS} - T_U\right) = T_K - T_U \\
	B_0 = \frac{T_{OS} - T_U}{T_K - T_U} = \frac{R_A(R_K + R_S + R_F)}{R_M(R_K + R_S + R_F) + R_AR_M}
\end{align*}