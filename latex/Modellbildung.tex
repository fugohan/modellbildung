\documentclass[a4paper,12p]{article}
\usepackage[utf8]{inputenc}
\usepackage[ngerman]{babel}
\usepackage[a4paper, left=2.5cm, right=2.5cm]{geometry}
\usepackage{amsmath}
\usepackage{graphicx}
\usepackage{mathtools}
\usepackage{tikz}
\usepackage{cancel}
\usetikzlibrary{intersections}
\usetikzlibrary{calc,arrows.meta}  

\title{\huge Modellbildung\\\large \huge Beispielsammlung}
\author{\huge 4.Semester ET-Studium}
\date{\huge November 2018}
\begin{document}	
\maketitle
\newpage
\tableofcontents
\newpage
\section{Einleitung}
Die Modellbildung beschäftigt sich mit der mathematischen Beschreibung von Systemen, um diese dann beispielsweise mit Hilfe eines Reglers zu automatisieren.
\newline
Der Inhalt dieser Ausarbeitung beinhaltet nur die Aufgaben, aus dem Skript der Vorlesung der Modellbildung, welche eine Pflichtlehrveranstaltung im 4.Semester im Bachelor-Studium der Elektrotechnik ist. Der Lösungsweg der hier erwähnten Beispiele wird im Skript nicht explizit ausgearbeitet. Variablen der Form \textbf{x} sind Vektoren und der Form \textbf{A} sind Matrizen. \\
Die Ausarbeitung enthält nur die Beispiele die ohne nötiger Hilfe von Maple gelöst werden können und wo man keine mathematischen Beweise durchführen muss. \\
Außerdem beinhaltet dieses Dokument weiters komplett durchgerechnete Prüfungen. \\
Hier wird keine Theorie ausgearbeitet, sondern außerschließlich nur Berechnungsaufgaben. \\
ACHTUNG nur Rechnungen in dieser Ausarbeitung. Sämtliche Bilder zu den Angaben und Angaben finden Sie im Vorlesungskript.\\
ACHTUNG nicht alle Beispiele und Prüfungen befinden sich durchgerechnet in dieser Sammlung.
\section{Mechanische Systeme}
\subsection{Punkt-Kinematik}
\textbf{\textit{\underline{Aufgabe 2.1}}} \\ \\
 \textbf{Lösungsweg:}
 \newline\newline
 Die Geschwindigkeit mit der Form
 \begin{equation*}
 	\textbf{v}_{t} = v_{x}\textbf{e}_{x} + v_{y}\textbf{e}_{y} + v_{z}\textbf{e}_{z} = \dot{x}\textbf{e}_{x} + \dot{y}\textbf{e}_{y} + \dot{z}\textbf{e}_{z}
 \end{equation*}
 erhält man, durch die Anwendung der Kettenregel der Differentiation hier mit der Form
 \begin{equation*}
 	\textbf{v}(t) = \left(\frac{\partial}{\partial r}\textbf{r}\right) \dot{r} \ + \ \left(\frac{\partial}{\partial \varphi}\textbf{r}\right)  \dot{\varphi} \ + \ \left(\frac{\partial}{\partial \theta}\textbf{r}\right)\dot{\theta}
 \end{equation*}
 Die Basisvektoren bilden sich folgendermaßen aus
 \begin{align*}
 	\tilde{\textbf{e}}_{r} & = \sin(\theta)\cos(\varphi)\textbf{e}_{x} \ + \ \sin(\theta)\sin(\varphi)\textbf{e}_{y} \ + \ \cos(\theta)\textbf{e}_{z} \\ 
 	\tilde{\textbf{e}}_{\theta} & = r\cos(\theta)\cos(\varphi)\textbf{e}_{x} \ + \ r\cos(\theta)\sin(\varphi)\textbf{e}_{y} \ - \ r\sin(\theta)\textbf{e}_{z} \\
 	\tilde{\textbf{e}}_{\varphi} & = -r\sin(\theta)\sin(\varphi)\textbf{e}_{x} \ + \ r\sin(\theta)\cos(\varphi)\textbf{e}_{y} 	
 \end{align*}
 Diese Vektoren können nur dann eine zulässige Basis eines Koordinatensystems sein, wenn die Matrix
\begin{equation*}
	\textbf{J} =
	 \begin{bmatrix}
	 	\sin(\theta)\cos(\varphi) & \sin(\theta)\sin(\varphi) & \cos(\theta) \\
	 	r\cos(\theta)\cos(\varphi) & r\cos(\theta)\sin(\varphi) & -r\sin(\theta) \\
	 	-r\sin(\theta)\sin(\varphi) & r\sin(\theta)\cos(\varphi) & 0
	 \end{bmatrix}
\end{equation*}
\\
regulär ist. D.h. die Determinante dieser Matrix muss $\neq 0$  sein.
Die Matrix \textbf{J} wird folgendermaßen gefüllt: \\
\begin{equation*}
	\textbf{J} = 
		\begin{bmatrix}
			x_r & y_r & z_r \\
			x_\theta & y_\theta & z_\theta \\
			x_\varphi & y_\varphi & z_\varphi \\
		\end{bmatrix}
\end{equation*}
Hier det(\textbf{J}) = $r\sin\theta \neq 0$ wenn $r \neq 0$ und es gilt $\theta: [0,\frac{\pi}{2}]$ $\Rightarrow$ det(\textbf{J}) $\in$ regulär.\\
\underline{Beweis:} \\
Da es sich hier um eine $ 3 \times 3 $-Matrix handelt kann die Determinante dieser Matrix mithilfe des Satzes von Sarus gelöst werden.
\begin{align*}
	det(\textbf{J}) & = r^2\sin^3(\theta)\sin^2(\varphi) + r^2\cos^2(\theta)\cos^2(\varphi)\sin(\theta) + r^2\cos^2(\theta)\sin^2(\varphi)\sin(\theta) + r^2\sin^3(\theta)\cos^2(\varphi) =\\
	& = r^2(\sin^3(\theta)\sin^2(\varphi) + \cos^2(\theta)\cos^2(\varphi)\sin(\theta) + \cos^2(\theta)\sin^2(\varphi)\sin(\theta) + \sin^3(\theta)\cos^2(\varphi)) =\\
	& = r^2(\cos^2(\theta)\cos^2(\varphi)\sin(\theta) + \cos^2(\theta)\sin^2(\varphi)\sin(\theta) + \sin^3(\theta)(\sin^2(\varphi) + \cos^2(\varphi)) =\\
	& = r^2(\cos^2(\theta)\cos^2(\varphi)\sin(\theta) + \cos^2(\theta)\sin^2(\varphi)\sin(\theta) + \sin^3(\theta)) =\\
	& = r^2(\sin^3(\theta) + \cos^2(\theta)\sin(\theta)(\cos^2(\varphi) + \sin^2(\varphi))) =\\
	& = r^2(\sin^3(\theta) + \cos^2(\theta)\sin(\theta)) =\\
	& = r^2(\sin(\theta)\sin^2(\theta) + \cos^2(\theta)\sin(\theta)) = \\
	& = r^2(\sin(\theta)(\sin^2(\theta) + \cos^2(\theta))) = \\
	& = r^2\sin(\theta) \\ \\
	det(\textbf{J}) & = r^2\sin(\theta)
\end{align*}
Anschließend muss man die gerade ermittelten Basisvektoren normieren.\newline
Die Längen der Vektoren $\tilde{\textbf{e}}_{r},\tilde{\textbf{e}}_{\varphi}$ und $\tilde{\textbf{e}}_{\theta}$ betragen $1,r\sin\theta$ und $r$.
\newline
Normierte Basisvektoren:
\begin{align*}
	\textbf{e}_{r} & = \sin(\theta)\cos(\varphi)\textbf{e}_{x} \ + \ \sin\theta)\sin(\varphi)\textbf{e}_{y} \ + \ \cos(\theta)\textbf{e}_{z} \\
	\textbf{e}_{\theta} & = \cos(\theta)\cos(\varphi)\textbf{e}_{x} \ + \ \cos(\theta)\sin(\varphi)\textbf{e}_{y} \ - \sin(\theta)\textbf{e}_{z} \\
	\textbf{e}_{\varphi} & = -\sin(\varphi)\textbf{e}_{x} \ + \ \cos(\varphi)\textbf{e}_{y}
\end{align*}
Mit diesen Basisvektoren lässt sich die Geschwindigkeit in der Form
\begin{equation*}
	\textbf{v}(t) = v_{r}\textbf{e}_{r} + v_{\theta}\textbf{e}_{\theta} + v_{\varphi}\textbf{e}_{\varphi} = \dot{r}\textbf{e}_{r} + r\dot{\theta}\textbf{e}_{\theta} +  r\sin(\theta)\dot{\varphi}\textbf{e}_{\varphi}
\end{equation*}
darstellen.
\begin{equation*}
	v_r = \dot{r} \ , \ v_\theta = r\dot{\theta} \ , \ v_\varphi = r\sin(\theta)\dot{\varphi}
\end{equation*}
\newpage
\subsection{Newtonsche Gesetze}
\subsubsection{Kräftesystem}
\textbf{\textit{\underline{Aufgabe 2.3}}} \\ \\
\textbf{Lösungsweg:} \\ \\
Zunächst müssen die Gleichungen für die Gleichgewichtsbedingung aufgestellt werden. Da es sich um ein 3-dimensionales Beispiel handelt muss man die Gleichgewichtsbedingung für alle drei Raumrichtungen aufstellen.
\begin{align*}
	(1) \ \ \ e_x & : f_{s1}\cos(\alpha) - f_{s2}\cos(\alpha) = 0 \\
	(2) \ \ \ e_y & : f_{s3}\cos(\gamma) - f_{s1}\cos(\alpha)\cos(\beta) - f_{s2}\cos(\alpha)\cos(\beta) = 0 \\  
	(3) \ \ \ e_z & : -f_M - f_{s3}\sin(\gamma) - f_{s2}\cos(\alpha)\sin(\beta) - f_{s1}\cos(\alpha)\sin(\beta) = 0
\end{align*}
Nun hat man ein einfaches Gleichungssystem mit 3 Gleichungen, welches nach diesen auch gelöst werden soll. Als erstes formt man die Gleichung (1) auf $f_{s1}$ um. \\ 
Hieraus folgt:
\begin{equation*}
	f_{s1} = f_{s2}
\end{equation*}
Im nächsten Schritt setzt man die umgeformte Gleichung (1) in Gleichung (2) ein und formt diese auf $ f_{s1} $ um. \\
Hieraus folgt:
\begin{align*}
	f_{s3}&\cos(\gamma) - f_{s1}\cos(\alpha)\cos(\beta) - f_{s1}\cos(\alpha)\cos(\beta) = 0 \\
	f_{s3}&\cos(\gamma) - 2f_{s1}\cos(\alpha)\cos(\beta) = 0 \\
	f_{s1}&=f_{s2} = \frac{f_{s3}\cos(\gamma)}{2\cos(\alpha)\cos(\beta)}
\end{align*}

\begin{flushleft}
    \textbf{\textit{\underline{Aufgabe 2.4}}}
\end{flushleft}
Geben Sie die Momentenbilanz für das Beispiel von Abbildung 2.14 um den Bezugspunkt \textit{C} an.  \newline
\begin{flushleft}
Ausgehend von den Kräften
\end{flushleft}
\begin{equation*}
	\textbf{f}_{A} = \left[\begin{array}{c}	f_{A,x} \\	f_{A,y} \\	f_{A,z} \\\end{array}\right] , 
	\textbf{f}_{B} = \left[\begin{array}{c}	f_{B,x} \\	f_{B,y} \\	f_{B,z} \\\end{array}\right] ,
	\textbf{f}_{C} = \left[\begin{array}{c}	f_{C,x} \\	f_{C,y} \\	f_{C,z} \\\end{array}\right]
\end{equation*}
sowie die zugehörigen Ortsvektoren
\begin{equation*}
	\textbf{r}_{0A} = \left[\begin{array}{c} \frac{a_{x}}{2} \\ 0 \\ \frac{a_{z}}{2} \\\end{array}\right] ,
	\textbf{r}_{0B} = \left[\begin{array}{c} \frac{a_{x}}{2} \\ \frac{a_{y}}{2} \\ a_{z} \\\end{array}\right]
	\textbf{r}_{0C} = \left[\begin{array}{c} 0 \\ 0 \\ 0 \\\end{array}\right]
	\textbf{r}_{0D} = \left[\begin{array}{c} a_{x} \\ 0 \\ 0 \\\end{array}\right]
\end{equation*}
\newpage
\begin{flushleft}
	Für den Bezugspunkt \textit{C} folgen die Momente zu
\end{flushleft}
\begin{align*}
	\boldsymbol{\tau}_{A}^{(C)} &= \underbrace{(\textbf{r}_{0A} - \textbf{r}_{0C})}_{\textbf{r}_{CA}} \times \textbf{f}_{A} = \left[\begin{array}{c} \frac{a_{x}}{2} \\ 0 \\ \frac{a_{z}}{2} \\\end{array}\right] \times \left[\begin{array}{c} f_{A,x} \\ f_{A,y} \\ f_{A,z} \\\end{array}\right] = \left[\begin{array}{c}
		-\frac{f_{A,y}a_{z}}{2} \\
		-\frac{f_{A,z}a_{x}}{2} + \frac{f_{a,x}a_{z}}{2} \\
		\frac{f_{A,y}a_{x}}{2} \\
	\end{array}\right]
	\\
	\boldsymbol{\tau}_{B}^{(C)} &= \underbrace{(\textbf{r}_{0B} - \textbf{r}_{0C})}_{\textbf{r}_{CB}} \times \textbf{f}_{B} = 
	\left[\begin{array}{c}
		\frac{a_{x}}{2} \\
		\frac{a_{y}}{2} \\
		a_{z} \\
	\end{array}\right]
	\times
	\left[
	\begin{array}{c}
		f_{B,x} \\
		f_{B,y} \\
		f_{B,z} \\
	\end{array}
	\right]
	=
	\left[\begin{array}{c}
	\frac{f_{B,y}a_{z}}{2} - f_{B,y}a_{z} \\
	-\frac{f_{B,z}a_{x}}{2} + f_{B,x}a_{z} \\
	\frac{f_{B,y}a_{x}}{2} - \frac{f_{B,x}a_{y}}{2} \\
	\end{array}\right]
	\\
	\boldsymbol{\tau}_{C}^{(C)} &= \underbrace{(\textbf{r}_{0C} - \textbf{r}_{0C})}_{\textbf{r}_{CC}} \times \textbf{f}_{C} = 
	\left[\begin{array}{c}
			0 \\
			0 \\
			0 \\
	\end{array}\right]
	\times
	\left[
	\begin{array}{c}
	f_{C,x} \\
	f_{C,y} \\
	f_{C,z} \\
	\end{array}
	\right]
	=
	\left[\begin{array}{c}
		0 \\
		0 \\
		0 \\
	\end{array}\right]
\end{align*}
\subsubsection{Schwerpunkt}
\begin{flushleft}
	\underline{\textbf{\textit{Aufgabe 2.6}}} \\
\end{flushleft}
\textbf{Lösungsweg:} \\
Schwerpunkt einer homogenen Halbkugel: \\
\begin{align*}
	dV &= \int_{0}^{2\pi}\int_{0}^{\frac{\pi}{2}}\int_{0}^{r}R^2\sin(\theta)dr d\theta d\varphi = \\
	dV &= \frac{2r^3\pi}{3}
\end{align*}
\noindent
Koordinaten des Ortsvektors zum Schwerpunkt:
\begin{align*}
	r_{Sx} &= \int_{0}^{2\pi}\int_{0}^{\frac{\pi}{2}}\int_{0}^{r}r^3\sin(\theta)\cos(\varphi)\sin(\theta)dr d\theta d\varphi = 0 \\
	r_{Sy} &= \int_{0}^{2\pi}\int_{0}^{\frac{\pi}{2}}\int_{0}^{r}r^3\sin(\theta)\sin(\varphi)\sin(\theta)dr d\theta d\varphi = 0 \\
	r_{Sz} &= \int_{0}^{2\pi}\int_{0}^{\frac{\pi}{2}}\int_{0}^{r}r^3\cos(\theta)\sin(\theta)dr d\theta d\varphi = \frac{\pi r^4}{4}
\end{align*}
\noindent
Koordinaten als Vektor:
\[
	\textbf{r}_{SK} = 
	\left[
		\begin{array}{c}
			0 \\
			0 \\
		    \frac{\pi r^4}{4} \\
		\end{array}
	\right]
\]
Schwerpunkt:
\[
	\textbf{r}_{S} = \frac{\textbf{r}_{SK}}{dV} = 
	\left[
		\begin{array}{c}
			0 \\
			0 \\
			\frac{3}{8}r \\
		\end{array}
	\right]
\]
\subsubsection{Impulserhaltung}
Die Aufgaben in diesem Kapitel beinhalten nur Beweise, aber diese Ausarbeitung behandelt keine mathematischen Beweise.
\subsubsection{Translatorische kinetische Energie und potentielle Energie  }
\underline{\textbf{\textit{Aufgabe 2.9}}}
\begin{flushleft}
	\textbf{Lösungsweg:}
\end{flushleft}
Serienschaltung:\\
Die entspannten Längen werde addiert und die Steifigkeit wird wie einer einer "Parallelschaltung" berechnet.
\[
	s_{0g} = s_{10} + s_{20}, \qquad c_{g} = \frac{c_{1}c_{2}}{c_{1} + c_{2}}
\]
\begin{flushleft}
Parallelschaltung:
\end{flushleft}
Steifigkeit:
\[ c_{g} = c_{1} + c_{2}\]
entspannte Länge:
\begin{align*}
	F_{F1} &= c_{1}s_{10} \\
	F_{F2} &= c_{2}s_{20} \\
	F_{g0} = F_{F1} + F_{F2} &= c_{1}s_{10} + c_{2}s_{20} \\
	s_{g0} = \frac{F_{g0}}{c_{g}} &= \frac{c_{1}s_{10} + c_{2}s_{20}}{c_{1} + c_{2}}
\end{align*}
\subsubsection{Dissipative Kräfte }
\underline{\textbf{\textit{Aufgabe 2.10}}}\\ \\
Gleichgewichtsbedingungen:
\begin{align*}
	\textbf{e}_{x} &: -f_{s}\cos(\beta) + f_{R}\cos(\alpha) + f_{N}\sin(\alpha) = 0 \\
	\textbf{e}_{y} &: -mg + f_{s}\sin(\beta) - f_{r}\sin(\alpha) + f_{N}\cos(\alpha) = 0 \\
\end{align*}
Haftreibungskraft $f_{R} = f_{N}\mu_{H}$ \\
Setzt man diese in die obigen Bedingungen lässt sich folgendes Gleichungssystem ermitteln: \\
$\textbf{e}_{x}:$
\begin{align*}
	 -f_{S}\cos(\beta) + f_{N}\mu_{H}\cos(\alpha) + f_{N}\sin(\alpha) = 0 \\
	 -f_{S}\cos(\beta) + f_{N}(\mu_{H}\cos(\alpha) + \sin(\alpha)) = 0 \\
\end{align*}
$\textbf{e}_{y}:$
\begin{align*}
	-mg + f_{S}\sin(\beta) - f_{N}\mu_{H}\sin(\beta) + f_{N}\cos(\alpha) = 0 \\
	-mg + f_{S}\sin(\beta) - f_{N}(\mu_{H}\sin(\alpha) + \cos(\alpha)) = 0 \\
\end{align*}
Gleichungssystem:
\begin{align*}
	I  &: -f_{S}\cos(\beta) + f_{N}(\mu_{H}\cos(\alpha) + \sin(\alpha)) = 0 \\
	II &: 	-mg + f_{S}\sin(\beta) - f_{N}(\mu_{H}\sin(\alpha) + \cos(\alpha)) = 0 \\
\end{align*}
Nun wird ganz normal dieses Gleichungssystem gelöst.\\
I auf $f_{N}$ umgeformt:
\[
	f_{N} = \frac{f_{S}\cos(\beta)}{\mu_{H}\cos(\alpha) + \sin(\alpha)}
\]
I in II eingesetzt und auf $f_{S}$ umgeformt:
\begin{align*}
	&-mg + f_{S}\sin(\beta) - \frac{f_{S}\cos(\beta)}{\mu_{H}\cos(\alpha) + \sin(\alpha)}(\mu_{H}\sin(\alpha) + \cos(\alpha)) = 0 \\
	&-mg(\mu_{H}\cos(\alpha) + \sin(\alpha)) + f_{S}\sin(\beta)(\mu_{H}\cos(\alpha) + \sin(\alpha)) - f_{S}\cos(\beta)(\mu_{H}\sin(\alpha) + \cos(\alpha)) = 0 \\
	&-mg(\mu_{H}\cos(\alpha) + \sin(\alpha)) + f_{S}(\mu_{H}\sin(\beta)\cos(\alpha) + \sin(\beta)\sin(\alpha)) - f_{S}(\mu_{H}\cos(\beta)\sin(\alpha) + \cos(\alpha)\cos(\beta)) = 0 \\
	&-mg(\mu_{H}\cos(\alpha) + \sin(\alpha)) + f_{S}(\mu_{H}\underbrace{(\sin(\beta)\cos(\alpha) - \cos(\beta)\sin(\alpha))}_{\sin(\beta - \alpha)} + \underbrace{(\sin(\beta)\sin(\alpha) + \cos(\beta)\cos(\alpha))}_{\cos(\beta - \alpha)}) = 0 \\
	&-mg(\mu_{H}\cos(\alpha) + \sin(\alpha)) + f_{S}(\mu_{H}\sin(\beta - \alpha) + \cos(\beta - \alpha)) = 0 \\
	&f_{S} = \frac{mg(\mu_{H}\cos(\alpha) + \sin(\alpha))}{\cos(\beta - \alpha) + \mu_{H}\sin(\beta - \alpha)}
\end{align*}
\newline
Der Körper haftet bei der Bedingung \( |f_{S}|\leq f_{H}\).\\
Damit der Körper sich nun bewegen kann muss \(|f_{S}| > f_{H}\) erfüllt. Daraus folgt für \(f_{S}\):
\[
	f_{S} > \frac{mg(\mu_{H}\cos(\alpha) + \sin(\alpha))}{\cos(\beta - \alpha) + \mu_{H}\sin(\beta - \alpha)}
\] 
\newpage
\renewcommand\thesection{\Alph{section}} % Umstellung des Section-Zählung auf Buchstaben
\setcounter{section}{1} % Damit die neue Section bei B beginnt
\section{Aufgaben zum Selbststudium}
Sämtliche Angaben zu diesem Beispielen finden Sie im Skript dieser Lehrveranstaltung. Diese Zusammenfassung soll die Lösungen aus dem Skript vereinfachen.
\subsection{Klappbrücke}
Anfangs werden der Träger und die Brücke einmal frei geschnitten. \\
Gleichgewichtsbedingungen für die Brücke:
\begin{align*}
	\textbf{e}_{x} &: f_{B,x} = 0 \\
	\textbf{e}_{y} &: f_{B,y} - m_Bg + f_{s1} = 0 \\
	\textbf{e}_{z} &: -m_Bgl_B\cos\varphi_B  + f_{s1}l_s\cos\varphi_B = 0
\end{align*}
Mit der Gleichgewichtsbedingung $\textbf{e}_{z}$ wird nun die die Seilkraft $f_{s1}$ ermittelt.
\begin{align*}
	-m_Bgl_B\cos\varphi_B  + f_{s1}l_s\cos\varphi_B &= 0 \\
	-m_Bgl_B + f_{s1}l_s &= 0 \\
	f_{s1}l_s &= m_Bl_B \\
	f_{s1} &= m_B\frac{l_B}{l_s}
\end{align*}
Die Seilkraft beträgt nun
\begin{equation*}
	f_{s1} = m_B\frac{l_B}{l_s}
\end{equation*}
Mit der Seilkraft $f_{s1}$ und der Gleichgewichtsbedingung $e_y$ schließt man:
\begin{align*}
	f_{B,y} - m_Bg + m_B\frac{l_B}{l_s} &= 0  \\
	f_{B,y} = + m_Bg - m_B\frac{l_B}{l_s} &= m_Bg \left(1 - \frac{l_B}{l_S}\right)
\end{align*}
Die Kraft in y-Richtung von Lager B beträgt nun:
\begin{equation*}
	f_{B,y} = m_Bg\left(1 - \frac{l_B}{l_S}\right)
\end{equation*}
Die Lagerkraft B $f_B$ setzt sich nun aus $f_{B,x}$ und $f_{B,y}$ zusammen.
\begin{align*}
	\textbf{f}_B = \left[ \begin{matrix}
		0 \\
		m_Bg\left(1 - \frac{l_B}{l_S}\right)
	\end{matrix}\right]
\end{align*}
\newpage
\noindent
Gleichgewichtsbedingung für den Träger:
\begin{align*}
\textbf{e}_{x} &: f_{A,x} = 0 \\
\textbf{e}_{y} &: -mg - f_{s2} + f_{A,y} - m_tg - f_{s1} = 0 \\
\textbf{e}_{z} &: mg\cos\varphi_B l_m + f_{s2}R - m_Tg\cos\varphi_Bl_T - f_{s1}\cos\varphi_Bl_s= 0
\end{align*}
Mit der Gleichgewichtungbedingung $\textbf{e}_x$ kann man $f_{s2}$ bestimmen.
\begin{align*}
	mg\cos\varphi_B l_m + f_{s2}R - m_Tg\cos\varphi_Bl_T - f_{s1}\cos\varphi_Bl_s &= 0 \\
	mg\cos\varphi_B l_m + f_{s2}R - m_Tg\cos\varphi_Bl_T - m_B\frac{l_B}{\bcancel{l_s}}\cos\varphi_B\bcancel{l_s} &= 0 \\
	f_{s2}R = - mg\cos\varphi_B l_m + m_Tg\cos\varphi_Bl_T + m_B\cos\varphi_Bl_B \\
	f_{s2} = \frac{- mg\cos\varphi_B l_m + m_Tg\cos\varphi_Bl_T + m_B\cos\varphi_Bl_B}{R}
\end{align*}
Die Lagerkraft A $f_A$ setzt sich nun aus $f_{A,x}$ und $f_{A,y}$ zusammen.
\begin{align*}
	\textbf{f}_A = \left[\begin{matrix}
	0 \\
	\frac{- mg\cos\varphi_B l_m + m_Tg\cos\varphi_Bl_T + m_B\cos\varphi_Bl_B}{R}
	\end{matrix}\right]
\end{align*}
\subsection{Mechanisches Starrkörpersystem}
Zuerst werden die Massen der einzelnen Teilkörper bestimmt. Allgemein wird die Masse mit der Formel \\
\[
	m_1 = \int_{\mathcal{V}}\rho dV
\]
Da es sich aber bei diesen Beispiel um einfache Quader handelt lässt sich das Volumen des Quaders einfach bestimmen.\\ \\
- Teilkörper 1: \(m_1 = \rho l_2bd\) \\
- Teilkörper 2: \(m_2 = \rho (l_1 2d)db \) \\
- Teilkörper 3: \(m_3 = \rho l_2 bd\) \\ \\
Im nächsten Schritt werden nun die Schwerpunkte der einzelnen Teilkörper bestimmt.\\ \\
\noindent
Teilkörper 1: 
\begin{align*}
	r_{Sm1,x} &= \frac{1}{m_1} \int_{\mathcal{V}}x\rho dV = \frac{1}{l_2bd\rho}\int_{-\frac{b}{2}}^{\frac{b}{2}} \int_{l_1+d}^{l_1+2d}\int_{-l_2}^{0} x\rho dxdydz = -\frac{l_2}{2} \\
	r_{Sm1,y} &= \frac{1}{m_1} \int_{\mathcal{V}}y\rho dV = l_1 + \frac{3}{2}d \\
	r_{Sm1,z} &= \frac{1}{m_1} \int_{\mathcal{V}}z\rho dV = 0 
\end{align*}
Daraus folgt:
\[
	\textbf{r}_{Sm1}^0 = \left[\begin{matrix}
		-\frac{l_2}{2}\\
		l_1 + \frac{3}{2}d \\ 
		0
	\end{matrix}\right]
\]
Analog zu Teilkörper 1 werden nun auch die Teilkörper 2 bestimmt
\end{document}