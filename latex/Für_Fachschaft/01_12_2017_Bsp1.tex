\textbf{Beispiel 1}\\ \\
a)\\ \\
Die beiden gezeigten Federn sind parallel zu betrachten, daher kann man für die gesuchten Größen schließen
\begin{align*}
	f_1 &= c_1(s - s_{01}) \\
	f_2 &= c_2(s - s_{02}) \\
	c_g &= c_1 + c_2 \\
	s_{0_g} &= \frac{c_1s_{01} + c_2s_{02}}{c_g}
\end{align*}
b)\\ \\
Mithilfe des Impulserhaltungssatz aus der Formelsammlung kann man auf die Bewegungsgleichung
\[
	m\ddot{s} = -c_g(s - s_{0_g}) + f_e
\]
schließen.\\ \\
c)\\ \\
Im stationären Fall gilt $\ddot{s} = 0$. Daraus folgt
\begin{align*}
	0 &= -c_g(s - s_{0_g}) + F \\
	s &= \frac{F + c_gs_{0_g}}{c_g}
\end{align*}
d)\\ \\
Die zweite Ableitung nach der Zeit für den gegebenen Ansatz lautet
\[
	\ddot{x} = - x_0\omega^2_0 \cos(\omega_0 t)
\]
Durch gezieltes Einsetzen in die vereinfachte Bewegungsgleichung, kann man schließlich auf die gesuchte Eigenfrequenz schließen.
\begin{align*}
	-m x_0\omega^2_0 \cos(\omega_0 t) &= -c_g x_0\cos(\omega_o t) \\
	m \omega_0^2 &= c_g \\
	\omega_0 &= \sqrt{\frac{c_g}{m}}
\end{align*}