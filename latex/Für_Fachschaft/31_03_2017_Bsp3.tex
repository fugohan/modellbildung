\newpage
\noindent
\textbf{Beispiel 3}\\ \\
a)\\ \\
Die gesuchte Verlustwärmestromdichte wird aus der Formelsammlung entnommen lautet
\[
	\dot{q} = k(T_i - T_a)
\]
mit dem Vorfaktor (aus der Formelsammlung)
\[
	k = \frac{1}{\frac{1}{\alpha_i} + \frac{\delta_w + \delta_p}{\lambda_w} + \frac{1}{\alpha_a}}
\]
b)\\ \\
Dadurch das ein unveränderter Wärmeverlust gefordert ist folgt
\[
	\dot{q} = \dot{q}_{neu}
\]
Wegen der Dirichlet'schen Randbedingung zufolge der integrierten Wandheizung folgt
\[
	\dot{q}_{neu} = k_{neu}(T_p - T_a)
\]
und dadurch
\[
	k(T_i - T_a) = k_{neu}(T_p - T_a)
\]
mit
\[
	k_{neu} = \frac{1}{\frac{\delta_d}{\lambda_d} + \frac{\delta_w}{\lambda_w} + \frac{1}{\alpha_a}}
\]
Aus den letzten beiden Gleichungen folgt schließlich
\[
	\delta_d = \lambda_d\left(\frac{T_h - T_a}{k(T_i - T_a)} - \frac{\delta_w}{\lambda_w} - \frac{1}{\alpha_a}\right)
\]
c)\\ \\
Der Temperaturverlauf lautet
\[
	T(x) = \left\{
	\begin{array}{lll}
	T_h & \text{für} \quad 0 \leq x \leq \delta_p \\
	T_h - \dot{q}\frac{x - \delta_p}{\lambda_d} & \text{für} \quad \delta_p \leq x \leq \delta_p + \delta_d\\
	T_h - \dot{q}\left(\frac{\delta_d}{\lambda_d} + \frac{x - \delta_p - \delta_d}{\lambda_w}\right) & \text{für} \quad \delta_p + \delta_d \leq x \leq \delta_p + \delta_d + \delta_w
	\end{array}
	\right.
\]
d)\\ \\
Durch Zusammenfassen der Konturen b und w folgt aus der Summationsregel
\[
	F_{h,d} = 1 - F_{h,b+w} - F-{h,h}
\]
Da die Kontur h eine ebene Wand ist, gilt $F_{h,h} = 0$. Der andere Sichtfaktor kann mittels der Tabelle aus der Formelsammlung ermittelt werden und somit folgt
\[
	F_{h,d} = 1 - \frac{1}{2l_H}\left(l_H + \sqrt{l_b^2 + l_w^2} - \sqrt{l_b^2 + (l_w - l_h)^2}\right)	
\]
\newpage
\noindent
e)\\ \\
Da es sich bei der Kontur b ebenfalls um eine ebene Wand handelt gilt $F_{b,b} = 0$. Mit der Summationsregel und dem Reziprozitätsgesetz folgt für die restlichen Sichtfaktoren
\begin{align*}
	F_{b,h} &= \frac{l_h}{l_b}F_{b,u} \\
	F_{u,h} &= \frac{l_h}{l_1 + l_2 + l_w}F_{h,u} \\
	F_{u,b} &= \frac{l_b}{l_1 + l_2 + l_w}F_{b,u} \\
	F_{u,u} &= 1 - \frac{1}{l_1 + l_2 + l_w}(l_bF_{b,u} + l_hF_{h,u})
\end{align*}
f)\\ \\
Die gesuchte Gleichung lautet 
\[
	\varepsilon(\textbf{E} - \textbf{F}(1 - \varepsilon))^{-1}(\textbf{E} - \textbf{F})\sigma\begin{bmatrix}
	T_h^4\\ T_u^4 \\ T_b^4
	\end{bmatrix}
	+
	\begin{bmatrix}
		\dot{q}_h \\ 0 \\ 0
	\end{bmatrix}
	-
	\begin{bmatrix}
		k_2(T-h - T_a) \\ 0 \\ 0
	\end{bmatrix}
	= \textbf{\text{0}}
\]
mit
\[
	\textbf{F} = \begin{bmatrix}
		0 & F_{h,u} & F_{h,b} \\
		F_{u,h} & F_{u,u} & F_{u,b} \\
		F_{b,h} & F_{b,u} & 0  
	\end{bmatrix}
	,
	\qquad
	k_2 = \frac{1}{\frac{\delta_w}{\lambda_w} + \frac{\delta_d}{\lambda_d} + \frac{1}{\alpha_a}}
\]
Der genaue Rechenweg ist im Vorlesungsskript aus dem Jahr 2019 nachzulesen.