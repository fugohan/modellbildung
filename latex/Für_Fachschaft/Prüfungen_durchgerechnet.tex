\documentclass[a4paper,12p]{article}
\usepackage[utf8]{inputenc}
\usepackage[ngerman]{babel}
\usepackage[a4paper, left=2.5cm, right=2.5cm]{geometry}
\usepackage{amsmath}
\usepackage{graphicx}
\usepackage{mathtools}
\usepackage{cancel}
\usepackage{amsmath}
\usepackage{amssymb}

\title{\huge Modellbildung\\\large \huge Beispielsammlung}
\author{\huge 4.Semester ET-Studium}
\date{\huge Oktober 2019}

\begin{document}
	
	\maketitle
	\newpage
	\tableofcontents
	\newpage
	
	\section{Einleitung}
	In dieser Ausarbeitung befinden sich sämtliche Rechenwege der Modellbildungsprüfung beginnend ab dem Jahr 2015. Die Angaben zu den hier ausgearbeiteten Prüfungen befinden sich auf der Homepage des ACIN. Sämtlichen verwendeten Formeln befinden sich in der Formelsammlung, welche ebenfalls auf der Homepage des ACIN zu finden ist. Ich hoffe dieses Dokument hilft euch weiter. 
	
	\section{Prüfungen}
	
	\subsection{17.05.2019}
	\noindent
\textbf{Beispiel 2} \\ \\
a) \\ \\

	\textbf{Beispiel 3} \\ \\
	a) \\ \\
	Der Vektor vom Ursprung zum Schwerpunkt des Rades kann direkt aus Angabe abgelesen werden und lautet deshalb:
	\begin{align*}
		\textbf{r}_r = \left[ \begin{matrix}
			p\cos\alpha - r\sin\alpha \\
			p\sin\alpha + r\cos\alpha
		\end{matrix}\right]
	\end{align*}
	Der translatorische Geschwindigkeitsvektor erhält man durch die Ableitung vom Ortsvektor nach den Freiheitsgraden.\\ \\
	Translatorischer Geschwindigkeitsvektor:
	\[
			\textbf{v}_r = \dot{\textbf{r}_r} = \left[\begin{matrix}
			\dot{p}\cos\alpha \\
			\dot{p}\sin\alpha
		\end{matrix}\right]
	\]
	Die rotatorische Geschwindigkeit lautet:
	\[ \omega_r = \frac{\dot{p}}{r}\]
	b)\\ \\
	Der Vektor zum Schwerpunkt des Stabes kann ebenfalls aus der Angabe abgelesen werden und lautet deshalb:
	\begin{align*}
		\textbf{r}_S = \left[\begin{matrix}
			p\cos\alpha - r\sin\alpha + l_s\sin\varphi \\
			p\sin\alpha + r\cos\alpha + l_s\cos\varphi
		\end{matrix}\right]
	\end{align*}
	Analog zu a) lautet der translatorische Geschwindigkeitsvektor:
	\[
		\textbf{v}_S = \dot{\textbf{r}_S} =\left[ \begin{matrix}
			\dot{p}\cos\alpha + l_s\cos\varphi\dot{\varphi} \\
			\dot{p}\sin\alpha - l_s\sin\varphi\dot{\varphi}
		\end{matrix}\right]
	\]
	\newpage
	\noindent
	c) \\ \\
	Als erstes wird die translatorische kinetische Energie des System wie folgt ermittelt:\\ \\
	Vereinfachungen:
	\begin{align*}
		\dot{\textbf{r}_r}^T\dot{\textbf{r}_r} &= \left[ \begin{matrix}
			\dot{p}\cos\alpha & \dot{p}\sin\alpha
		\end{matrix}\right]
		\left[\begin{matrix}
			\dot{p}\cos\alpha \\
			\dot{p}\sin\alpha
		\end{matrix}\right] \\
		&= \dot{p}^2\cos^2\alpha + \dot{p}^2\sin^2\alpha \\
		&= \dot{p}^2\underbrace{\left( \cos^2\alpha + \sin^2\alpha\right)}_{=1} \\
		&= \dot{p}^2 
	\end{align*}
	\begin{align*}
		\dot{\textbf{r}_S}^T\dot{\textbf{r}_S} &= \left[\begin{matrix}
		\dot{p}\cos\alpha + l_s\cos\varphi\dot{\varphi} & \dot{p}\sin\alpha - l_s\sin\varphi\dot{\varphi}
		\end{matrix}\right] \left[\begin{matrix}
		\dot{p}\cos\alpha + l_s\cos\varphi\dot{\varphi} \\
		\dot{p}\sin\alpha - l_s\sin\varphi\dot{\varphi}
		\end{matrix}\right] \\
		&=\left(\dot{p}\cos\alpha + l_s\cos\varphi\dot{\varphi}\right)^2 + \left(\dot{p}\sin\alpha + l_s\sin\varphi\dot{\varphi}\right)^2 \\
		&=\dot{p}^2\cos^2\alpha + 2l_s\dot{p}\dot{\varphi}\cos\alpha\cos\varphi + l_s^2\dot{\varphi}^2\cos^2\varphi + \dot{p}^2\sin^2\alpha - 2l_s\dot{p}\dot{\varphi}\sin\alpha\sin\varphi + l_s^2\sin^2\varphi\dot{\varphi}^2 \\
		&=\dot{p}^2\underbrace{\left(\sin^2\alpha + \cos^2\alpha\right)}_{=1} + 2l_s\dot{p}\dot{\varphi}\underbrace{\left(\cos\varphi\cos\alpha - \sin\varphi\sin\alpha\right)}_{\cos\left(\varphi + \alpha\right)} + l_s^2\dot{\varphi}^2\underbrace{\left(\sin^2\varphi + \cos^2\varphi\right)}_{=1} \\
		&= \dot{p}^2 + 2l_s\dot{p}\dot{\varphi}\cos\left(\varphi + \alpha\right) + l_s^2\dot{\varphi}^2
	\end{align*}
	Nun kann man schließlich die translatorische kinetischen Energie des Rades und des Stabes bestimmen.
	\begin{align*}
		T_{trans,r} &=  \frac{1}{2}m_r\dot{p}^2\\
		T_{trans,s} &=  \frac{1}{2}m_s\left(\dot{p}^2 + l_s^2\dot{\varphi}^2 + 2l_s\dot{p}\dot {\varphi}\cos\left(\varphi + \alpha\right)\right)
	\end{align*}
	Die kinetische Energie besitzt jedoch auch einen rotatorischen Anteil. Dieser lautet für die beiden Teilsysteme:
	\begin{align*}
		T_{rot,r} &= \frac{1}{2} \Theta_r \frac{\dot{p}^2}{r^2} \\
		T_{rot,s} &= \frac{1}{2} \Theta_s \dot{\varphi}^2
	\end{align*}
	Da wir nun sämtliche Teilenergien ermittelt haben, beträgt die gesamte kinetische Energie des vorliegenden Systems:
	\begin{align*}
		T &= T_{trans,r} + T_{rot,r} + T_{trans,s} + T_{rot,s}\\
		  &= \frac{1}{2}m_r\dot{p}^2 + \frac{1}{2} \Theta_r \frac{\dot{p}^2}{r^2} + \frac{1}{2}m_s\left(\dot{p}^2 + l_s^2\dot{\varphi}^2 + 2l_s\dot{p}\dot{\varphi}\cos\left(\varphi + \alpha\right)\right) + \frac{1}{2} \Theta_s \dot{\varphi}^2
	\end{align*}
	\newpage
	\noindent
	Als nächstes wird nun die gesamte potentielle Energie des gegebenen System ermittelt. Zuerst berechnet man wieder die Energien der Teilsysteme und addiert dieser zum Schluss wieder zusammen.
	\begin{align*}
		V_r &= m_rg\left(p\sin\alpha + r\cos\alpha\right) \\
		V_s &= m_sg\left(p\sin\alpha + r\cos\alpha + l_s\cos\varphi\right) \\
		V = V_r + V_s &= m_rg\left(p\sin\alpha + r\cos\alpha\right) + m_sg\left(p\sin\alpha + r\cos\alpha + l_s\cos\varphi\right)
	\end{align*}
	d)\\ \\
	Um den Vektor der generalisierten Kräfte zu bestimmen benötigt man zuerst den Richtungsvektor zu den Angriffspunkten der extern wirkenden Kräfte, hier $f_{ext}$. \\ \\
	Angriffspunkt der Kraft:
	\[
		\textbf{r}_f = \left[\begin{matrix}
			p\cos\alpha - r\sin\alpha + 2l_s\sin\varphi \\
			p\sin\alpha + r\cos\alpha + 2l_s\cos\varphi
		\end{matrix}\right]
	\]
	Weiters benötigt man auch den Vektor der externen Kräfte. \\ \\
	Kraftvektor:
	\[
		\textbf{f}_{ext}^T = f_{ext}\left[\begin{matrix}
			\cos\beta & \sin\beta
		\end{matrix}\right]
	\]
	Nun werden die partiellen Ableitung nach $\textbf{q}$ vom Angriffspunkt der Kraft gebildet:
	\begin{align*}
		\frac{\partial\textbf{r}_f}{\partial\varphi} = \left[\begin{matrix}
			2l_s\cos\varphi \\
			-2l_s\sin\varphi
		\end{matrix}\right] \qquad
		\frac{\partial\textbf{r}_f}{\partial p} = \left[\begin{matrix}
			\cos\alpha \\
			\sin\alpha
		\end{matrix}\right]
	\end{align*}
	Der Vektor der generalisierten Kräfte wird nun wie folgt ermittelt:
	\begin{align*}
		\textbf{f}_q = \textbf{f}_{ext}^T \frac{\partial \textbf{r}_f}{\partial \textbf{q}}
	\end{align*}
	generalisierte Kräfte:
	\begin{align*}
		f_{q,\varphi} &= f_{ext}\left[\begin{matrix}
		\cos\beta & \sin\beta
		\end{matrix}\right] \left[\begin{matrix}
		2l_s\cos\varphi \\
		-2l_s\sin\varphi
		\end{matrix}\right] \\
		&= f_{ext} \left(2l_s\cos\beta\cos\varphi - 2l_s\sin\beta\sin\varphi\right) \\
		&= f_{fext}2l_s \underbrace{\left(\cos\beta\cos\varphi - 2l_s\sin\beta\sin\varphi\right)}_{\cos\left(\beta + \varphi\right)} \\
		&= f_{ext}2l_s\cos\left(\beta + \varphi\right)
	\end{align*}
	\begin{align*}
		f_{q,p} &= f_{ext} \left[\begin{matrix}
		\cos\beta & \sin\beta
		\end{matrix}\right] \left[\begin{matrix}
			\cos\alpha \\
			\sin\alpha
		\end{matrix}\right] \\
		&= f_{ext} \underbrace{\left(\cos\beta\cos\alpha + \sin\beta\sin\alpha\right)}_{\cos\left(\beta - \alpha\right)} \\
		&= f_{ext} \cos\left(\beta - \alpha\right)
	\end{align*}
	gesamter Vektor:
	\[
		\textbf{f}_q = f_{ext} \left[\begin{matrix}
			2l_s\cos\left(\beta + \varphi\right) \\
			\cos\left(\beta - \alpha\right)
		\end{matrix}\right]
	\]
	\newpage
	\noindent
	e) \\ \\
	Zum Schluss sollen noch die Bewegungsgleichungen mithilfe des Euler-Lagrange-Formalismus bestimmt werden.
	\begin{align*}
		L &= T - V \\
		 &= \frac{1}{2}m_r\dot{p}^2 + \frac{1}{2} \Theta_r \frac{\dot{p}^2}{r^2} + \frac{1}{2}m_s\left(\dot{p}^2 + l_s^2\dot{\varphi}^2 + 2l_s\dot{p}\dot{\varphi}\cos\left(\varphi + \alpha\right)\right) + \frac{1}{2} \Theta_s \dot{\varphi}^2 \\
		 & - m_rg\left(p\sin\alpha + r\cos\alpha\right) - m_sg\left(p\sin\alpha + r\cos\alpha + l_s\cos\varphi\right)
	\end{align*}
	Bewegungsgleichungen:
	\begin{align*}
		\frac{d}{dt}\left(\frac{\partial L}{\partial \dot{\varphi}}\right) - \left(\frac{\partial L}{\partial \varphi}\right) &= 2l_sf_{ext}\cos\left(\beta + \varphi\right) \\
		\frac{d}{dt}\left(\frac{\partial L}{\partial \dot{p}}\right) - \left(\frac{\partial L}{\partial p}\right) &= f_{ext}\cos\left(\beta - \alpha\right)
	\end{align*}
	Zwischenschritte:
	\begin{align*}
		\frac{\partial L}{\partial \dot{\varphi}} &= m_s\left(l_s^2\dot{\varphi} + l_s\dot{p}\cos\left(\varphi + \alpha\right)\right) + \Theta_s \dot{\varphi} \\
		\frac{\partial L}{\partial \dot{p}} &= m_r\dot{p} + \frac{\Theta_r}{r^2}\dot{p} + m_s\left(\dot{p} + 2l_s\dot{\varphi}\cos\left(\varphi + \alpha\right)\right)
	\end{align*}
	auftretende Ableitungen:
	\begin{align*}
		\frac{\partial L}{\partial \varphi} &= -m_sl_s\sin\left(\varphi + \alpha\right)\dot{p}\dot{\varphi} + m_sgl_s\sin\varphi \\
		\frac{\partial L}{\partial p} &= -g\left(m_s + m_r\right)\sin\alpha \\
		\frac{d}{dt}\left(\frac{\partial L}{\partial \dot{\varphi}}\right) &= m_sl_s\cos\left(\varphi + \alpha\right)\ddot{p} + \left(m_sl_s^2 + \Theta_s\right)\ddot{\varphi} - m_sl_s\dot{p}\sin\left(\varphi + \alpha\right)\dot{\varphi} \\
		\frac{d}{dt}\left(\frac{\partial L}{\partial \dot{p}}\right) &= \left(m_r + \frac{\Theta_r}{r^2} + m_s\right)\ddot{p} + m_sl_s\left(\ddot{\varphi}\cos\left(\varphi + \alpha\right) - \dot{\varphi}\sin\left(\varphi + \alpha\right)\dot{\varphi} \right)
	\end{align*}
	
	\newpage
	\subsection{12.07.2019}
	\noindent
\textbf{Beispiel 1} \\ \\
a) \\ \\
Der Richtungsvektor vom Urspung zum Schwerpunkt der Masse $m$ lautet:
\[
	\textbf{p}_m = \left[ \begin {array}{c} r\cos \left( \alpha \right) +b
	\\ r\sin \left( \alpha \right) -h\end {array}
	\right]	
\]
Mithilfe diesen Vektor kann ebenfalls auch der Geschwindigkeitsvektor bestimmt werden, indem man den Richtungsvektor nach der Zeit ableitet.
\[
	\dot{\textbf{p}}_m =\left[ \begin {array}{c} -r \dot{\alpha}
	  \sin \left( \alpha 
	\right) + \dot{b} \\ 
	r \dot{\alpha} \cos \left( \alpha  \right) -
		\dot{h} \end {array} \right] 
\]
b) \\ \\
Um die kinetischen Energien zu berechnen, müssen zuerst einige Nebenrechnungen durchgeführt werden.\\
Nebenrechnungen:
\begin{align}
	\dot{\textbf{p}}_m^T \dot{\textbf{p}}_m &= \left[ \begin{matrix}
		-r\dot{\alpha}\sin \left( \alpha \right) + \dot{b} & r\dot{\alpha}\cos \left( \alpha \right) -\dot{h}
	\end{matrix}\right] \left[ \begin {array}{c} -r \dot{\alpha}
	\sin \left( \alpha 
	\right) + \dot{b} \\ 
	r \dot{\alpha} \cos \left( \alpha  \right) -
	\dot{h} \end {array} \right] \\
	&= r^2\dot{\alpha}^2\underbrace{\left(\cos \ \alpha  + \sin  \alpha\right)  }_{=1} -2r\dot{\alpha}\dot{b}\sin\alpha - 2r\dot{\alpha}\dot{h}\cos\alpha + \dot{b}^2 + \dot{h}^2 \\
	&= r^2\dot{\alpha}^2 -2r\dot{\alpha}\dot{b}\sin\alpha - 2r\dot{\alpha}\dot{h}\cos\alpha + \dot{b}^2 + \dot{h}^2 \\
	\varphi &= \arctan\left(\frac{h}{b}\right) \\
	\dot{\varphi} &= \frac{\dot{h}b - h\dot{b}}{b^2 + h^2}
\end{align}
kinetische Energie des Systems:
\begin{align*}
	T_{tm} &= \frac{1}{2}m\dot{\textbf{p}}_m^T \dot{\textbf{p}}_m = \frac{1}{2} m \left( r^2\dot{\alpha}^2 -2r\dot{\alpha}\dot{b}\sin\left(\alpha\right) - 2r\dot{\alpha}\dot{h}\cos\left(\alpha\right) + \dot{b}^2 + \dot{h}^2\right) \\
	T_{rm} &= \frac{1}{2} \Theta_m\dot{\varphi}^2 = \frac{1}{2} \Theta_m \left( \frac{\dot{h}b - h\dot{b}}{b^2 + h^2}\right)^2 \\
	T_{rr} &= \frac{1}{2}\Theta_r\dot{\alpha}^2 \\
	T &= T_{tm} + T_{rm} + T_{rr} \\
	  &= \frac{1}{2}m\dot{\textbf{p}}_m^T \dot{\textbf{p}}_m = \frac{1}{2} m \left( r^2\dot{\alpha}^2 -2r\dot{\alpha}\dot{b}\sin\left(\alpha\right) - 2r\dot{\alpha}\dot{h}\cos\left(\alpha\right) + \dot{b}^2 + \dot{h}^2\right) 
	  + \frac{1}{2} \Theta_m \left( \frac{\dot{h}b - h\dot{b}}{b^2 + h^2}\right)^2 + \frac{1}{2}\Theta_r\dot{\alpha}^2
\end{align*}
c) \\ \\
potentielle Energie des Systems:
\begin{align*}
	V_m &= mg\left(r\sin\alpha - h\right) \\
	V_c &= \frac{1}{2} c \left(\sqrt{b^2 + h^2} s_0\right)^2 \\
	V &= V_m + V_c = mg\left(r\sin\alpha - h\right) + \frac{1}{2} c \left(\sqrt{b^2 + h^2} s_0\right)^2
\end{align*}
d) \\ \\
viskose Reibkraft
\[
	\textbf{f}_V = \mu_V \frac{\dot{h}b - h\dot{b}}{b^2 + h^2} \left[ \begin{matrix}
	b \\ -h
	\end{matrix}\right]
\]
partielle Ableitungen:
\[
	\frac{\partial \dot{\textbf{p}}_m}{\partial \alpha}  = \left[ \begin{matrix}
		-r\sin\alpha \\ r\cos\alpha
	\end{matrix}\right] \qquad \frac{\partial \dot{\textbf{p}}_m}{\partial b} = \left[ \begin{matrix}
	 1 \\ 0
	\end{matrix}\right] \qquad \frac{\partial \dot{\textbf{p}}_m}{\partial h} = \left[ \begin{matrix}
	0 \\ -1
	\end{matrix}\right]
\]
\newpage
\noindent
generalisierte Kräfte:\\ \\
Multipliziert man die viskose Reibungskraft mit allen partiellen Ableitungen erhält man
\begin{align*}
	\textbf{f}_{q,v} &= \mu_V \frac{\dot{h}b - h\dot{b}}{b^2 + h^2} \left[ \begin{matrix}
		r\sin\alpha b + r\cos\alpha h \\
		-b \\
		-h
	\end{matrix}\right]
\end{align*}
Die andere externe Kraft ist
\[
	\textbf{f}_x = \left[ \begin{matrix}
		f_x \\
		0
	\end{matrix}\right]
\]
Multipliziert mit den partiellen Ableitungen erhält man
\[
	\textbf{f}_{q,x} = \left[ \begin{matrix}
		-f_x r \sin\alpha \\
		f_x \\
		0
	\end{matrix}\right]
\]
Der gesamte Vektor der generalisierten Kräfte beträgt
\[
	\textbf{f}_q = \mu_V \frac{\dot{h}b - h\dot{b}}{b^2 + h^2} \left[ \begin{matrix}
	r\sin\alpha b + r\cos\alpha h \\
	-b \\
	-h
	\end{matrix}\right] 
	+ 
	\left[ \begin{matrix}
	-f_x r \sin\alpha \\
	f_x \\
	0
	\end{matrix}\right]
\]
e) \\ \\
Verwendet man aus der Formelsammlung im Punkt generalisierte Kräfte die 2.te Formel und passt man diese an den stationären Fall an erhält man
\[
	\frac{\partial V}{\partial \textbf{q}} = \left[ \begin{matrix}
		mgr\cos\alpha\\
		\frac{c\left( \sqrt{b^2 + h^2} - s_0\right)}{\sqrt{b^2 + h^2}} b \\
		\frac{c\left( \sqrt{b^2 + h^2} - s_0\right)}{\sqrt{b^2 + h^2}} h - mg
	\end{matrix}\right]
	=
	\left[ \begin{matrix}
	-f_x r \sin\alpha \\
	f_x \\
	0
	\end{matrix}\right]
\]
Nun wird $s_0$ mit 0 angenommen und anschließend werden die generalisierten Koordinaten bestimmt. \\ \\
1.Koordinate:
\begin{align*}
	mgr\cos\alpha &= -f_xr\sin\alpha \\
	-\frac{\cos\alpha}{\sin\alpha} &= \frac{mg}{f_x}  \\
	-\tan\alpha &= \frac{mg}{f_x} \\
	\alpha &= -\arctan\left( \frac{mg}{f_x}\right)
\end{align*}
2.Koordinate:
\begin{align*}
	\frac{c\left(\cancel{\sqrt{b^2 + h^2}}\right)}{\cancel{\sqrt{b^2 + h^2}}} b &= f_x \\
	cb &= f_x \\
	b &= \frac{f_x}{c}
\end{align*}
3.Koordinate
\begin{align*}
	\frac{c\left(\cancel{\sqrt{b^2 + h^2}}\right)}{\cancel{\sqrt{b^2 + h^2}}} h -mg &= 0 \\
	ch - mg &= 0 \\
	h &= \frac{mg}{c}
\end{align*}
f) \\ \\
Die resultierende Reibkraft wird mithilfe der Formel 2.96 aus dem Vorlesungsskript berechnet und lautet hier:
\[
	\textbf{f}_F = -c_W A \frac{\rho}{2} |\dot{\textbf{p}}_m|\dot{\textbf{p}}_m
\]
\begin{align*}
	|\dot{\textbf{p}}_m| &= \sqrt{\left( -r \dot{\alpha}
		\sin \left( \alpha 
		\right) + \dot{b}\right)^2 
	+
	\left( 	r \dot{\alpha} \cos \left( \alpha  \right) -
	\dot{h}\right)^2} \\
	&= \sqrt{r^2\dot{\alpha}^2\sin^2\alpha -2r\dot{\alpha}\dot{b}\sin\alpha + \dot{b}^2 + r^2\dot{\alpha}^2\sin^2\alpha - 2r\dot{\alpha}\dot{h}\cos\alpha + \dot{h}^2} \\
	&= \sqrt{r^2\dot{\alpha}^2\underbrace{\left(\sin^2\alpha + \cos^2\alpha\right)}_{=1} - 2r\dot{\alpha}\dot{b}\sin\alpha - 2r\dot{\alpha}\dot{h}\sin\alpha + \dot{b}^2 + \dot{h}^2} \\
	&= \sqrt{r^2\dot{\alpha}^2 - 2r\dot{\alpha}\dot{b}\sin\alpha - 2r\dot{\alpha}\dot{h}\sin\alpha + \dot{b}^2 + \dot{h}^2}
\end{align*}
Multipliziert man nun $\textbf{f}_F$ mit den partiellen Ableitungen aus d) und vereinfacht man so weit wi möglich erhält man
\begin{align*}
	\textbf{f}_{q,F} &= -c_W A \frac{\rho}{2}|\dot{\textbf{p}}_m|
	\left[\begin{matrix}
		r\dot{\alpha}\underbrace{\left( \sin^2\alpha + \cos^2\alpha \right)}_{=1} - r\dot{b}\sin\alpha - r\dot{h}\cos\alpha \\
		-r\dot{\alpha}\sin \left( \alpha \right) + \dot{b} \\
		-r\dot{\alpha}\cos \left( \alpha \right) +\dot{h}
	\end{matrix}
	\right] \\
	&= -c_W A \frac{\rho}{2}|\dot{\textbf{p}}_m|
	\left[\begin{matrix}
	r\dot{\alpha} - r\dot{b}\sin\alpha - r\dot{h}\cos\alpha \\
	-r\dot{\alpha}\sin \left( \alpha \right) + \dot{b} \\
	-r\dot{\alpha}\cos \left( \alpha \right) +\dot{h}
	\end{matrix}
	\right]
\end{align*}
	\newpage
\noindent
\textbf{Beispiel 2} \\ \\
a) \\ \\
Da es sich hier um ein Problem mit Zylinderkoordinaten hält, verwendet man nun einfach die Wärmeleitgleichung für Zylinderkoordinaten aus Formelsammlung und adaptiert diese entsprechend der Angabe. \\
\newline
Wärmeleitgleichung:
\[
	0 = \lambda \left( \frac{1}{r} \frac{d}{dr}\left( \frac{d}{dr} T_i\left( r\right)\right)\right)
\]
Die 0 auf der linken Seite beruht darauf, das es sich hier um ein stationäres Problem handelt und die Terme für $\varphi$ und $z$ fallen ebenfalls weg. \\
\newline
Lösung der DGL:
\begin{align*}
	\lambda_i \left( \frac{1}{r} \frac{d}{dr}\left( r \frac{d}{dr} T_i\left( r\right)\right)\right) &= 0
	\\
	\frac{d}{dr}\left( r \frac{d}{dr} T_i\left( r\right)\right) &= 0
	\\
	\int \frac{d}{dr}\left( r \frac{d}{dr} T_i\left( r\right)\right) dr &= \int 0 dr
	\\
	r \frac{d}{dr} T_i\left( r\right) &= C_3
	\\
	\frac{d}{dr} T_i\left( r\right) &= \frac{C_3}{r}
	\\
	\int \frac{d}{dr} T_i\left( r\right) dr &= \int \frac{C_3}{r} dr
	\\
	T_i\left( r\right) &= C_3 ln\left(r \right) + C_4
\end{align*}
Randbedingungen:
\begin{align*}
	T_i\left(2R\right) &= C_3 ln\left( 2R\right) + C_4 = T_2 \\
	T_i\left(2R\right) &= C_3 ln\left( 3R\right) + C_4 = T_3
\end{align*}
Integrationskonstanten: \\
\newline
Ausgehend von dem Gleichungssystem der Randbedingung kann man die Integrationskonstanten sehr leicht bestimmen. Als ersten formt man die 2.te Gleichung auf $C_4$ um.
\begin{align*}
	T_3 &= C_3 ln\left( 3R\right) + C_4\\
	C_4 &= T_3 - C_3 ln\left( 3R\right)
\end{align*}
Dann wird in Gleichung 1 eingesetzt
\begin{align*}
	T_2 &= C_3 ln\left( 2R\right) + T_3 - C_3 ln\left( 3R\right) \\
	T_2 - T_3 &= C_3 \underbrace{\left( ln\left( 2R\right) - ln\left( 3R\right) \right)}_{ln\left( \frac{2}{3}\right)} \\
	C_3 &= \frac{T_2 - T_3}{ln\left( \frac{2}{3}\right)}
\end{align*}
Nun setzt man $C_3$ in $C_4$ ein
\begin{align*}
	C_4 &= T_3 -  \frac{T_2 - T_3}{ln\left( \frac{2}{3}\right)} ln\left( 3R\right) \\
	C_4 &= \frac{T_3 \left( ln\left( 2R\right) - ln\left( 3R\right) \right)}{ln\left( \frac{2}{3}\right)} - \frac{T_2 - T_3}{ln\left( \frac{2}{3}\right)} ln\left( 3R\right) \\
	C_4 &= \frac{T_3 ln\left( 2R \right) - T_2 ln\left( 3R\right)}{ln\left( \frac{2}{3}\right)}
\end{align*}
b) \\ \\	
Die Leistungsdichte $g$ wird wie folgt berechnet
\[
	g = \rho_e |\textbf{J}|^2 = \rho_e \left( \frac{I}{A}\right)^2
\]
mit
\[
	A = \left(4R^2 - R^2\right)\pi
\]
Dies ist die Fläche eine Hohlleiters. Setzt man nun diese in $g$ ein, erhält man
\[
	g = \rho_e \left( \frac{I}{\left(4R^2 - R^2\right)\pi}\right)^2
\]
c) \\ \\
DGL:
\begin{align*}
	\lambda_l \left( \frac{1}{r}\frac{d}{dr}\left( r\frac{d}{dr} T_l\left(r\right)\right)\right) + g = 0 \\
	\frac{d}{dr}\left( r\frac{d}{dr} T_l\left(r\right)\right) = -\frac{gr}{\lambda_l} \\
	\int \frac{d}{dr}\left( r\frac{d}{dr} T_l\left(r\right)\right) dr = \int -\frac{gr}{\lambda_l} dr \\
	 r\frac{d}{dr} T_l\left(r\right) = -\frac{gr^2}{2\lambda_l} + C_1 \\
	 \frac{d}{dr} T_l\left(r\right) = -\frac{gr}{2\lambda_l} + \frac{C_1}{r} \\
	 \int \frac{d}{dr} T_l\left(r\right) dr = \int -\frac{gr}{2\lambda_l} + \frac{C_1}{r} dr \\
	 T_l\left(r\right) = -\frac{gr^2}{4\lambda_l} + C_1\ln\left( r \right) + C_2
\end{align*}
Randbedingungen
\begin{align*}
	T_l\left(R\right) = -\frac{gR^2}{4\lambda_l} + C_1\ln\left(R\right) + C_2 = T_1 \\
	T_l\left(2R\right) = -\frac{g4R^2}{4\lambda_l} + C_1 \ln\left(2R\right) + C_2 = T_2
\end{align*}
\newpage
\noindent
Integrationskonstanten: \\ \\
Die beiden Konstanten werden wie in a) bestimmt.
\begin{align*}
	-\frac{gR^2}{4\lambda_l} + C_1\ln\left(R\right) + C_2 = T_1 \\
	C_2 = T_1 + \frac{gR^2}{4\lambda_l} - C_1\ln\left(R\right)
\end{align*}
Diese Gleichung wird nun in die andere eingesetzt und man erhält
\begin{align*}
	-\frac{g4R^2}{4\lambda_l} + C_1 \ln\left(2R\right) + T_1 + \frac{gR^2}{4\lambda_l} - C_1\ln\left(R\right) = T_2 \\
	\frac{g}{4\lambda_l}\left(R^2 - 4R^2\right) 
	+ T_1 - T_2 = C_1 \underbrace{\left( ln\ln\left(1\right) - \ln\left(2\right)\right)}_{\ln\left( \frac{1}{2}\right)} \\
	C_1 = \frac{\frac{g}{4\lambda_l}\left(R^2 - 4R^2\right) + T_1 - T_2}{\ln\left( \frac{1}{2}\right)}
\end{align*}
Rückeingesetzt erhält man
\begin{align*}
	C_2 &= T_1 + \frac{gR^2}{4\lambda_l} - \frac{\frac{g}{4\lambda_l}\left(R^2 - 4R^2\right) + T_1 - T_2}{\ln\left( \frac{1}{2}\right)}\ln\left(R\right) \\
	&= \frac{T_1\left(\ln\left(R\right) - \ln\left(2R\right)\right)}{ln\left(\frac{1}{2}\right)} + 
	\frac{\frac{gR^2}{4\lambda_l}\left(\ln\left(R\right) - \ln\left(2R\right)\right)}{\ln\left( \frac{1}{2}\right)} \\
	&- \frac{\frac{g}{4\lambda_l}\left(R^2\ln\left(R\right) - 4R^2\ln\left(R\right)\right) + T_1\ln\left(R\right) - T_2\ln\left(R\right)}{\ln\left( \frac{1}{2}\right)} \\
	&= \frac{\frac{g}{4\lambda_l}\left(4R^2\ln\left(R\right) - R^2\ln\left(2R\right)\right) + T_2\ln\left(R\right) - T_1\ln\left(2R\right)}{\ln\left(\frac{1}{2}\right)}
\end{align*}
d) \\ \\
Zur eindeutigen Bestimmung der Temperaturen \(T_1 , T_2 , T_3\) muss einfach nur die Gleichungen für die Wärmeübergänge aufstellen. Diese lauten für \\ \\
\(T_2:\)
\[
	\lambda_i\frac{d}{dr}T_i\left(r\right)|_{r = 2R} = \lambda_l \frac{d}{dr}T_l\left(r\right)|_{r = 2R}
\]
\(T_3:\)
\[
	\alpha_0\left(T_\infty - T_3\right) = \lambda_i\frac{d}{dr}T_i\left(r\right)|_{r = 3R}
\]
\(T_1:\)
\[
	\alpha_f\left(T_f - T_1\right) = -\lambda_l\frac{d}{dr}T_l\left(r\right)|_{r = R}
\]
\newpage
\noindent
e) \\ \\
1) Temperatur der Kühlflüssigkeit senken \\
2) Oberfläche der Isolierung vergrößern (z.B. durch Kühlrippen) \\
3) Wärmeleitfähigkeit der Isolierung vergrößern \\
\\
\textit{Hinweis}:\\
\textit{Die Lösung in diesen Punkt erhält man durch logisch denken.}

\end{document}