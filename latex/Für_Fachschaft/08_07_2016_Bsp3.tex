\textbf{Beispiel 3}\\ \\
a)\\ \\
Aufgrund der ebenen Oberflächen strahlen die drei Schienen nicht auf sich selbst ab daher kann man daraus für die Sichtfaktoren
\[
	F_{11} = F_{22} = F_{33} = 0
\]
schließen. Aufgrund der Symmetrie kann man außerdem 
\[
	F_{12} = F_{13} = F_{21} = F_{31} = \overline{F}
\]
sagen.
Aufgrund der gegebenen Anordnung folgt
\[
	F_{23} = F_{32} = 0
\]
Die übrigen Faktoren werden mithilfe der Summationsregel bestimmt. Diese lauten daher
\begin{align*}
	F_{1\infty} &= 1 - 2\overline{F} \\
	F_{2\infty} &= 1 - \overline{F} \\
	F_{3\infty} &= 1 - \overline{F}
\end{align*}
Damit lautet die Sichtfaktormatrix
\begin{align*}
	\textbf{F} = \begin{bmatrix}
		0 & \overline{F} & \overline{F} & 1 - 2\overline{F} \\
		\overline{F} & 0 & 0 & 1 - \overline{F} \\
		\overline{F} & 0 & 0 & 1 - \overline{F} \\
		0 & 0 & 0 & 1
	\end{bmatrix}
\end{align*}
b)\\ \\
Geometrische Zusammenhänge
\begin{align*}
	\sin(\varTheta_{2,0}) &= \frac{L - x}{\sqrt{a^2 + (L - x)^2}} \\
	\sin(\varTheta_{2,1}) &= \frac{2L - x}{\sqrt{a^2 + (2L - x)^2}}
\end{align*}
Nun kann man den Sichtfaktor exakt bestimmen durch
\begin{align*}
	F_{21} &= \frac{1}{2L}\int_{0}^{L} \sin(\varTheta_{2,1}) - \sin(\varTheta_{2,0}) \text{d}x \\
	       &= \frac{1}{2L}\int_{0}^{L} \frac{2L - x}{\sqrt{a^2 + (2L - x)^2}} - \frac{L - x}{\sqrt{a^2 + (L - x)^2}} \\
	       &= \frac{1}{2L}\left(\int_{0}^{L} \frac{2L - x}{\sqrt{a^2 + (2L - x)^2}} \text{d}x - \int_{0}^{L}\frac{L - x}{\sqrt{a^2 + (L - x)^2}}\text{d}x\right)
\end{align*}
Mit der Substitution für das erste Integral 
\begin{align*}
	\sigma &= a^2 + (2L - x)^2 \\
	\text{d}\sigma &= - 2(2L - x) \text{d}x \\
	\text{d}x &= -\frac{\text{d}\sigma}{2(2L - x) }
\end{align*}
und für das zweite Integral
\begin{align*}
	\sigma &= a^2 + (L - x)^2 \\
	\text{d}\sigma &= -2(L - x) \\
	\text{d}x &= -\frac{\text{d}\sigma}{2(L - x)}
\end{align*}
Nun folgt
\begin{align*}
	F_{21} &= \frac{1}{4L}\left( \int_{a^2 + L^2}^{a^2 + 4L^2} \sigma^{-\frac{1}{2}}\text{d}\sigma -  \int_{a^2}^{a^2 + L^2}\sigma^{-\frac{1}{2}}\text{d}\sigma\right) \\
	&= \frac{1}{2L} \left(\sqrt{a^2 + 4L^2} +  a - 2\sqrt{a^2 + L^2}\right)
\end{align*}
c)\\ \\
Aufgrund der Symmetrie folgt 
\[
	A_4 = A_2 + A_3
\]
woraus man 
\[
	\frac{1}{2}A_4 = A_2 = A_3
\]
schließen kann.
Da die Schienen 2 und 3 zu einem Körper zusammengefasst werden, folgt für den Sichtfaktor des zusammengefassten Körper zur Schiene 1
\[
	F_{41} = \overline{F}
\]
Mit dem Reziprozitätsgesetz folgt
\[
	F_{14} = 2\overline{F}
\]
Damit lautet die Sichtfaktormatrix
\[
	\textbf{F} = \begin{bmatrix}
		0 & 2\overline{F} \\
		\overline{F} & 0
	\end{bmatrix}
\]
d)\\ \\
Die Wärmeleitgleichung hier lautet
\[
	\frac{\partial^2}{\partial z^2}T_S(z) = 0
\]
Die Randbedingungen für dieses Wärmeleitproblem lauten
\begin{align*}
	-\lambda\frac{\partial}{\partial z}T_S(0) &= \dot{q}_4 \\
	 T_S(b) &= T_\infty
\end{align*}
Lösungsweg der Wärmeleitgleichung:
\begin{align*}
	\frac{d^2}{d z^2}T_S(z) &= 0 \\
	dT_S(z) &= C_1 dz \\
	T_S(z) &= C_1 z + C_2
\end{align*}
Setzt man nun beide Randbedingungen ein folgt für die Konstanten
\begin{align*}
	C_1 &= -\frac{\dot{q}_4}{\lambda} \\
	C_2 &= T_\infty + \frac{\dot{q}_4}{\lambda}b
\end{align*}
Nun lautet die vollständige Lösung der Wärmeleitgleichung
\[
	T_S(z) = T_\infty + \frac{\dot{q}_4}{\lambda}(b - z)
\]
e)\\ \\
Im stationären Fall müssen sich die Leistungen zu Null bilanzieren. Aus der längenbezogenen Leistungsbilanz für die Schiene 1
\[
	\int_{0}^{L}\dot{q}_1\text{d}x = \int_{0}^{L}\int_{0}^{c}\rho_1\left(\frac{I}{cL}\right)^2\text{d}x\text{d}z
\]
folgt schließlich der Wärmestrom
\[
	\dot{q}_1 = \frac{\rho_1}{cL^2}I^2
\]
Da die Oberflächentemperatur $T_4$ der Schienen 2 und 3 der stationären Lösung am Rand entsprechen muss lautet das nichtlineare Gleichungssystem
\begin{align*}
	\dot{q}_1(T_1,T_4) - \frac{\rho_1}{cL^2}I^2 &= 0 \\
	T_4 - T_\infty - \frac{\dot{q}_4(T_1,T_4)}{\lambda}b &= 0
\end{align*}
Alternativ kann auch die eine Randbedingung aus Punkt d) angepasst werden. Diese lautet damit
\[
	-\lambda\frac{\partial}{\partial z}T_S(0) = \dot{q}_4(T_1,T_4)
\]