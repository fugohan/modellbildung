\textbf{Beispiel 3}\\ \\
a)\\ \\
Da sich hier um konvexe Flächen handelt lauten die jeweiligen Sichtfaktoren auf sich selbst
\begin{align*}
	F_{aa} &= 0 \\
	F_{bb} &= 0 \\
	F_{xx} &= 0
\end{align*}
b) \\ \\
Hier lauten die gesuchten Sichtfaktoren
\begin{align*}
	F_{a1} &= 1 \\
	F_{1a} &= \frac{l_a}{l_1}
\end{align*}
Der zweite Sichtfaktor wurde mithilfe des Reziprozitätsgesetz bestimmt.\\ \\
c) \\ \\
Aus der Summationregel folgen die drei Gleichungen
\begin{align*}
	1 &= F_{ab} + F_{ax} \\
	1 &= F_{ba} + F_{bx} \\
	1 &= F_{xa} + F_{xb}
\end{align*}
\newpage
\noindent
d) \\ \\
Nun kann das Gleichungssystem aufgestellt werden und dieses lautet
\[
	\begin{bmatrix}
		1 \\
		1 \\
		1
	\end{bmatrix}
	=
	\begin{bmatrix}
		1 & 1 & 0 \\
		\frac{l_a}{l_b} & 0 & 1 \\
		0 & \frac{l_a}{l_x} & \frac{l_b}{l_x}
	\end{bmatrix}
	\begin{bmatrix}
		F_{ab} \\
		F_{ax} \\
		F_{bx}
	\end{bmatrix}
\]
Beweis:
\begin{align*}
	1 &= F_{ab} + F_{ax} \\
	1 &= F_{ab}\frac{l_a}{l_b} + F_{bx} \\
	1 &= F_{ax}\frac{l_a}{l_x} + F_{bx}\frac{l_b}{l_x} \\
	F_{bx} &= \left(1 - F_{ax}\frac{l_a}{l_x}\right) \frac{l_x}{l_b} \\
	       &= \frac{l_x}{l_b} - F_{ax}\frac{l_a}{l_b} 
\end{align*}
Durch einsetzen in die zweite Gleichung folgt
\begin{align*}
	1 &= F_{ab}\frac{l_a}{l_b} + \frac{l_x}{l_b} - F_{ax}\frac{l_a}{l_b} \\
	F_{ax} &= \left(F_{ab}\frac{l_a}{l_b} + \frac{l_x}{l_b} - 1\right) \frac{l_b}{l_a} \\
	&= F_{ab} + \frac{l_x}{l_a} - \frac{l_b}{l_a}
\end{align*} 
Jetzt muss man noch in die erste Gleichung einsetzten und daraus folgt
\begin{align*}
	1 &= F_{ab} + F_{ab} + \frac{l_x}{l_a} - \frac{l_b}{l_a} \\
	2F_{ab} &= 1 - \frac{l_x}{l_a} + \frac{l_b}{l_a} \\
	2F_{ab} &= \frac{l_a + l_b - l_x}{l_a} \\
	F_{ab} &= \frac{l_a + l_b - l_x}{2l_a}
\end{align*}
e) \\ \\
Der Sichtfaktor $F_{ac}$ kann auch mit der Cross-String-Methode bestimmt und lautet daher
\[
	F_{ac} = \frac{l_x + l_y - l_b - l_d}{2l_a}
\]
Der Lösungsweg von $F_{ad}$ ist analog zum dem in Punkt d und daher lautet der Sichtfaktor
\[
	F_{ad} = \frac{l_a + l_d - l_y}{2l_a}
\]
f) \\ \\
Der Lösungsweg in der Musterlösung scheint nicht korrekt zu sein. Daher wurde keine Berechnung durchgeführt die man mit dieser hätte vergleichen können.