\textbf{Beispiel 1} \\ \\
a)\\ \\
Die einzelnen Koordinaten der Position des Stabschwerpunktes können direkt aus der Angabe abgelesen werden und lauten daher
\begin{align*}
	x_s &= r\cos\alpha + l_s\sin\beta \\
	y_s &= r\sin\alpha - l_s\cos\beta
\end{align*}
Dessen Ableitungen lauten
\begin{align*}
	\dot{x}_s &= -r\dot{\alpha}\sin\alpha + l_s\dot{\beta}\cos\beta \\
	\dot{y}_s &= r\dot{\alpha}\cos(\alpha) + l_s\dot{\beta}\sin\beta
\end{align*}
b) \\ \\
Anfangs werden die beiden abgeleiteten Koordinaten zu einem Vektor zusammengefasst.
\[
	\textbf{v}_s = \begin{bmatrix}
		-r\dot{\alpha}\sin\alpha + l_s\dot{\beta}\cos\beta \\
		r\dot{\alpha}\cos(\alpha) + l_s\dot{\beta}\sin\beta
	\end{bmatrix}
\]
Bevor nun aber die kinetische Energie des Systems bestimmt werden kann muss aber vorher noch ein kleine Zwischenrechnung durchgeführt werden.
\begin{align*}
	v_s^2 &= \textbf{v}_s^T\textbf{v}_s \\
		  &= (-r\dot{\alpha}\sin\alpha + l_s\dot{\beta}\cos\beta)^2 + (r\dot{\alpha}\cos(\alpha) + l_s\dot{\beta}\sin\beta)^2 \\
		  &= r^2\dot{\alpha}^2\sin^2\alpha - 2rl_s\dot{\alpha}\dot{\beta}\sin\alpha\sin\beta + l_s^2\dot{\beta}^2\cos^2\beta + \\
		  &+ r^2\dot{\alpha}^2\cos^2\alpha + 2rl_s\dot{\alpha}\dot{\beta}\cos\alpha\cos\beta + l_s^2\dot{\beta}^2\sin^2\beta \\
		  &= r^2\dot{\alpha}^2 + l_s^2\dot{\beta}^2 - 2rl_s\dot{\alpha}\dot{\beta}(\sin\alpha\sin\beta - \cos\alpha\cos\beta) \\
		  &= r^2\dot{\alpha}^2 + l_s^2\dot{\beta}^2 - 2rl_s\dot{\alpha}\dot{\beta}\sin(\alpha - \beta)
\end{align*}
Die kinetische Energie hier lautet
\begin{align*}
	T &= \frac{1}{2}m_sv_s^2 + \frac{1}{2}\Theta_T\dot{\alpha} + \frac{1}{2}\Theta_s\dot{\beta} + \frac{1}{2}m_L(r\dot{\alpha} + \dot{l}_c)
\end{align*}
Der letzte Klammerausdruck setzt sich aus der Rotationsgeschwindigkeit $r\dot{\alpha}$ und der Translationsgeschwindigkeit $\dot{l}_c$ zusammen.