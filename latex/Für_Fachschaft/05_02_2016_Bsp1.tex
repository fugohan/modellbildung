\textbf{Beispiel 1} \\ \\
Bei diesen Beispiel muss man den physikalischen Hintergrund von Seilrollen genau beachten. \\ \\
a)\\ \\
Um die kinetische Energie T zu bestimmen, benötigt man sämtliche im System auftretenden Geschwindigkeiten.
\begin{align*}
	\omega_{R_1} &= \frac{v}{r_1} \\
	v_{R_2} &= v \frac{r_1}{2r_2} \\
	\omega_{R_2} &= v\frac{r_1}{2r_2^2}
\end{align*}
Somit ergibt sich die kinetische Energie zu
\[
	T = \frac{m_Lv^2}{2} + \frac{\Theta_1\omega_{R_1}^2}{2} + \frac{\Theta_2\omega_{R_2}^2}{2} + \frac{m_2v_{R_2}^2}{2}
\]
b) \\ \\
Die Energie 
\[
	W = F \cdot s
\]
die beim Verschieben benötigt wird lautet
\[
	W = \mu_cm_Lg\cos\phi s
\]
c) \\ \\
Zum Zeitpunkt $t_0$ soll
\[
	V(t_0) = 0
\]
gültig sein.
Zum Zeitpunkt $t_1$ sieht die potentielle Energie folgendermaßen aus
\[
	V(t_1) = m_Lgs\sin\phi - m_2g\frac{r_1}{2r_2}
\]
d) \\ \\
Die aufzuwendende Arbeit durch die Kraft $\textbf{F}$ lautet
\[
	W = Fs\frac{r_1}{2r_2}
\]
e) \\ \\
Aus der Energieerhaltung ergibt sich
\[
	T(t_0) + V(t_0) + W_F - W = T(t_1) + V(t_1)
\]
Bei den Termen mit $t_0$ müssen für $v$ und $\omega$, $v_0$ und $\omega(t_0)$ und bei $t_1$, $v_1$ und $\omega(t_1)$ eingesetzt werden. Schreibt man nun die obige Gleichung komplett aus und formt dieses dann auf die gesuchte Kraft um erhält man
\begin{align*}
	F &=  \frac{2r_2}{sr_1}\left( m_L\frac{v_1^2 - v_0^2}{2} + \Theta_1 \frac{\omega_{R_1}(t_1)^2 - \omega_{R_1}(t_0)^2}{2} + \Theta_2 \frac{\omega_{R_2}(t_1)^2 - \omega_{R_2}(t_0)^2}{2}\right) + \\
	&+\frac{2r_2}{sr_1}  \left(m_2\frac{v_{R_2}(t_1)^2 - v_{R_2}(t_0)^2}{2} + m_Lgs(\sin\phi + \mu_c\cos\phi) - m_2gs\frac{r_1}{2r_2}\right)
\end{align*}