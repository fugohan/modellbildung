\textbf{Beispiel 2}\\ \\
a)\\ \\
Die beiden gesuchten Vektoren lauten
\begin{align*}
	\textbf{s}_m = \begin{bmatrix}
		l\sin(\alpha)\cos(\varphi) \\
		l\sin(\alpha)\sin(\varphi) \\
		L - l\cos(\alpha)
	\end{bmatrix}
	\qquad
	\textbf{s}_A = \begin{bmatrix}
		0 \\
		0 \\
		L - 2b\cos(\alpha)
	\end{bmatrix}
\end{align*}
b)\\ \\
Durch Ableiten nach der Zeit erhält man den gesuchten Geschwindigkeitsvektor und dieser lautet
\[
	\textbf{v}_m = l \begin{bmatrix}
			\dot{\alpha}\cos(\alpha)\cos(\varphi) - \dot{\varphi}\sin(\alpha)\sin(\varphi) \\
			\dot{\alpha}\cos(\alpha)\sin(\varphi) + \dot{\varphi}\sin(\alpha)\cos(\varphi) \\
			\dot{\alpha}\sin(\alpha)
	\end{bmatrix}
\]
\newpage
\noindent
c)\\ \\
Die potentielle Energie des Systems lautet
\[
	V = 2 m g (L - l\cos(\alpha))
\]
Der Ansatz für die kinetische Energie lautet
\[
	T = 2 \frac{1}{2} m \textbf{v}_m^T\textbf{v}_m = m \textbf{v}_m^T\textbf{v}_m
\]
d)\\ \\
Anfangs muss die Lagrange-Funktion bestimmt werden. Diese lautet hier
\[
	L = T - V = ml^2\left(\dot{\alpha}^2 + \dot{\varphi}^2\sin^2(\alpha)\right) - 2 m g (L - l\cos(\alpha))
\]
Die Form des Euler-Lagrange-Formalismus kann der Formelsammlung entnommen werden. Die notwendigen Ableitung lauten
\begin{align*}
	\frac{\partial L}{\partial \alpha} &= 2ml^2\dot{\varphi}^2\sin(\alpha)\cos(\alpha) - 2mgl\sin(\alpha) \\
	\frac{\partial L}{\partial \varphi} &= 0 \\
	\frac{\partial L}{\partial \dot{\alpha}} &= 2ml^2\dot{\alpha} \\
	\frac{\partial L}{\partial \dot{\varphi}} &= 2ml^2\dot{\varphi}\sin^2(\alpha)
\end{align*}
und
\begin{align*}
	\frac{\text{d}}{\text{d}t}\frac{\partial L}{\partial \dot{\alpha}}  &= 2ml^2\ddot{\alpha} \\
	\frac{\text{d}}{\text{d}t}\frac{\partial L}{\partial \dot{\varphi}} &= 2ml^2\ddot{\varphi} + 4ml^2\dot{\varphi}\sin(\alpha)\cos(\alpha)\dot{\alpha}
\end{align*}
Die beiden Euler-Lagrange-Gleichungen lauten
\begin{align*}
	2ml^2\ddot{\alpha} - 2ml^2\dot{\varphi}^2\sin(\alpha)\cos(\alpha) + 2mgl\sin(\alpha) &= 0 \\
	2ml^2\ddot{\varphi} + 4ml^2\dot{\varphi}\sin(\alpha)\cos(\alpha)\dot{\alpha} &= \tau
\end{align*}
Vereinfacht man diese beiden Gleichungen soweit wie möglich erhält man die Bewegungsgleichungen
\begin{align*}
	\ddot{\alpha} &= \frac{\sin(\alpha)(l\dot{\varphi}^2\cos(\alpha) - g)}{l} \\
	\ddot{\varphi} &= \frac{\tau - 4ml^2\dot{\alpha}\dot{\varphi}\sin(\alpha)\cos(\alpha)}{2ml^2\sin^2(\alpha)}
\end{align*}
\newpage
\noindent
e)\\ \\
Im stationären Fall mit konstanter Winkelgeschwindigkeit existieren die Bedingungen
\[
	\dot{\alpha} = \ddot{\alpha} = \ddot{\varphi} = 0
\]
Aus der ersten Bewegungsgleichung kann man auf die Gleichung
\[
	0 = \frac{\sin(\alpha)(l\dot{\varphi}^2\cos(\alpha) - g)}{l}
\]
schließen und daraus folgt
\[
	   \alpha \in \left\{0, \arccos\left(\frac{q}{l\dot{\varphi}^2}\right)
	  \right\}
\]
f)\\ \\
Die gesuchte Funktion kann einfach durch die z-Komponente des zweiten Vektors aus Punkt a) bestimmt werden und lautet deswegen
\[
	h(\dot{\varphi}) = L - \frac{2bg}{l\dot{\varphi}^2}
\]