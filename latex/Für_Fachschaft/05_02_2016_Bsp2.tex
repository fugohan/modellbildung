\textbf{Beispiel 2} \\ \\
a)\\ \\
Die einzelnen Koordinaten der Position des Stabschwerpunktes können direkt aus der Angabe abgelesen werden und lauten daher
\begin{align*}
	x_s &= r\cos\alpha + l_s\sin\beta \\
	y_s &= r\sin\alpha - l_s\cos\beta
\end{align*}
Dessen Ableitungen lauten
\begin{align*}
	\dot{x}_s &= -r\dot{\alpha}\sin\alpha + l_s\dot{\beta}\cos\beta \\
	\dot{y}_s &= r\dot{\alpha}\cos(\alpha) + l_s\dot{\beta}\sin\beta
\end{align*}
b) \\ \\
Anfangs werden die beiden abgeleiteten Koordinaten zu einem Vektor zusammengefasst.
\[
	\textbf{v}_s = \begin{bmatrix}
		-r\dot{\alpha}\sin\alpha + l_s\dot{\beta}\cos\beta \\
		r\dot{\alpha}\cos(\alpha) + l_s\dot{\beta}\sin\beta
	\end{bmatrix}
\]
Bevor nun aber die kinetische Energie des Systems bestimmt werden kann muss aber vorher noch ein kleine Zwischenrechnung durchgeführt werden.
\begin{align*}
	v_s^2 &= \textbf{v}_s^T\textbf{v}_s \\
		  &= (-r\dot{\alpha}\sin\alpha + l_s\dot{\beta}\cos\beta)^2 + (r\dot{\alpha}\cos(\alpha) + l_s\dot{\beta}\sin\beta)^2 \\
		  &= r^2\dot{\alpha}^2\sin^2\alpha - 2rl_s\dot{\alpha}\dot{\beta}\sin\alpha\sin\beta + l_s^2\dot{\beta}^2\cos^2\beta + \\
		  &+ r^2\dot{\alpha}^2\cos^2\alpha + 2rl_s\dot{\alpha}\dot{\beta}\cos\alpha\cos\beta + l_s^2\dot{\beta}^2\sin^2\beta \\
		  &= r^2\dot{\alpha}^2 + l_s^2\dot{\beta}^2 - 2rl_s\dot{\alpha}\dot{\beta}(\sin\alpha\sin\beta - \cos\alpha\cos\beta) \\
		  &= r^2\dot{\alpha}^2 + l_s^2\dot{\beta}^2 - 2rl_s\dot{\alpha}\dot{\beta}\sin(\alpha - \beta)
\end{align*}
Die kinetische Energie hier lautet
\begin{align*}
	T &= \frac{1}{2}m_sv_s^2 + \frac{1}{2}\Theta_T\dot{\alpha} + \frac{1}{2}\Theta_s\dot{\beta} + \frac{1}{2}m_L(r\dot{\alpha} + \dot{l}_c)
\end{align*}
Der letzte Klammerausdruck setzt sich aus der Rotationsgeschwindigkeit $r\dot{\alpha}$ und der Translationsgeschwindigkeit $\dot{l}_c$ zusammen.
\newpage
\noindent
c) \\ \\
Die potentielle Energie setzt sich aus 2 Komponenten zusammen, die Energie bezüglich der Gewichtskraft
\[
	V_g = m_Lg(c - l_c - r\alpha) + m_sgy_s
\]
mit der wählbaren Länge $c$ und der in den Federn gespeicherten Energie
\[
	V_f = c_1\frac{(\beta - \alpha)^2}{2} + c_2 \frac{(l_c - l_{c,0})^2}{2}
\]
Die Summe der beiden ergibt die gesamte potentielle Energie
\begin{align*}
	V &= V_g + V_f \\
	  &= m_Lg(c - l_c - r\alpha) + m_sgy_s + c_1\frac{(\beta - \alpha)^2}{2} + c_2 \frac{(l_c - l_{c,0})^2}{2}
\end{align*}
d) \\ \\
Die einzige externe Kraft ist jene Kraft $F_e$, die an der Last angreift. Der Vektor zum Angriffspunkt dieser Kraft lautet
\[
	\textbf{p}_f = \begin{bmatrix}
		-r \\
		-r\alpha - l_c - c
	\end{bmatrix}
\]
Dessen Ableitungen nach den gewünschten generalisierten Koordinaten lauten
\begin{align*}
	\frac{\partial \textbf{p}_f}{\partial \alpha} &= \begin{bmatrix}
		0 \\
		-r
	\end{bmatrix}
	\\
	\frac{\partial \textbf{p}_f}{\partial \beta} &= \begin{bmatrix}
		0 \\
		0
	\end{bmatrix}
	\\
	\frac{\partial \textbf{p}_f}{\partial l_c} &= \begin{bmatrix}
		0 \\
		-1
	\end{bmatrix}
\end{align*}
Mit dem Vektor
\[
	\textbf{F}_e = \begin{bmatrix}
		0 \\
		-F_e
	\end{bmatrix}
\]
kann mit der Formel
\[
	\tau = \textbf{F}^T \frac{\partial \textbf{p}}{\partial \textbf{q}}
\]
die generalisierten Kräfte bestimmt werden.
Diese lauten
\begin{align*}
	\tau_\alpha &= rF_e \\
	\tau_\beta &= 0 \\
	\tau_{l_c} &= F_e
\end{align*}
\newpage
\noindent
e) \\ \\
Die Lagrange-Funktion sieht hier folgendermaßen aus.
\begin{align*}
	L &= T - V \\
	  &= \frac{1}{2}m_sv_s^2 + \frac{1}{2}\Theta_T\dot{\alpha} + \frac{1}{2}\Theta_s\dot{\beta} + \frac{1}{2}m_L(r\dot{\alpha} + \dot{l}_c) - \\
	  &- m_Lg(c - l_c - r\alpha) - m_sgy_s - c_1\frac{(\beta - \alpha)^2}{2} - c_2 \frac{(l_c - l_{c,0})^2}{2}
\end{align*}
Die Euler-Lagrange-Gleichungen lauten daher hier
\begin{align*}
	\frac{d}{dt}\left(\frac{\partial L}{\partial \dot{\alpha}}\right) - \frac{\partial L}{\partial \alpha} &= \tau_\alpha \\
	\frac{d}{dt}\left(\frac{\partial L}{\partial \dot{\beta}}\right) - \frac{\partial L}{\partial \beta} &= \tau_\beta \\
	\frac{d}{dt}\left(\frac{\partial L}{\partial \dot{l}_c}\right) - \frac{\partial L}{\partial l_c} &= \tau_{l_c}
\end{align*}