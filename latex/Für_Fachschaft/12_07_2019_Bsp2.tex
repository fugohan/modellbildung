\newpage
\noindent
\textbf{Beispiel 2} \\ \\
a) \\ \\
Da es sich hier um ein Problem mit Zylinderkoordinaten hält, verwendet man nun einfach die Wärmeleitgleichung für Zylinderkoordinaten aus Formelsammlung und adaptiert diese entsprechend der Angabe. \\
\newline
Wärmeleitgleichung:
\[
	0 = \lambda \left( \frac{1}{r} \frac{d}{dr}\left( \frac{d}{dr} T_i\left( r\right)\right)\right)
\]
Die 0 auf der linken Seite beruht darauf, das es sich hier um ein stationäres Problem handelt und die Terme für $\varphi$ und $z$ fallen ebenfalls weg. \\
\newline
Lösung der DGL:
\begin{align*}
	\lambda_i \left( \frac{1}{r} \frac{d}{dr}\left( r \frac{d}{dr} T_i\left( r\right)\right)\right) &= 0
	\\
	\frac{d}{dr}\left( r \frac{d}{dr} T_i\left( r\right)\right) &= 0
	\\
	\int \frac{d}{dr}\left( r \frac{d}{dr} T_i\left( r\right)\right) dr &= \int 0 dr
	\\
	r \frac{d}{dr} T_i\left( r\right) &= C_3
	\\
	\frac{d}{dr} T_i\left( r\right) &= \frac{C_3}{r}
	\\
	\int \frac{d}{dr} T_i\left( r\right) dr &= \int \frac{C_3}{r} dr
	\\
	T_i\left( r\right) &= C_3 ln\left(r \right) + C_4
\end{align*}
Randbedingungen:
\begin{align*}
	T_i\left(2R\right) &= C_3 ln\left( 2R\right) + C_4 = T_2 \\
	T_i\left(2R\right) &= C_3 ln\left( 3R\right) + C_4 = T_3
\end{align*}
Integrationskonstanten: \\
\newline
Ausgehend von dem Gleichungssystem der Randbedingung kann man die Integrationskonstanten sehr leicht bestimmen. Als ersten formt man die 2.te Gleichung auf $C_4$ um.
\begin{align*}
	T_3 &= C_3 ln\left( 3R\right) + C_4\\
	C_4 &= T_3 - C_3 ln\left( 3R\right)
\end{align*}
Dann wird in Gleichung 1 eingesetzt
\begin{align*}
	T_2 &= C_3 ln\left( 2R\right) + T_3 - C_3 ln\left( 3R\right) \\
	T_2 - T_3 &= C_3 \underbrace{\left( ln\left( 2R\right) - ln\left( 3R\right) \right)}_{ln\left( \frac{2}{3}\right)} \\
	C_3 &= \frac{T_2 - T_3}{ln\left( \frac{2}{3}\right)}
\end{align*}
Nun setzt man $C_3$ in $C_4$ ein
\begin{align*}
	C_4 &= T_3 -  \frac{T_2 - T_3}{ln\left( \frac{2}{3}\right)} ln\left( 3R\right) \\
	C_4 &= \frac{T_3 \left( ln\left( 2R\right) - ln\left( 3R\right) \right)}{ln\left( \frac{2}{3}\right)} - \frac{T_2 - T_3}{ln\left( \frac{2}{3}\right)} ln\left( 3R\right) \\
	C_4 &= \frac{T_3 ln\left( 2R \right) - T_2 ln\left( 3R\right)}{ln\left( \frac{2}{3}\right)}
\end{align*}
b) \\ \\	
Die Leistungsdichte $g$ wird wie folgt berechnet
\[
	g = \rho_e |\textbf{J}|^2 = \rho_e \left( \frac{I}{A}\right)^2
\]
mit
\[
	A = \left(4R^2 - R^2\right)\pi
\]
Dies ist die Fläche eine Hohlleiters. Setzt man nun diese in $g$ ein, erhält man
\[
	g = \rho_e \left( \frac{I}{\left(4R^2 - R^2\right)\pi}\right)^2
\]
c) \\ \\
DGL:
\begin{align*}
	\lambda_l \left( \frac{1}{r}\frac{d}{dr}\left( r\frac{d}{dr} T_l\left(r\right)\right)\right) + g = 0 \\
	\frac{d}{dr}\left( r\frac{d}{dr} T_l\left(r\right)\right) = -\frac{gr}{\lambda_l} \\
	\int \frac{d}{dr}\left( r\frac{d}{dr} T_l\left(r\right)\right) dr = \int -\frac{gr}{\lambda_l} dr \\
	 r\frac{d}{dr} T_l\left(r\right) = -\frac{gr^2}{2\lambda_l} + C_1 \\
	 \frac{d}{dr} T_l\left(r\right) = -\frac{gr}{2\lambda_l} + \frac{C_1}{r} \\
	 \int \frac{d}{dr} T_l\left(r\right) dr = \int -\frac{gr}{2\lambda_l} + \frac{C_1}{r} dr \\
	 T_l\left(r\right) = -\frac{gr^2}{4\lambda_l} + C_1\ln\left( r \right) + C_2
\end{align*}
Randbedingungen
\begin{align*}
	T_l\left(R\right) = -\frac{gR^2}{4\lambda_l} + C_1\ln\left(R\right) + C_2 = T_1 \\
	T_l\left(2R\right) = -\frac{g4R^2}{4\lambda_l} + C_1 \ln\left(2R\right) + C_2 = T_2
\end{align*}
\newpage
\noindent
Integrationskonstanten: \\ \\
Die beiden Konstanten werden wie in a) bestimmt.
\begin{align*}
	-\frac{gR^2}{4\lambda_l} + C_1\ln\left(R\right) + C_2 = T_1 \\
	C_2 = T_1 + \frac{gR^2}{4\lambda_l} - C_1\ln\left(R\right)
\end{align*}
Diese Gleichung wird nun in die andere eingesetzt und man erhält
\begin{align*}
	-\frac{g4R^2}{4\lambda_l} + C_1 \ln\left(2R\right) + T_1 + \frac{gR^2}{4\lambda_l} - C_1\ln\left(R\right) = T_2 \\
	\frac{g}{4\lambda_l}\left(R^2 - 4R^2\right) 
	+ T_1 - T_2 = C_1 \underbrace{\left( ln\ln\left(1\right) - \ln\left(2\right)\right)}_{\ln\left( \frac{1}{2}\right)} \\
	C_1 = \frac{\frac{g}{4\lambda_l}\left(R^2 - 4R^2\right) + T_1 - T_2}{\ln\left( \frac{1}{2}\right)}
\end{align*}
Rückeingesetzt erhält man
\begin{align*}
	C_2 &= T_1 + \frac{gR^2}{4\lambda_l} - \frac{\frac{g}{4\lambda_l}\left(R^2 - 4R^2\right) + T_1 - T_2}{\ln\left( \frac{1}{2}\right)}\ln\left(R\right) \\
	&= \frac{T_1\left(\ln\left(R\right) - \ln\left(2R\right)\right)}{ln\left(\frac{1}{2}\right)} + 
	\frac{\frac{gR^2}{4\lambda_l}\left(\ln\left(R\right) - \ln\left(2R\right)\right)}{\ln\left( \frac{1}{2}\right)} \\
	&- \frac{\frac{g}{4\lambda_l}\left(R^2\ln\left(R\right) - 4R^2\ln\left(R\right)\right) + T_1\ln\left(R\right) - T_2\ln\left(R\right)}{\ln\left( \frac{1}{2}\right)} \\
	&= \frac{\frac{g}{4\lambda_l}\left(4R^2\ln\left(R\right) - R^2\ln\left(2R\right)\right) + T_2\ln\left(R\right) - T_1\ln\left(2R\right)}{\ln\left(\frac{1}{2}\right)}
\end{align*}
d) \\ \\
Zur eindeutigen Bestimmung der Temperaturen \(T_1 , T_2 , T_3\) muss einfach nur die Gleichungen für die Wärmeübergänge aufstellen. Diese lauten für \\ \\
\(T_2:\)
\[
	\lambda_i\frac{d}{dr}T_i\left(r\right)|_{r = 2R} = \lambda_l \frac{d}{dr}T_l\left(r\right)|_{r = 2R}
\]
\(T_3:\)
\[
	\alpha_0\left(T_\infty - T_3\right) = \lambda_i\frac{d}{dr}T_i\left(r\right)|_{r = 3R}
\]
\(T_1:\)
\[
	\alpha_f\left(T_f - T_1\right) = -\lambda_l\frac{d}{dr}T_l\left(r\right)|_{r = R}
\]
\newpage
\noindent
e) \\ \\
1) Temperatur der Kühlflüssigkeit senken \\
2) Oberfläche der Isolierung vergrößern (z.B. durch Kühlrippen) \\
3) Wärmeleitfähigkeit der Isolierung vergrößern \\
\\
\textit{Hinweis}:\\
\textit{Die Lösung in diesen Punkt erhält man durch logisch denken.}
