\newpage
\noindent
\textbf{Beispiel 3}\\ \\
a)\\ \\
Die richtige Formel für die Konvektion kann aus der Formelsammlung entnommen werden. Daher lautet die gesuchte Wärmestromdichte
\[
	\dot{q}_{aw}(x,t) = \alpha_{aw}\left(T_a(x,t) - T_w(x,t)\right)
\]
b)\\ \\
Die gesuchte Wärmestromdichten können mittels des Wärmeleitgesetzes bestimmt werden und lauten somit
\begin{align*}
	\dot{q}_a &= -\lambda\frac{\partial}{\partial x}T_a(x,t) \\
	\dot{q}_w &= -\lambda\frac{\partial}{\partial x}T_w(x,t)
\end{align*}
c)\\ \\
Mithilfe des Hinweises aus der Angabe und der Wärmeleitgleichung für kartesische Koordinaten lautet die gesuchte Differentialgleichung
\[
	\rho_a c_a \frac{\partial}{\partial t}T_a(x,t) = \lambda\frac{\partial^2}{\partial x^2}T_a(x,t) + \frac{\alpha_a}{d_a}\dot{q}_s - \frac{\alpha_{aw}}{d_w}\left(T_a(x,t) - T_w(x,t)\right)
\]
Der zweite und der dritte Term auf der rechten Seite der Gleichung sind der zufließende und abfließende Wärmestrom aus dem Kontrollvolumen. \\ \\
d)\\ \\
Analog zu Punkt c) lautet die Differentialgleichung hier
\begin{align*}
	\underbrace{\rho_wc_w\frac{\partial}{\partial t}T_w(x,t) = \lambda\frac{\partial^2}{\partial x^2}T_w(x,t) - \rho_wc_wv_w\frac{\partial}{\partial x}T_w(x,t)}_{\text{Wärmeleitgleichung}} + \frac{\alpha_{aw}}{d_w}\left(T_a(x,t) - T_w(x,t)\right) \\
	\rho_w c_w \left(\frac{\partial}{\partial t}T_w(x,t) + v_w\frac{\partial}{\partial x}T_w(x,t)\right) = \lambda\frac{\partial^2}{\partial x^2}T_w(x,t) + \frac{\alpha_{aw}}{d_w}\left(T_a(x,t) - T_w(x,t)\right)
\end{align*}
e)\\ \\
Für das gesamte System werden hier 4 Randbedingungen benötigt. Nämlich 2 pro Differentialgleichung.\\ \\
f)\\ \\
Die Randbedingungen dieses Systems lauten 
\begin{align*}
	\dot{q}_a(0,t) &= \dot{q}_a(L,t) = 0 \\
	T_w(0,t) &= T_w^{in}(t) \\
	\dot{q}_w(L,t) &= 0
\end{align*}