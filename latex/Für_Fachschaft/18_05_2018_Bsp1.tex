\textbf{Beispiel 1}\\ \\
a)\\ \\
Die gesuchte Differentialgleichung kann mithilfe der Energieerhaltung aufgestellt werden und lautet daher
\[
	\rho c_pl_chd\left(\frac{\partial T}{\partial t}\right) = P - \dot{Q}
\]
Der Wärmestrom $\dot{Q}$ besitzt ein negatives Vorzeichen, da dieser aus dem Chip austritt.\\ \\
b)\\ \\
Die gesuchte Temperatur lautet im stationären Zustand
\[
	T_0 = \frac{P}{dh}\left(kl_w  - \frac{l_w^3}{3} + \frac{l_k}{\lambda_k} + \frac{1}{\alpha_{wk}} + \frac{1}{\alpha_{ku}}\right) + T_\infty
\]
Diese Gleichung erhält man durch Umformen un der Formel \textit{Wärmestromdichte und Wärmeübergangskoeffizient bei mehrschichtigen Wandaufbau} aus der Formelsammlung.\\ \\
c)\\ \\
Die maximale Wärmeleistung pro Chipfläche lautet
\[
	p_{max} = \frac{1}{k}\left(\frac{\partial T_w(x)}{\partial x}\right)_{\text{max}}
\]
Durch den gegebenen Temperaturgradienten lässt sich diese auf
\[
	p_{max} = \frac{G}{k}
\]
vereinfachen.