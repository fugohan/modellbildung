\textbf{Beispiel 4}\\ \\
a)\\ \\
Die beiden Wärmeleitgleichungen lauten
\begin{align*}
	\rho_Ec_{p,E}T_{E,t}(t,x) &= \lambda_ET_{E,xx}(t,x) \\
	\rho_Ec_{p,W}T_{W,t}(t,x) &= \lambda_WT_{W,xx}(t,x)
\end{align*}
mit der Randbedingung
\[
	T_E(t,d(t)) = T_W(t,d(t))
\]
b)\\ \\
Die beiden gesuchten Randbedingungen lauten
\begin{align*}
	T_E(t,0) &= T_0 \\
	T_{W,x}(t,D) &= 0
\end{align*}
c)\\ \\
Die Energiebilanz lautet für dieses Problem
\[
	h\rho_E\dot{d}(t) = \lambda_ET_{E,x}(t,d(t)) - \lambda_WT_{W,x}(t,d(t))
\]
\newpage
\noindent
d) \\ \\
Die stationäre Verhalten der Gleichungen lauten
\begin{align*}
	\text{Eis} &: \left\{
	\begin{array}{lll}
		0 &= T_{E,xx}(x,t) \\
		T_E(0,t) &= T_0 \\
		T_E(d,t) &= T_s
	\end{array}
	\right.
	\\
	\text{Wasser} &: \left\{
		\begin{array}{lll}
			0 &= T_{W,xx}(x,t) \\
			T_W(d,t) &= T_s \\
			T_{W,x}(D,t) &= 0
		\end{array}
		\right.
\end{align*}
Durch Lösen der Gleichungen aus a) und Einsetzen dieser Verhalten lauten die Lösungen der Gleichungen
\begin{align*}
	T_E(x,t) &= T_0 + (T_s - T_0)\frac{x}{d(t)} \\
	T_W(x,t) &= T_s
\end{align*}
e) \\ \\
Durch Einsetzen der Ergebnisse von d) in die Energiebilanz von c) erhält man die Differentialgleichung
\[
	h\rho_E\dot{d}(t) = \lambda\frac{T_s - T_0}{d(t)}
\]
Durch Lösen dieser Gleichung beim gegebenen Randwert erhält man
\[
	d(t) = \sqrt{\frac{2\lambda_E(T_s - T_0)}{h\rho_E}t}
\]