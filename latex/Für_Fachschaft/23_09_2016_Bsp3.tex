\newpage
\noindent
\textbf{Beispiel 3}\\ \\
a)\\ \\
Man kann das System noch mit dem Vektor $\textbf{q} = [x_W , x_{1L} , y_{1L}]^T$ beschrieben werden. Die letzten beiden Koordinaten sind die Koordinaten des Lastschwerpunktes bezogen auf das Inertialsystem. Ein weiteres Beispiel wäre $\textbf{q} =[x_w , \alpha , l]^T$, wobei $l$ die abgewickelte Seillänge ist.
b)\\ \\
Die erforderlichen Ortsvektoren sind jene die vom Ursprung des Inertialsystem zu den Schwerpunkten der einzelnen Teile zeigen.
\begin{align*}
	\textbf{r}_W &= \begin{bmatrix}
		x_W \\
		0 \\
		0
	\end{bmatrix}
	\\
	\textbf{r}_L &= \begin{bmatrix}
		x_W - d_0 - d_1 + x_L - (l_0 + r_T\varphi)\sin(\alpha) \\
		-y_L - (l_0 + r_T\varphi)\cos(\alpha) \\
		0
	\end{bmatrix}
\end{align*}
c)\\ \\
Die relevanten Geschwindigkeitsvektoren und Winkelgeschwindigkeitsvektor lauten
\begin{align*}
	\dot{\textbf{r}}_W &= \begin{bmatrix}
		\dot{x}_W \\
		0 \\
		0
	\end{bmatrix}
	\\
	\dot{\textbf{r}}_L &= \begin{bmatrix}
		\dot{x}_W - r_T\dot{\varphi}\sin(\alpha) - (l_0 + r_T\varphi)\cos(\alpha)\dot{\alpha} \\
		-r_T\dot{\varphi}\cos(\alpha) + (l_0 + r_T\varphi)\sin(\alpha)\dot{\alpha} \\
		0
	\end{bmatrix}
	\\
	\omega_T &= \begin{bmatrix}
	 	0 \\
	 	0 \\
	 	\dot{\varphi}
	\end{bmatrix}
\end{align*}
Die kinetische Energien der einzelnen Teile lauten
\begin{align*}
	T_W &= \frac{1}{2}m_W\dot{x}^2_W \\
	T_T &= \frac{1}{2}J_T\dot{\varphi}^2 \\
	T_L &= \frac{1}{2}m_L ((	\dot{x}_W - r_T\dot{\varphi}\sin(\alpha) - (l_0 + r_T\varphi)\cos(\alpha)\dot{\alpha})^2 \\
	&+ (	-r_T\dot{\varphi}\cos(\alpha) + (l_0 + r_T\varphi)\sin(\alpha)\dot{\alpha})^2)
\end{align*}
Die gesamte kinetische Energie des Systems lautet
\[
	T_{ges} = T_W + T_T + T_L
\]
\newpage
\noindent
d)\\ \\
Die potentiellen Einzelenergien lauten
\begin{align*}
	V_W &= 0 \\
	V_L &= -m_L g (y_L + (l_0 + r_T\varphi)\cos(\alpha))
\end{align*}
Die potentielle Energie des Wagens ist 0, weil sich der Wagen immer in derselben Ebene befindet. Die gesamte potentielle Energie des Systems lautet
\[
	V_{ges} = V_L
\]
e)\\ \\
Der Vektor der generalisierten Kräfte kann unmittelbar aus Angabe bestimmt werden, da jede externe Kraft bzw. Moment unmittelbar auf der entsprechenden generalisierten Koordinate wirkt. Der Vektor lautet daher
\[
	\tau = \begin{bmatrix}
		F_{an} - F_R \\
		0 \\
		M_T
	\end{bmatrix}
\]
f)\\ \\
Die Lagrange-Funlktion lautet
\[
	L = T_{ges} - V_{ges}
\]
und damit die Euler-Lagrange-Gleichungen
\[
	\frac{\partial}{\partial t} \left( \frac{\partial L}{\partial \dot{\textbf{q}}}\right) - \frac{\partial L}{\partial \textbf{q}} =  \tau
\]