\textbf{Beispiel 1}\\ \\
a)\\ \\
i)\\ \\
Die Masse des Aufbaues lautet
\[
	m_a = (l + l_1)bh\rho
\]
Um den gesuchten Vektor zu ermitteln, müssen zuerst die Schwerpunktsvektoren der Einzelvolumina bestimmt werden. Anschließend kann dann der gtesuchte Vektor bestimmt werden und dieser lautet
\[
	\textbf{s}_a = \frac{1}{l + l_1}\begin{bmatrix}
		bl_1 \\
		\frac{l^2}{2} + ll_1 - \frac{l_1^2}{2}
	\end{bmatrix}
\]
ii)\\ \\
Sämtliche Formel die zur Berechnung benötigt werden stehen in der Formelsammlung. Durch die korrekte Anwendung dieser Formel ergeben sich für die Trägheitsmomente der seperaten Volumina
\begin{align*}
	\varTheta_{\overline{y}\,\overline{y},Q1}^{(S)} &= h\rho\frac{b^3l + bl^3}{12} \\
	\varTheta_{\overline{y}\,\overline{y},Q2}^{S} &= h\rho\frac{b^3l_1 + bl_1^3}{12}
\end{align*}
Mithilfe des Satz von Steiner kann nun das gesuchte Massenträgheitsmoment bestimmt werden.
\[
	\varTheta_{\overline{y}\,\overline{y},a} = \varTheta_{\overline{y}\,\overline{y},Q1} + m_{Q1}(\textbf{s}_{Q1} - \textbf{s}_a)^T(\textbf{s}_{Q1} - \textbf{s}_a) + \varTheta_{\overline{y}\,\overline{y},Q2} + m_{Q2}(\textbf{s}_{Q2} - \textbf{s}_a)^T(\textbf{s}_{Q2} - \textbf{s}_a) 
\]
mit den Massen
\[
	m_{Q1} = lbh\rho \qquad m_{Q2} = l_1bh\rho
\]
und den Vektoren
\[
	\textbf{s}_{Q1} = \begin{bmatrix}
		0 \\
		\frac{l}{2}
	\end{bmatrix}
	\qquad
	\textbf{s}_{Q2} = \begin{bmatrix}
		b \\
		l - \frac{l_1}{2}
	\end{bmatrix}
\]
b)\\ \\
Der Schwerpunktsvektor des Rades lautet
\[
	\textbf{r}_r = \begin{bmatrix}
		\alpha R \\
		0
	\end{bmatrix}
\]
und der des Aufbaues lautet
\[
	\textbf{r}_a = \begin{bmatrix}
		\alpha R + s_{ax}\cos(\beta) + s_{az}\sin(\beta) \\
		- s_{ax}\sin(\beta) + s_{az}\cos(\beta)
	\end{bmatrix}
\]
Die jeweiligen Geschwindigkeitsvektoren lauten
\begin{align*}
	\textbf{v}_r &= \begin{bmatrix}
		\dot{\alpha}R \\
		0
	\end{bmatrix} \\
	\textbf{v}_a &= \begin{bmatrix}
		\dot{\alpha}R + \dot{\beta}(-s_{ax}\sin(\beta) + s_{az}\cos(\beta)) \\
		\dot{\beta}(- s_{ax}\cos(\beta) - s_{az}\sin(\beta))
	\end{bmatrix}
\end{align*}
c)\\ \\
Die beiden gewünschten Energien lauten hier
\begin{align*}
	V = mgs_{az}\cos(\beta) \\
\end{align*}
Um die kinetische Energie zu bestimmen müssen zuerst wieder einmal Nebenrechnungen durchgeführt werden.
\begin{align*}
	\textbf{v}_r^T\textbf{v}_r &= \dot{\alpha}^2R^2 \\
	\textbf{v}_a^T\textbf{v}_a &= (\dot{\alpha}R + \dot{\beta}(-s_{ax}\sin(\beta) + s_{az}\cos(\beta)))^2 + (\dot{\beta}(-s_{ax}\cos(\beta) - s_{az}\sin(\beta)))^2 \\
	&= \dot{\alpha}^2R^2 + \dot{\beta}^2s^2_{az} + 2\dot{\alpha}R\dot{\beta}s_{az}\cos(\beta)
\end{align*}
Die kinetische Energie lautet
\[
	T = \frac{1}{2}m_a\left(\dot{\alpha}^2R^2 + \dot{\beta}^2s^2_{az} + 2\dot{\alpha}R\dot{\beta}s_{az}\cos(\beta)\right) + \frac{1}{2}m_r\dot{\alpha}^2R^2 + \frac{1}{2}\Theta_{yy,a}\dot{\beta}^2 + \frac{1}{2}\Theta_{yy,r}\dot{\alpha}^2
\]
d)\\ \\
Der Vektor für die generalisierten Kräfte kann unmittelbar aus der Angabe ermittelt werden und lautet daher
\[
	\textbf{f} = \begin{bmatrix}
		\tau_a + FR \\
		Fl\cos(\beta)
	\end{bmatrix}
\]
e)\\ \\
Für den Euler-Lagrange-Formalismus müssen erst einmal die nötigen Ableitungen durchgeführt werden. Für diese wiederum wird die Lagrange-Funktion benötigt und diese lautet hier
\begin{align*}
	L &= T - V \\
	  &= \frac{1}{2}m_a\left(\dot{\alpha}^2R^2 + \dot{\beta}^2s^2_{az} + 2\dot{\alpha}R\dot{\beta}s_{az}\cos(\beta)\right) + \frac{1}{2}m_r\dot{\alpha}^2R^2 + \frac{1}{2}\Theta_{yy,a}\dot{\beta}^2 + \frac{1}{2}\Theta_{yy,r}\dot{\alpha}^2 - mgs_{az}\cos(\beta)
\end{align*}
\newpage
\noindent
Die nötigen Ableitungen lauten
\begin{align*}
	\frac{\partial L}{\partial \alpha} &= 0 \\
	\frac{\partial L}{\partial \beta} &= -m_a\dot{\alpha}R\dot{\beta}s_{az}\sin(\beta) + mgs_{az}\sin(\beta) \\
	\frac{\partial L}{\partial \dot{\alpha}} &= m_a \dot{\alpha}R^2 + m_aR\dot{\beta}s_{az}\cos(\beta) + m_r\dot{\alpha}R^2 + \Theta_{yy,r}\dot{\alpha} \\
	\frac{\partial L}{\partial \dot{\beta}} &= m_a\dot{\beta}s^2_{az} + m_a\dot{\alpha}Rs_{az}\cos(\beta) + \Theta_{yy,a}\dot{\beta}
\end{align*}
Die notwendigen Zeitableitungen lauten
\begin{align*}
	\frac{\text{d}}{\text{d}t}\frac{\partial L}{\partial \dot{\alpha}} &= m_a\ddot{\alpha}R^2 + m_aR\ddot{\beta}s_{az}\cos(\beta) - m_aR\dot{\beta}^2s_{az}\sin(\beta) + m_r\ddot{\alpha}R^2 + \Theta_{yy,r}\ddot{\alpha} \\
	&= (\Theta_{yy,r} + (m_a + m_r)R^2)\ddot{\alpha} + \left(\cos(\beta)\ddot{\beta} - \sin(\beta)\dot{\beta}^2\right)m_aRs_{az} \\
	\frac{\text{d}}{\text{d}t}\frac{\partial L}{\partial \dot{\beta}} &= m_a\ddot{\beta}s^2_{az} + m_a\ddot{\alpha}Rs_{az}\cos(\beta) - m_a\dot{\alpha}\dot{\beta}Rs_{az}\sin(\beta) + \Theta_{yy,a}\ddot{\beta} \\
	&= (\Theta_{yy,a} + m_as^2_{az})\ddot{\beta} + \cos(\beta)\ddot{\alpha}m_aRs_{az} - m_a\dot{\alpha}R\dot{\beta}s_{az}\sin(\beta)
\end{align*}
Die Form der Euler-Lagrange-Gleichung sind in der Formelsammlung ersichtlich und diese lauten daher
\[
	\begin{bmatrix}
		 (\Theta_{yy,r} + (m_a + m_r)R^2)\ddot{\alpha} + \left(\cos(\beta)\ddot{\beta} - \sin(\beta)\dot{\beta}^2\right)m_aRs_{az} \\
		 (\Theta_{yy,a} + m_as^2_{az})\ddot{\beta} + \cos(\beta)\ddot{\alpha}m_aRs_{az} - mgs_{az}\sin(\beta)
	\end{bmatrix}
	=
	\begin{bmatrix}
		\tau_a + FR \\
		Fl\cos(\beta)
	\end{bmatrix}
\]
f)\\ \\
Mit den angegebenen Bedingungen folgt aus den Bewegungsgleichungen
\[
	\begin{bmatrix}
	\tau_a + FR \\
	Fl\cos(\beta)
	\end{bmatrix}
	=
	\begin{bmatrix}
		0 \\
		- gm_as_{az}\sin(\beta)
	\end{bmatrix}
\]
Hieraus folgt für das gesuchte Moment
\[
	\tau_a = - FR
\]
und den gesuchten Winkel
\begin{align*}
	Fl\cos(\beta) &= - gm_as_{az}\sin(\beta) \\
	-\tan(\beta) &= \frac{Fl}{s_{az}m_ag} \\
	\beta &= - \arctan\left(\frac{Fl}{s_{az}m_ag}1\right)
\end{align*}