\newpage
\noindent
\textbf{Beispiel 4} \\ \\
a) \\ \\
Die zu berechnenden Wärmestromdichten können mit der Formel für Konvektion aus der Formelsammlung des ACIN bestimmt werden. Diese lauten
\begin{align*}
	&\dot{\textbf{q}}_0 = \alpha_0 \left(T_M - T_L\right) , \qquad \dot{\textbf{q}}_{L,2} = \alpha_L \left(T_2 - T_L\right) \\
	&\dot{\textbf{q}}_M = \alpha_M \left(T_M - T_1\right) , \qquad \dot{\textbf{q}}_{L,3} = \alpha_L \left(T_3 - T_L\right) \\
	&\dot{\textbf{q}}_B = \alpha_B \left(T_4 - T_B\right)
\end{align*}
Bei den Ausdrücken, wo laut Formel sich eigentlich ein \(-\) ergibt, wird das Vorzeichen vertauscht, damit keine negativen Wärmestromdichten mehr vorkommen. \\ \\
b) \\ \\
Nun soll die Sichtfaktormatrix \(\textbf{F}\) bestimmt werden.\\
Bestimmung der einzelnen Einträge von \(\textbf{F}\):
\[
	F_{11} = F_{44} = 0
\]
Wand 1 und 4 strahlen thermisch nicht auf sich selbst. Die anderen Einträge der Hauptdiagonalen können mit der Tabelle auf Seite 5 der Formelsammlung des ACIN ermittelt werden und lauten deswegen
\begin{align*}
	F_{22} = F_{33} = 1 -\frac{\sqrt{1^2+2^2}}{1+2} = \frac{3 - \sqrt{5}}{3}
\end{align*}
Die restlichen Einträge werden entweder mit der Cross-String-Methode, dem Reziprozitätsgesetz, der Summationsregel oder durch Symmetrie des Aufbaues ermittelt.
\begin{align*}
	&F_{24} = F_{34} = \frac{3 + \sqrt{5} - \left(1 + \sqrt{5}\right)}{2\cdot3} = \frac{2}{2 \cdot 3} = \frac{1}{3} \\
	&F_{42} = F_{43} = \frac{A_2}{A_4}F_{24} = \frac{3}{3} \frac{1}{3} = \frac{1}{3} \\
	&F_{41} = 1 - F_{42} - F_{43} - F_{44} = 1 - \frac{1}{3} - \frac{1}{3} - 0 = \frac{1}{3} \\
	&F_{14} = \frac{A_4}{A_1}F_{41} = \frac{3}{3} \cdot \frac{1}{3} = \frac{1}{3} \\
	&F_{12} = F_{13} = \frac{\sqrt{5} + 3 - \left(1 - \sqrt{5}\right)}{2\cdot 3} = \frac{2}{2 \cdot 3} = \frac{1}{3} \\
	&F_{21} = F_{31} = \frac{A_1}{A_2}F_{12} = \frac{3}{3}\cdot \frac{1}{3} = \frac{1}{3} \\
	&F_{23} = 1 - F_{21} - F_{22} - F_{24} = 1 - \frac{1}{3} -\frac{3 - \sqrt{5}}{3} - \frac{1}{3} = \frac{\sqrt{5} - 2}{3} \\
	&F_{32} = \frac{A_2}{A_3}F_{23} = \frac{3}{3} \cdot \frac{\sqrt{5} - 2}{3} = \frac{\sqrt{5} - 2}{3}
\end{align*}
\newpage
\noindent
Die komplette Sichtfaktormatrix lautet
\[
	\textbf{F} = \left[\begin{matrix}
	0 & \frac{1}{3} & \frac{1}{3} & \frac{1}{3} \\
	\frac{1}{3} & \frac{3 - \sqrt{5}}{3} & \frac{\sqrt{5} - 2}{3} & \frac{1}{3} \\
	\frac{1}{3} & \frac{\sqrt{5} - 2}{3} & \frac{3 - \sqrt{5}}{3} & \frac{1}{3} \\
	\frac{1}{3} & \frac{1}{3} & \frac{1}{3} & 0
	\end{matrix}\right]
\]
c) \\ \\
Die folgenden Ausdrücke kann man einfach aus der Formelsammlung herauslesen. 
\begin{align*}
	\dot{\textbf{q}} = \left[\begin{matrix}
		\dot{\textbf{q}}_M \\
		-\dot{\textbf{q}}_{L,2} \\
		-\dot{\textbf{q}}_{L,3} \\
		-\dot{\textbf{q}}_B
	\end{matrix}\right]
	\qquad
	\textbf{T}^4 = \left[\begin{matrix}
		T_1^4 \\
		T_2^4 \\
		T_3^4 \\
		T_4^4
	\end{matrix}\right]
	\qquad
	\varepsilon = \left[\begin{matrix}
		\varepsilon \\
		\varepsilon \\
		\varepsilon \\
		\varepsilon
	\end{matrix}\right] \\
	\textbf{P} = diag\{\varepsilon\}\left(\textbf{E} - \left(\textbf{E} - diag\{\varepsilon\}\right)^{-1}\right)\left(\textbf{E}-\textbf{F}\right)\sigma\textbf{T}^4
\end{align*}