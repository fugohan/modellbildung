\newpage
\noindent
\textbf{Beispiel 3} \\ \\
a) \\ \\
%\[q^T=\left[x_W \qu \alpha \qu \varphi \right]\]
\noindent
Begründung: Das System hat 3 Freiheitsgrade, daher fallen i und iv schon mal weg. In der Skizze ist zu sehen das \(x_W\) nicht konstant ist und daher lautet meine Antwort iii. \\ \\ b) \\ \\ \noindent
Um die Lage des kombinierten Schwerpunktes zu bestimmen multipliziert man jeweils den Ort des Schwerpunktes mit der Masse des jeweiligen Schwerpunktes und dividiert diese mit der Gesamtmasse.
\[l_{SP} = \frac{\frac{l_s}{2}m_s+l_s\,3\,m_s}{4\,m_s} = \frac{7\,l_s}{8}\]

\noindent
Mittels dem Satz vom Steiner lässt sich das neue Massenträgheitsmoment, wie folgt, bestimmen:
\[\theta_{SP} = \theta_S+3 m_s \left(\frac{49}{64}l_s^2 \right)\] %TODO: Laut Lösung 3/16 und nicht 3*49/64
c) \\\\
\noindent
Die potentielle Energie in diesem System besteht aus zwei Komponenten. Zunächst wäre da mal der Feder-Anteil der aus \(V_{c1}(q) = \frac{c}{2}(x_W - \alpha R)^2 \) besteht und die andere Komponente ist die potentielle Energie im Stab. Zusammen ergibt sich daraus die gesamte potentielle Energie: \[V(q) = m_{SP}\, g\, \cos{(\varphi)}\, l_{SP} + \frac{c}{2} (x_W - \alpha R)^2\]

\noindent d) \\\\

\noindent
Die kinetische Energie des Systems besteht aus dem Translatorischen-Anteil und dem Rotatorischen-Anteil.

\[T(q,\dot{q}) = \frac{m_W}{2} \dot{x}_w^2 + \theta_R \dot{alpha}^2 + \frac{m_{SP}}{2} \dot{r}_{SP}^T\dot{r}_{SP} + \frac{\theta_{SP}}{2}\dot{\varphi}^2 \]

\[r_{SP} = \begin{bmatrix} x_W + \sin{(\varphi)}\ l_{SP}  \\ \cos{(\varphi)}\ l_{SP}\end{bmatrix},\quad  \dot{r}_{SP}=\begin{bmatrix}
  \dot{x_W} + \dot{\varphi}\ \cos{(\varphi)}\ l_{SP}  \\ -\dot{\varphi}\ \sin{(\varphi)}\ l_{SP}
\end{bmatrix}\]
e) \\\\
\noindent
Um die Generalisierte Kräfte zu bestimmen muss man zunächst einmal die äußere Störkraft \(F_d\) bestimmen. Diese lautet \(F_d=-f_d \begin{bmatrix}
  1 \\ 0
\end{bmatrix}\) Außerdem braucht man die Ableitungen des Ortes der äußeren Störkraft nach den generalisierten Koordininaten:
\[ \frac{\partial r_S}{\partial x_W} = \begin{bmatrix}
  1 \\ 0
\end{bmatrix}, \quad \frac{\partial r_S}{\partial \alpha }  = \begin{bmatrix}
  0 \\ 0
\end{bmatrix}, \quad \frac{\partial r_S}{\partial \varphi} = \begin{bmatrix}
  \cos(\varphi)\ l_S \\ - \sin(\varphi)\ l_S
\end{bmatrix}\]
Mit der Formel aus der Formelsammlung folgt:
\[f_q = \sum_{i=0}^N F_D \frac{\partial r_S}{\partial q_i} = \begin{bmatrix}
  -f_d \\ \tau_m \\ -\cos(\varphi)\ f_d
\end{bmatrix}\]
