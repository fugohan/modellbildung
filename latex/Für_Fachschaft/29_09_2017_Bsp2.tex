\textbf{Beispiel 2}\\ \\
a)\\ \\
Dieses System wird durch einen Freiheitsgrad komplett beschrieben, z.B der Kurbelwinkel $\alpha$. \\ \\
b)\\ \\
Als geeigneter Freiheitsgrad wurde der Winkel $\alpha$ gewählt. Es gilt die Zwangsbeziehung
\[
	R_1\sin(\alpha) = R_2\sin(\beta)
\]
Dadurch lautet der Winkel $\beta$
\[
	\beta = \arcsin\left(\frac{R_1}{R_2}\sin(\alpha)\right)
\]
c)\\ \\
Die Ortsvektoren zu den Körperschwerpunkten lauten für die Kurbelwelle 
\[
	\textbf{r}_{S1} = \begin{bmatrix}
		e\cos(\alpha) \\
		e\sin(\alpha) \\ 
		0
	\end{bmatrix}
\]
die Pleuelstange
\[
	\textbf{r}_{S2} = \begin{bmatrix}
		R_1\cos(\alpha) + \frac{R_2}{2}\cos(\beta(\alpha)) \\
		\frac{R_2}{2}\sin(\beta(\alpha))
	\end{bmatrix}
\]
und der für den Kolben
\[
	\textbf{r}_K = \begin{bmatrix}
		R_1\cos(\alpha) + R_2\cos(\beta(\alpha)) \\
		0 \\
		0		
	\end{bmatrix}
\]
\newpage
\noindent
Die Geschwindigkeiten lauten für die Kurbelwelle
\[
	\textbf{v}_{S1} = \begin{bmatrix}
		-e\sin(\alpha)\dot{\alpha }\\
		e\cos(\alpha)\dot{\alpha} \\ 
		0
	\end{bmatrix}
	\qquad
	\omega_{S1} = \dot{\alpha}
\]
die Pleuelstange
\[
	\textbf{v}_{S2} = \begin{bmatrix}
		-R_1\sin(\alpha)\dot{\alpha} - \frac{R_2}{2}\sin(\beta)\frac{d\beta}{d\alpha}\dot{\alpha} \\
		\frac{R_2}{2}\cos(\beta)\frac{d\beta}{d\alpha}\dot{\alpha} \\
		0
	\end{bmatrix}
	\qquad
	\omega_{S2} = \frac{d\beta}{d\alpha}\dot{\alpha} 
\]
und für den Kolben
\[
	\textbf{v}_K = \begin{bmatrix}
		-R_1\sin(\alpha)\dot{\alpha} - R_2\sin(\beta)\frac{d\beta}{d\alpha}\dot{\alpha} \\
		0 \\
		0
	\end{bmatrix}
	\qquad
	\omega_K = 0
\]
e)\\ \\
Die kinetischen Teilenergien lauten hier
\[
	T_{rot} = \frac{1}{2}\left(\varTheta_1\dot{\alpha}^2 + \varTheta_2\left(\frac{d\beta}{d\alpha}\dot{\alpha}\right)^2\right)
\]
und 
\[
	T_{trans} = \frac{1}{2}(m_1\textbf{v}_{S1}^T\textbf{v}_{S1} + m_2\textbf{v}_{S2}^T\textbf{v}_{S2} + m_K\textbf{v}_K^T\textbf{v}_K)
\]
Daher ergibt sich für die gesamte kinetische Energie
\[
	T = T_{rot} + T_{trans}
\]
f)\\ \\
Der Ortsvektor zum Angriffspunkt der externen Kraft lautet
\[
	\textbf{r}_{F_p} = \begin{bmatrix}
	R_1\cos(\alpha) + R_2\cos(\beta(\alpha)) \\
	0 \\
	0		
	\end{bmatrix}
\]
Der Vektor der externen Kraft lautet
\[
	\textbf{f}_{F_p} = \begin{bmatrix}
		-F_p \\
		0 \\
		0
	\end{bmatrix}
\]
Die notwendige Ableitung lautet
\[
	\frac{\partial\textbf{r}_{F_p} }{\partial \alpha} = \begin{bmatrix}
	-R_1\sin(\alpha)\dot{\alpha} - R_2\sin(\beta)\frac{d\beta}{d\alpha} \\
	0 \\
	0
	\end{bmatrix}
\]
Mit der Berechnungsvorschrift für den gesuchten Vektor aus der Formelsammlung, ergibt sich für diesen
\[
	Q = F_p\left(R_1\sin(\alpha) + R_2\sin(\beta)\frac{d\beta}{d\alpha}\right) 
\]