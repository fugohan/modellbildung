\newpage
\noindent
\textbf{Beispiel 3}\\ \\
a)\\ \\
Die benötigte Formel kann aus der Formelsammlung entnommen werden. Die gesuchte Größe lautet somit
\[
	\dot{q}^\circ = (T_l - T_\infty)\frac{2\pi}{\underbrace{\frac{1}{r_i\alpha_i} + \frac{1}{\lambda_{isol} \ln\frac{r_O}{r_i}} + \frac{1}{r_O\alpha_O}}_{:= k^\circ}}
\]
b)\\ \\
Da die gegebenen Leitungsparameter homogen verteilt sind gilt 
\[
	\int_{\mathcal{V}}\rho_lc_{p,l}\frac{\text{d}T_l}{\text{d}t}\text{d}\mathcal{V} = r_i^2\pi\rho_lc_{p,l}L\frac{\text{d}T_l}{\text{d}t}
\]
Die zugeführte elektrische Leistung lautet
\[
	P_{el} = I^2R'L
\]
und der Wärmestrom lautet
\[
	\dot{Q} = -\dot{q}^\circ L
\]
Durch Einsetzen in den gegebenen Energieerhaltungssatz erhält man die Differentialgleichung
\begin{align*}
	r_i^2\pi\rho_lc_{p,l}&L\frac{\text{d}T_l}{\text{d}t} = -\dot{q}^\circ L + I^2R'L \\
	\frac{\text{d}T_l}{\text{d}t} &= \frac{-\dot{q}^\circ + I^2R'}{r_i^2\pi\rho_lc_{p,l}}
\end{align*}
c)\\ \\
Im stationären Fall fällt die zeitliche Ableitung fällt, somit ergibt sich für gesuchte Temperatur
\begin{align*}
	0 &= -(T_l - T_\infty)k^\circ + I^2R' \\
	T_l &= T_\infty + \frac{I^2R'}{k^\circ}
\end{align*}