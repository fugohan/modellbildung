\textbf{Beispiel 1}\\ \\
a)\\ \\
Die Massen der beiden Starrkörper lauten
\begin{align*}
	m_1 &= \rho bhl_1 \\
	m_2 &= \rho bhl_2
\end{align*}
Die Lagen der einzelnen Schwerpunkte lauten
\begin{align*}
	l_{1,S} &= \frac{l_1}{2} \\
	l_{2,S} &= \frac{l_2}{2}
\end{align*}
b)\\ \\
Die Ortsvektoren zu diesen Schwerpunkten lauten
\begin{align*}
	\textbf{r}_{S,1} &= \begin{bmatrix}
		x - \frac{\sqrt{l_1^2 - y^2}}{2} \\
		\frac{y}{2}
	\end{bmatrix}
	\\
	\textbf{r}_{S,2} &= \begin{bmatrix}
	x + \frac{\sqrt{l_1^2 - y^2}}{2} \\
	\frac{y}{2}
	\end{bmatrix}
\end{align*}
Die dazugehörigen Geschwindigkeiten lauten
\begin{align*}
	\dot{\textbf{r}}_{S,1} &= \begin{bmatrix}
		\dot{x} + \frac{y\dot{y}}{2\sqrt{l_1^2 - y^2}} \\
		\frac{\dot{y}}{2}
	\end{bmatrix}
	\\
	\dot{\textbf{r}}_{S,2} &= \begin{bmatrix}
		\dot{x} - \frac{y\dot{y}}{2\sqrt{l_1^2 - y^2}} \\
		\frac{\dot{y}}{2}
	\end{bmatrix}
\end{align*}
c) \\ \\
Die kinetische Energie des Systems lautet
\[
	T(\dot{q},\dot{\textbf{q}}) = \frac{1}{2}m_1\dot{\textbf{r}}_{S,1}\dot{r}_{S,1} + \frac{1}{2}m_2\dot{\textbf{r}}_{S,2}\dot{r}_{S,2} +\frac{1}{2}\theta_{1,zz}^{(S)}\dot{\alpha}_1^2 + \frac{1}{2}\theta_{2,zz}^{(S)}\dot{\alpha}_2^2
\]
mit den Winkel
\begin{align*}
	\alpha_1 &= \arcsin\left(\frac{y}{l_1}\right) \\
	\alpha_2 &= \arcsin\left(\frac{y}{l_2}\right)
\end{align*}
und dessen Ableitungen
\begin{align*}
	\dot{\alpha}_1 &= \frac{\dot{y}}{\sqrt{l_1^2 - y^2}} \\
	\dot{\alpha}_2 &= \frac{\dot{y}}{\sqrt{l_2^2 - y^2}}
\end{align*}
d)\\ \\
Die potentielle Energie lautet hier
\[
	V(\textbf{q}) = m_1g\frac{y}{2} + m_sg\frac{y}{2} + \frac{1}{2}c_1s_1^2 + \frac{1}{2}c_2s_2^2
\]
mit den Längen
\begin{align*}
	s_1 &= x - \sqrt{l_1^2 - y^2} \\
	s_2 &= L - x - \sqrt{l_1^2 - y^2}
\end{align*}
e) \\ \\ 
Da man nun den stationären Fall betrachten soll gilt
\[
	\frac{\partial V}{\partial \dot{q}} = \tau
\]
Die ausgewertete potentielle Energie lautet
\[
	V(\textbf{q}) = m_1g\frac{y}{2} + m_sg\frac{y}{2} + \frac{1}{2}c_1\left(x - \sqrt{l_1^2 - y^2}\right)^2 + \frac{1}{2}c_2\left(L - x - \sqrt{l_1^2 - y^2}\right)^2
\]
\newpage
\noindent
Die Ableitungen lauten
\begin{align*}	
	\frac{\partial V(\textbf{q})}{\partial x} &= c_1\left(x - \sqrt{l_1^2 - y^2}\right) - c_2\left(L - x - \sqrt{l_2^2 - y^2}\right) \\
	\frac{\partial V(\textbf{q})}{\partial y} &= \frac{1}{2}m_1g + \frac{1}{2}m_2g - c_1\left(x - \sqrt{l_1^2 - y^2}\right)\frac{y}{\sqrt{l_1^2 - y^2}} \\
	& + c_2\left(L - x - \sqrt{l_1^2 - y^2}\right)\frac{y}{\sqrt{l_2^2 - y^2}} 
\end{align*}
Damit ergeben sich für die Postion $x$ und die Kraft $F_y$
\begin{align*}
	x &= \frac{c_1\sqrt{l_1^2 - y^2} + c_2(L - \sqrt{l_2^2 - y^2})}{c_1 + c_2} \\
	F_y &= \frac{1}{2}m_1g + \frac{1}{2}m_2g - c_1\left(x - \sqrt{l_1^2 - y^2}\right)\frac{y}{\sqrt{l_1^2 - y^2}} \\
		& + c_2\left(L - x - \sqrt{l_1^2 - y^2}\right)\frac{y}{\sqrt{l_2^2 - y^2}} 
\end{align*}