\textbf{Beispiel 3}\\ \\
a)\\ \\
Die Leistungsbilanz für das elektronische Bauteil lautet
\[
	-\dot{q}_s + g_0h - \dot{q}_b = 0
\]
und daraus folgt
\[
	\dot{q}_s = g_0h - \dot{q}_b
\]
b) \\ \\
Mit der Formel
\[
	\dot{q} = \alpha (T - T_\infty)
\]
folgt für die Oberflächentemperatur
\[
	T_s = T_\infty + \frac{\dot{q}_s}{\alpha}
\]
\newpage
\noindent
c)\\ \\
Um die Differentialgleichung bestimmen zu können muss man die Wärmeleitgleichung für die kartesischen Koordinaten hernehmen. Da sich die Wärme nur in z-Richtung ausbreitet muss auch nur dieser Term der Formel verwendet werden.
\[
	\lambda\frac{d^2T}{dz^2} + g_0 = 0
\]
Mit den Randbedingungen
\begin{align*}
	\frac{dT}{dz}(0) &= \frac{\dot{q}_b}{\lambda} \\
	T(h) &= T_s
\end{align*}
d) \\ \\
Für den Temperaturverlauf im Bauteil muss jetzt die Differentialgleichung gelöst werden.
\begin{align*}
		\lambda\frac{d^2T}{dz^2} + g_0 = 0 \\
		\frac{d^2T}{dz^2} = -\frac{g_0}{\lambda} \\
		\frac{dT}{dz} = -\frac{g_0}{\lambda}z + C_1 \\
		T(z) = -\frac{g_0}{\lambda}\frac{z^2}{2} + C_1z + C_2
\end{align*}
Durch Einsetzen der Randbedingungen folgt für die Konstanten
\begin{align*}
	\frac{dT}{dz} &= 0 + C_1 = \frac{\dot{q}_b}{\lambda} \\
	C_1 &= \frac{\dot{q}_b}{\lambda} 
\end{align*}
und
\begin{align*}
	T_s &= -\frac{g_0}{\lambda}\frac{h^2}{2} + \frac{\dot{q}_b}{\lambda}h + C_2 \\
	C_2 &= T_s + \frac{g_0}{\lambda}\frac{h^2}{2} - \frac{\dot{q}_b}{\lambda}h
\end{align*}
Jetzt noch die Konstanten einsetzen und man erhält schließlich
\begin{align*}
	T(z) &= -\frac{g_0}{\lambda}\frac{z^2}{2} + \frac{\dot{q}_b}{\lambda}z + T_s + \frac{g_0}{\lambda}\frac{h^2}{2} - \frac{\dot{q}_b}{\lambda}h \\
	&= T_s \frac{1}{2}\frac{\dot{q}_b}{\lambda}(h^2 - z^2) + \frac{\dot{q}_b}{\lambda}(z - b)
\end{align*}
\newpage
\noindent
e) \\ \\
Die maximale Position lautet
\[
	z_{max} = \frac{\dot{q}_b}{g_0}
\]
Für die maximale Temperatur existier leider keine Lösung zum vergleichen. \\ \\