\newpage
\noindent
\textbf{Beispiel 3}\\ \\
a)\\ \\
Das gewünschte Massenträgheitsmoment lautet
\begin{align*}
	\theta_h &= \int_{-\frac{d_H}{2}}^{\frac{d_H}{2}}\int_{-\frac{\varphi_H}{2}}^{\frac{\varphi_H}{2}}\int_{r_i}^{r_a}\rho r^3\,\text{d}r\text{d}\varphi\text{d}z \\
	&= \int_{-\frac{d_H}{2}}^{\frac{d_H}{2}}\int_{-\frac{\varphi_H}{2}}^{\frac{\varphi_H}{2}}\frac{1}{4}\rho(r_a^4 - r_i^4)\,\text{d}\varphi\text{d}z \\
	&=  \int_{-\frac{d_H}{2}}^{\frac{d_H}{2}}\frac{1}{4}\rho(r_a^4 - r_i^4\varphi_H\,\text{d}z \\
	&=\frac{1}{4}\rho(r_a^4 - r_i^4\varphi_Hd_H
\end{align*}
b) \\ \\
Durch die Rechenvorschrift zur Ermittelung eines Schwerpunktes aus dem Skript folgt für den gesuchten Abstand
\[
	r_p = \frac{2}{3}\frac{r_a^3 - r_i^3}{r_a^2 - r_i^2}\frac{\sin\left(\frac{\varphi_H}{2}\right)}{\frac{\varphi_H}{2}}
\]
c)\\ \\
Die gesuchten Ortsvektoren lauten
\[
	\textbf{r}_c = \begin{bmatrix}
		x_C \\
		0 \\
		0
	\end{bmatrix}
	\quad,\quad
	\textbf{r}_p = \begin{bmatrix}
		l_x + r_p\sin(\alpha) \\
		l_y - r_p\cos(\alpha) \\
		0
	\end{bmatrix}
\]
Die entsprechenden Geschwindigkeitsvektoren lauten
\[
	\dot{\textbf{r}}_c = \begin{bmatrix}
		\dot{x}_c \\
		0 \\
		0		
	\end{bmatrix}
	\quad,\quad
	\dot{\textbf{r}}_p = \begin{bmatrix}
		r_p\cos(\alpha)\dot{\alpha} \\
		r_p\sin(\alpha)\dot{\alpha} \\
		0
	\end{bmatrix}
\]
d) \\ \\
Die kinetische Energie des Systems lautet hier
\[
	T_{kin} = \frac{1}{2}m_c\dot{x}^2_c \frac{1}{2}\theta_p\dot{\alpha}^2
\]
\newpage
\noindent
e)\\ \\
Bei der potentiellen Energie berücksichtigt man die des Hammer und die der Federn. Diese lauten
\begin{align*}
	V_h &= m_pgr_p(1 - \cos(\alpha)) \\
	V_1 &= \frac{1}{2}c_{f,1}\left(x_c - \frac{l_L}{3} - l_{0,1}\right)^2 \\
	V_2 &= \frac{1}{2}c_{f,2}\left(\sqrt{(\textbf{p}_p - \textbf{p}_c)^T(\textbf{p}_p - \textbf{p}_c)} - l_{0,2}\right)^2 \\
	&= \frac{1}{2}c_{f,2}\left(\sqrt{(l_x - x_c)^2 + l_y^2 + r_p^2 + 2r_p((l_x - x_c)\sin(\alpha) - l_y\cos(\alpha))}\right)
\end{align*}
f)\\ \\
Der Vektor der externen Kraft
\[
	\textbf{F} = F\begin{bmatrix}
		\cos(\delta) \\
		\sin(\delta) \\
		0
	\end{bmatrix}
\]
Der Ortsvektor zum Angriffspunkt lautet
\[
	\textbf{r}_f = \begin{bmatrix}
		l_x + r_i\sin\left(\alpha + \frac{\varphi_H}{2}\right) \\
		l_y - r_i\cos\left(\alpha + \frac{\varphi_H}{2}\right) \\
		0		
	\end{bmatrix}
\]
Die notwendigen Ableitungen lauten
\[
	\frac{\partial \textbf{r}_f}{\partial x_c} = \begin{bmatrix}
		0 \\
		0 \\
		0
	\end{bmatrix}
	\quad,\quad
		\frac{\partial \textbf{r}_f}{\partial \alpha} = \begin{bmatrix}
	r_i\cos\left(\alpha + \frac{\varphi_H}{2}\right) \\
	r_i\sin\left(\alpha + \frac{\varphi_H}{2}\right)\\
	0
	\end{bmatrix}
\]
Somit ergibt sich mit der Rechenvorschrift aus der Formelsammlung für die generalisierten Kräfte
\[
	\tau = \begin{bmatrix}
		0 \\
		-F\cos(\delta)r_i\cos\left(\alpha + \frac{\varphi_H}{2}\right) - F\sin(\delta)r_i\sin\left(\alpha + \frac{\varphi_H}{2}\right)
	\end{bmatrix}
\]